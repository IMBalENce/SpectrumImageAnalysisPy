\documentclass[12pt]{article}

\usepackage{amsmath, mathtools}
\usepackage{amsfonts}
\usepackage{amssymb}
\usepackage{graphicx}
\usepackage{colortbl}
\usepackage{xr}
\usepackage{hyperref}
\usepackage{longtable}
\usepackage{xfrac}
\usepackage{tabularx}
\usepackage{float}
\usepackage{siunitx}
\usepackage{booktabs}
\usepackage{caption}
\usepackage{pdflscape}
\usepackage{afterpage}

%\usepackage{refcheck}

\hypersetup{
    bookmarks=true,         % show bookmarks bar?
      colorlinks=true,       % false: boxed links; true: colored links
    linkcolor=red,          % color of internal links (change box color with
%linkbordercolor)
    citecolor=green,        % color of links to bibliography
    filecolor=magenta,      % color of file links
    urlcolor=cyan           % color of external links
}

%% Comments

\usepackage{color}

\newif\ifcomments\commentstrue

\ifcomments
\newcommand{\authornote}[3]{\textcolor{#1}{[#3 ---#2]}}
\newcommand{\todo}[1]{\textcolor{red}{[TODO: #1]}}
\else
\newcommand{\authornote}[3]{}
\newcommand{\todo}[1]{}
\fi

\newcommand{\wss}[1]{\authornote{blue}{SS}{#1}}
\newcommand{\an}[1]{\authornote{magenta}{Author}{#1}}


% For easy change of table widths
\newcommand{\colZwidth}{1.0\textwidth}
\newcommand{\colAwidth}{0.13\textwidth}
\newcommand{\colBwidth}{0.82\textwidth}
\newcommand{\colCwidth}{0.1\textwidth}
\newcommand{\colDwidth}{0.05\textwidth}
\newcommand{\colEwidth}{0.8\textwidth}
\newcommand{\colFwidth}{0.17\textwidth}
\newcommand{\colGwidth}{0.5\textwidth}
\newcommand{\colHwidth}{0.28\textwidth}

% Used so that cross-references have a meaningful prefix
\newcounter{defnum} %Definition Number
\newcommand{\dthedefnum}{GD\thedefnum}
\newcommand{\dref}[1]{GD\ref{#1}}
\newcounter{datadefnum} %Datadefinition Number
\newcommand{\ddthedatadefnum}{DD\thedatadefnum}
\newcommand{\ddref}[1]{DD\ref{#1}}
\newcounter{theorynum} %Theory Number
\newcommand{\tthetheorynum}{T\thetheorynum}
\newcommand{\tref}[1]{T\ref{#1}}
\newcounter{tablenum} %Table Number
\newcommand{\tbthetablenum}{T\thetablenum}
\newcommand{\tbref}[1]{TB\ref{#1}}
\newcounter{assumpnum} %Assumption Number
\newcommand{\atheassumpnum}{P\theassumpnum}
\newcommand{\aref}[1]{A\ref{#1}}
\newcounter{goalnum} %Goal Number
\newcommand{\gthegoalnum}{P\thegoalnum}
\newcommand{\gsref}[1]{GS\ref{#1}}
\newcounter{instnum} %Instance Number
\newcommand{\itheinstnum}{IM\theinstnum}
\newcommand{\iref}[1]{IM\ref{#1}}
\newcounter{reqnum} %Requirement Number
\newcommand{\rthereqnum}{P\thereqnum}
\newcommand{\rref}[1]{R\ref{#1}}
\newcounter{lcnum} %Likely change number
\newcommand{\lthelcnum}{LC\thelcnum}
\newcommand{\lcref}[1]{LC\ref{#1}}

\newcommand{\progname}{SpectrumImageAnalysisPy} % PUT YOUR PROGRAM NAME HERE

\usepackage{fullpage}

\begin{document}
\bibliographystyle{ieeetr}
\title{SpectrumImageAnalysisPy} 
\author{Isobel Bicket}
\date{\today}
	
\maketitle

~\newpage

\pagenumbering{roman}


\section{Revision History}

\begin{tabularx}{\textwidth}{p{4cm}p{2cm}X}
	\toprule {\bf Date} & {\bf Version} & {\bf Notes}\\
	\midrule
	October 4, 2017  & 1.0 & First draft\\
	\today & 1.1 & Edits to first draft\\
	\bottomrule
\end{tabularx}

~\newpage

\tableofcontents

\section{Reference Material}

This section contains reference information to enable the reader to better
understand the document.

\subsection{Table of Units}

Throughout this document the system of units standard to the field of electron
microscopy is used
as the unit system, for consistency with relevant literature and practices. For
each unit, the symbol is given followed by a
description of the quantity described by the symbol and the unit name.
~\newline

\renewcommand{\arraystretch}{1.2}
%\begin{table}[ht]
  \noindent \begin{tabular}{l l l} 
    \toprule		
    \textbf{symbol} & \textbf{quantity} & \textbf{unit name}\\
    \midrule 
    counts & intensity & number of particles\\
    \si{\nano\metre} & length & nanometre\\
    \si{\electronvolt} & energy	& electron volt\\
    \bottomrule
  \end{tabular}
  %	\caption{Provide a caption}
%\end{table}

\wss{Is ``number of particles'' a standard measure of intensity?  I did a quick
  google search and I did not see any examples of intensity measured by the
  number of particles.}

\subsection{Table of Symbols}

The table that follows summarizes the symbols used in this document along with
their units. The symbols are listed in alphabetical order.

\renewcommand{\arraystretch}{1.2}
%\noindent \begin{tabularx}{1.0\textwidth}{l l X}
\noindent \begin{longtable*}{l l p{12cm}} \toprule
\textbf{symbol} & \textbf{unit} & \textbf{description}\\
\midrule 
$A^2$ & \si{\nano\metre^2} & Area of interest on the sample\\
$b(k)$ & counts & Background signal on a detector\\
$c$ & -- & Index of the current iteration in deconvolution \\
$E$ & \si{\nano\metre} or \si{\electronvolt} & Energy of the collected particle,
represented by wavelength (CL) or energy loss (EELS)\\
$f(\rho')$ & -- & Probability of an electron landing on the detector at position
$\rho'$\\
$f(k')$ & -- & Probability of an electron landing on the detector at pixel
$k'$\\
$g(k)$ & -- & Gain value at each pixel\\
$h(\rho, \rho')$ & -- & Probability of detecting an electron at position $\rho$
if it hits position $\rho'$\\
$h(k, k')$ & -- & Probability of detecting an electron at pixel $k$ if it hits
pixel $k'$\\
$i$ & -- & Index for pixel number in a spectrum\\
$I(E)$ & counts & Intensity of a spectrum as a function of energy\\
$I_{background}(E)$ & counts & Intensity of a background contribution to the
spectrum (from the spectrometer), as a function of energy\\
$I_{collected}(E)$ & counts & Intensity of a spectrum as received from the
spectrometer, as a function of energy\\
$I_{corrected}(E)$ & counts & Intensity of a spectrum from the sample with gain
artifacts removed as a function of energy\\
$I_{deconvolved}(E)$ & counts & Intensity of a deconvolved spectrum as a
function of energy\\
$I_{measured}(E)$ & counts & Intensity of a measured spectrum as a function of
energy\\
$I_{norm}(E)$ & counts & Intensity of a normalized spectrum as a function of
energy\\
$I_{PSF}(E)$ & counts & Intensity of a reference point spread function
(spectrum) as a function of energy\\
$I_{real}^0(E)$ & counts & Initial guess for a ``real'' spectrum for
deconvolution, as a function of energy\\
$I_{real (j)}^R(E)$ & counts & Intensity for a deconvolved spectrum after R
deconvolution iterations\\
$I_{sample}(E)$ & counts & Intensity of a spectrum from the sample with
background artifacts removed as a function of energy\\
$j$ & -- & Index for pixel number in a spectrum\\
$k$ & -- & Pixel (index) on a spectrometer\\
$k_1$ & -- & First pixel (index) on a spectrometer (user selected)\\
$k_2$ & -- & Second pixel (index) on a spectrometer (user selected)\\
$K$ & -- & Maximum pixel (index) on a spectrometer\\
$l$ & -- & Index for pixel number in a spectrum\\
$N(k)$ & counts & Noise signal from the detector\\
$P(x,y)$ & \si{\nano\metre} & Position coordinates describing possible locations
of the electron beam on the sample\\
$R$ & -- & Integer number of deconvolution iterations the user desires the
software to perform\\
$S(k)$ & counts & Signal entering the detector\\
$S'(k)$ & counts & Signal entering the detector, resulting from a beam which has
not interacted with the sample\\
$x$ & \si{\nano\metre} & Displacement of the beam in one spatial dimension\\
$X$ & \si{\nano\metre} & Maximum displacement of the beam in one spatial
dimension\\
$y$ & \si{\nano\metre} & Displacement of the beam in one spatial dimension,
perpendicular to $x$\\
$Y$ & \si{\nano\metre} & Maximum displacement of the beam in one spatial
dimension, perpendicular to $x$\\
$\rho$ & -- & Location of detection of an electron hitting a detector\\
$\rho'$ & -- & Location of an electron hitting a detector\\
$\zeta(k)$ & counts & Signal at each pixel as read out by a detector\\
\bottomrule
\end{longtable*}

\subsection{Abbreviations and Acronyms}

\renewcommand{\arraystretch}{1.2}
\begin{tabular}{l l} 
  \toprule		
  \textbf{symbol} & \textbf{description}\\
  \midrule 
  3D & Three-dimensions\\
  A & Assumption\\
  CCD & Charge-Coupled Device\\
  CL & Cathodoluminescence Spectroscopy\\
  DD & Data Definition\\
  EELS & Electron Energy Loss Spectroscopy\\
  GD & General Definition\\
  GS & Goal Statement\\
  IM & Instance Model\\
  LC & Likely Change\\
  PS & Physical System Description\\
  PSF & Point Spread Function\\
  R & Requirement\\
  SI & Spectrum Image\\
  SRS & Software Requirements Specification\\
  STEM & Scanning Transmission Electron Microscope/Microscopy\\
  SEM & Scanning Electron Microscope/Microscopy\\
  TEM & Transmission Electron Microscope/Microscopy\\
  T & Theoretical Model\\
  \bottomrule
\end{tabular}\\

\newpage
\pagenumbering{arabic}

\section{Introduction}
Three-dimensional datasets are produced by many spectro-microscopy techniques:
techniques in which the response of a sample is mapped in two spatial domains
and one spectral dimension. The visualization of 3D spectro-microscopy datasets
is a challenging and often non-intuitive task. The motivation behind this
software is to enable easy visualization and navigation through a 3D dataset to
discover the features of the sample in both spatial and spectral domains. The
data processing methods of 3D datasets are not trivial to perform and require a
dedicated software designed to handle 3D data. This document provides the
Software Requirements Specification (SRS) for a software designed to visualize
and process 3D spectro-microscopy datasets derived from electron energy loss
spectroscopy or cathodoluminescence.

\subsection{Purpose of Document}

This document details the requirements of the software \progname{}. The
responsibilities of the user and software are laid out and the requirements that
the software must satisfy are explicitly detailed.

\subsection{Scope of Requirements} 
The scope of the requirements for the software \progname{} is limited to the
import; removal of instrumentation artifacts; visualization and navigation
through \wss{Do you want the word through here?}; and the export of EELS and CL spectrum imaging data.

\subsection{Characteristics of Intended Reader} 
The reader of this document should have an understanding of spectrum imaging
techniques, particularly EELS and CL, and the data processing methods used to
remove effects of the acquisition system from the data acquired. A basic
knowledge of convolution theory will be helpful to the reader and an
understanding of the characteristics of 3D datasets. The reader should
understand EELS and CL, as relevant to the sections of the document.  \wss{Can
  you also relate these characteristics to a standard degree and level?  For
  instance, are you assuming an undergraduate degree in physics?}

\subsection{Organization of Document}

The document follows the organizational scheme laid out by Smith \textit{et al.}
\cite{SmithAndLai2005, smith_requirements_2007}, starting with general reference
material for the reader. The introduction of the document details the purpose of
the document and intended reader, and is followed by the General System
Description. The Specific System Description reduces the problem under
consideration to a set of clearly defined definitions and models. The
Requirements section of the document clearly outlines both the Functional and
Non-functional Requirements that the software must comply with. Finally, the
document contains a list of Likely Changes to the software and Traceability
matrices and graphs.

\section{General System Description}

This section identifies the interfaces between the system and its environment
and
describes the user characteristics and lists the system constraints.

\subsection{System Context}

Use of the software \progname{} confers the following responsibilities to the
part of the user and to the program.

\begin{itemize}
	\item User Responsibilities:
	\begin{itemize}
		\item Provide the correct input data to the program;
		\item Be capable of interacting with the software via a mouse, keyboard, and
screen;
		\item Have the necessary dependencies for the program installed;
		\item Possess sufficient knowledge on what processing steps need to be
performed and be able to judge that the quality of the output data is sufficient
for the application, or change the processing steps performed.
	\end{itemize}
	\item \progname{} Responsibilities:
	\begin{itemize}
		\item Read inputs (either files or data arrays) and check the inputs for the
correct data type and size;
		\item Display the data and graphical user interface for the user to interact
with;
		\item Respond to user commands and accept user input as appropriate, including
visualization and data processing commands;
		\item Export data in the correct file format(s).
	\end{itemize}
\end{itemize}

\subsection{User Characteristics} \label{SecUserCharacteristics}

The end user of \progname{} should be familiar with the concept of a spectrum
image and an understanding of what the data represents and the appropriate
actions needed to process spectrum images and extract useful information. A
basic familiarity with programming is expected. An
understanding of the spectro-microscopy technique (CL or EELS) used to generate
the data will
be beneficial, but is not strictly required. 

\subsection{System Constraints}

The software must be able to read .dm3 files for EELS data import, and .h5 files
for CL data import to be able to interface with output from the data acquisition
software.  \wss{Great example of a constraint.}

\section{Specific System Description}

This section first presents the problem description, which gives a high-level
view of what the program is expected to do. The solution characteristics
specification follows, detailing the assumptions that the software will make;
the theoretical models the data analysis is based on; definitions for the types
of data that the software must be able to process; and the instance models,
which detail the data processing algorithms that the software will implement.

\subsection{Problem Description} \label{Sec_pd}

\progname{} is a software to allow users to import, process, navigate, and
export 3D spectrum image data. 

\subsubsection{Terminology and  Definitions}

This subsection provides a list of terms that are used in the subsequent
sections and their meaning, with the purpose of reducing ambiguity and making it
easier to correctly understand the requirements:

\begin{itemize}
	\item Beam: the probe used by an electron microscope, composed of electrons
accelerated to high speeds;
	\item Beam current: the number of electrons travelling in the beam per unit
time;
	\item Cathodoluminescence: a technique in a STEM or SEM in which light emitted
from the sample is collected by a mirror and directed to a spectrometer or other
analysis system;
	\item Charge-Coupled Device: the camera used in the spectrometer to collect
signal and output it to the microscope acquisition software;
	\item Electron energy loss spectroscopy: a technique performed in a TEM or STEM
which detects the energy lost by an electron beam after passing through a
sample;
	\item Point Spread Function (PSF): a function representing the spreading in the
spectrometer, as defined in \ddref{PSF};
	\item Sample: a physical sample of scientific interest that has been probed by
the beam in the EELS or CL instrument; 
	\item Scanning transmission electron microscope: a TEM with the capability of
scanning the electron beam in a raster scan over the sample;
	\item Slice: a subset of the dataset;
	\item Spectrometer: an instrument attached to a microscope and equipped with a
CCD or equivalent camera, used for collecting signal as a function of energy;
	\item Scanning electron microscope: a microscope which uses an electron beam to
probe the sample and acquires data based on scattering off the surface or near
the surface of the sample, operating using a raster scan of the electron beam;
	\item Spectrum Image: a 3D dataset with two spatial dimensions and one spectral
dimension, obtained using spectro-microscopy techniques (\textit{i.e.}, EELS or
CL) (\ddref{SI});
	\item Transmission electron microscope: a microscope which uses an electron
beam to probe a sample and collects the signal of the transmitted beam.
\end{itemize}

\subsubsection{Physical System Description}

The physical system that the program is concerned with the analysis of includes
the following elements:

\begin{itemize}
	\item[PS1:] Sample under study.
	\item[PS2:] STEM equipped with EEL spectrometer.
	\item[PS3:] SEM or STEM equipped with CL collection system.
\end{itemize}

\subsubsection{Goal Statements}

\noindent Given a spectrum image dataset collected by either EELS or CL and a
desire to visualize and process this dataset, the goal statements for
\progname{} are:

\begin{itemize}
	\item[GS\refstepcounter{goalnum}\thegoalnum \label{G_ImportDisplay}:] Import a
3D dataset and display it such that the user can interact with it and navigate
all three dimensions.
	\item[GS\refstepcounter{goalnum}\thegoalnum \label{G_Processing}:] Provide
processing options, including normalization and deconvolution for EELS SI;
background subtraction and gain correction for a CL SI.
	\item[GS\refstepcounter{goalnum}\thegoalnum \label{G_Extraction}:] Extract
slices and areas from the dataset as desired by the user and communicated
through the user interface, export these as desired by the user.
\end{itemize}

\wss{Sounds good!}

\subsection{Solution Characteristics Specification}

The assumptions used in consideration of spectrum imaging datasets are given in
this section, followed by the models and definitions used to clearly define the
data processing steps.

\subsubsection{Assumptions}

This section simplifies the original problem and helps in developing the
theoretical model by filling in the missing information for the physical
system. The numbers given in the square brackets refer to the theoretical model
[T], general definition [GD], data definition [DD], instance model [IM], or
likely change [LC], in which the respective assumption is used.

\begin{itemize}

	\item[A\refstepcounter{assumpnum}\theassumpnum \label{EELS_System_Response}:]
The EELS data can be described as the convolution of the ``real'' spectrum of
the sample with the response of the microscope and spectrometer system function.
The microscope and spectrometer system typically causes broadening in the peaks
of the ``real'' spectrum;
	
	\item[A\refstepcounter{assumpnum}\theassumpnum \label{EELS_PSF_variability}:]
The same PSF is valid for all pixels in the EELS spectrum image;
	
	\item[A\refstepcounter{assumpnum}\theassumpnum
\label{EELS_Intensity_Fluctuations}:] Fluctuations in the intensity of the EELS
signal are due to changes in the beam current as collected by the spectrometer;
	
	\item[A\refstepcounter{assumpnum}\theassumpnum \label{CL_Background}:] A
subtraction is sufficient \wss{to?} remove the contributions of the dark signal and the
substrate signal from the CL sample signal;
	
	\item[A\refstepcounter{assumpnum}\theassumpnum \label{CL_System_Response}:] The
CL spectrometer wavelength sensitivity can be accurately modelled by
experimental reference data.

\end{itemize}

\subsubsection{Theoretical Models}\label{sec_theoretical}

This section focuses on the general equations and laws underlying the data
acquisition process on a camera, providing the motivation for the data
processing models that \progname{} will implement.

~\newline

\noindent
\begin{minipage}{\textwidth}
	\renewcommand*{\arraystretch}{1.5}
	\begin{tabular}{| p{\colAwidth} | p{\colBwidth}|}
		  \hline
		  \rowcolor[gray]{0.9}
		  Number& T\refstepcounter{theorynum}\thetheorynum \label{intensity}\\
		  \hline
		  Label&\bf Intensity \\
		  \hline
		  Equation& $\zeta(k) = g(k) S(k) + b(k) + N(k)$ \\
		  \hline
		  Description & The signal, $\zeta(k)$ (\tref{signal}), obtained from each
pixel, $k$, on the CCD detector can be considered as a sum of several different
contributions: 
		  \begin{itemize}
			\item the signal obtained from the microscope $S(k)$, which is multiplied by
the gain factor $g(k)$ of the pixel; 
			\item the background signal of each pixel $b(k)$; 
			\item and the noise present at each pixel $N(k)$.
		  \end{itemize}\\
		  \hline
		  Source & \cite{zuo_electron_2000}\\
		  \hline
		  Ref.\ By & \ddref{PSF}, \iref{background}, \iref{gain} \\
		  \hline
	\end{tabular}
\end{minipage}\\

%T2
~\newline

\noindent
\begin{minipage}{\textwidth}
	\renewcommand*{\arraystretch}{1.5}
	\begin{tabular}{| p{\colAwidth} | p{\colBwidth}|}
		  \hline
		  \rowcolor[gray]{0.9}
		  Number& T\refstepcounter{theorynum}\thetheorynum \label{signal}\\
		  \hline
		  Label&\bf Signal \\
		  \hline
		  Equation& $S(k)=\int_{pixel}d\rho \lfloor \int f(\rho') h(\rho, \rho')
d\rho' \rfloor$ \\
		  & $ S(k)=\sum_{k'} f(k') h(k, k')$ \wss{Maybe in brackets
                    after each option for $S(k)$ you should mention which
                    version it is (continuous versus discrete).  Otherwise it is
          confusing why you define a function twice.}\\
		  \hline
		  Description & The signal ($S(k)$) from the microscope, as read at one pixel
on the CCD ($k$), is an integral over the area of one pixel. Inside the integral
is the probability ($f(\rho')$) of an electron landing at position $\rho'$ on
the detector given the interaction of the beam with the sample, multiplied by
the probability ($h(\rho, \rho')$) of detecting an electron at location $\rho$
if the electron arrives at position $\rho'$ on the detector. The probability
$h(\rho, \rho')$ of detecting an electron at one location after it hits the
detector at another location is the point spread function (PSF: \ddref{PSF}).
                                \wss{When do you use the continuous versus the
                                discrete version?}\\
		  & The second equation is the discretized version of the first. The signal
$S(k)$ is the sum over all pixels $k'$ of the probability ($f(k')$) of an
electron hitting pixel $k'$ multiplied by the probability ($h(k, k')$) of
detecting an electron at $k$ after it hits pixel $k'$.\\
		  \hline
		  Source & \cite{zuo_electron_2000}\\
		  \hline
		  Ref.\ By & \tref{intensity}, \ddref{Spectrum}, \ddref{PSF},
\iref{deconvolution}\\
		  \hline
	\end{tabular}
\end{minipage}\\

~\newline

\subsubsection{General Definitions}\label{sec_gendef}

This section provides a definition beneficial to understanding the data
definitions.

%GD1
~\newline

\noindent
\begin{minipage}{\textwidth}
	\renewcommand*{\arraystretch}{1.5}
	\begin{tabular}{| p{\colAwidth} | p{\colBwidth}|}
		\hline
		\rowcolor[gray]{0.9}
		Number& GD\refstepcounter{defnum}\thedefnum \label{grid}\\
		\hline
		Label & \textbf{Scanning grid}  \\
		\hline
		Units& \si{\nano\metre}\\
		\hline
		Equation& $P(x,y) = \{\langle x, y \rangle \in A^2 \mid x \in \{0..X\}, y \in
\{0..Y\} \}$\\
		\hline
		Description & $P(x,y)$ is a set of position coordinates describing the
possible locations of the beam in two spatial coordinates $(x, y)$ during a
raster scan across the sample in a rectangular grid pattern across the area on
the sample, $A^2$. The limits on $(x,y)$ are set by the operator on the
instrument, ranging from 0 to the maximum $(X,Y)$. \wss{Do you really mean
                              $A^2$?  I would have thought you just meant $A$.}
		\\
		  \hline
		  Ref.\ By & \ddref{SI}\\
		  \hline
	\end{tabular}
\end{minipage}\\

\subsubsection{Data Definitions}\label{sec_datadef}

This section collects and defines the data types needed to build the instance
models.

% DD1
~\newline

\noindent
\begin{minipage}{\textwidth}
	\renewcommand*{\arraystretch}{1.5}
	\begin{tabular}{| p{\colAwidth} | p{\colBwidth}|}
		\hline
		\rowcolor[gray]{0.9}
		Number& DD\refstepcounter{datadefnum}\thedatadefnum \label{Spectrum}\\
		\hline
		Label& \bf Spectrum\\
		\hline
		Symbol & $I(E)$\\
		\hline
		Units & Electron counts (EELS)\\
		& Photon counts (CL)\\
		& or arbitrary units (a.u.) for EELS or CL\\
		  \hline
		  Equation &$I(E) = \zeta(k=0..K)$\\
		  \hline
		  Description & Each pixel ($k$) on the spectrometer CCD has an energy loss
		  (EELS) or wavelength (CL) value associated with it ($E$, where $E$ is also
		  function of $k$), obtained from the spectrometer acquisition settings, and a
		  collected signal ($\zeta(k)$, \tref{signal}). The intensity of the spectrum
		  ($I(E)$) is obtained from the collected signal over the whole CCD, from the
		  0th pixel to the final ($K$th) pixel on the CCD. \wss{Which count applies is
		  not clear.  Does the user select this, or does it change depending on the
		  type of imaging, or something else?  This might be something you can address
		  under your requirements?}\\
		  \hline
		  Sources & \cite{egerton_introduction_2011} \\
		  \hline
		  Ref.\ By & \tref{signal}, \ddref{SI}, \ddref{PSF}, \iref{normalization},
\iref{deconvolution}, \iref{background}, \iref{gain}\\
		  \hline
	\end{tabular}
\end{minipage}\\

% DD2
~\newline

\noindent
\begin{minipage}{\textwidth}
\renewcommand*{\arraystretch}{1.5}
\begin{tabular}{| p{\colAwidth} | p{\colBwidth}|}
	\hline
	\rowcolor[gray]{0.9}
	Number& DD\refstepcounter{datadefnum}\thedatadefnum \label{SI}\\
	\hline
	Label& \bf Spectrum Image\\
	\hline
	Symbol &$I(x, y, E)$\\
	\hline
	Units & Electron counts (EELS)\\
	& Photons (CL)\\
	& or arbitrary units (a.u.) for EELS or CL\\
	\hline
	Equation& $I(x, y, E): \forall \langle x, y \rangle \in P(x,y), \exists
I(x,y,E) \in \mathbb{R}$\\
	\hline
	Description & $I(x, y, E)$ is a dataset composed of Spectra (\ddref{Spectrum})
collected by the spectrometer detector at each value of $P(x,y)$
                      (\dref{grid}). \wss{I suggest you restate the math from
                      the above equation in natural language.}
	\\
	\hline
	Sources&~\cite{jeanguillaume_spectrum-image:_1989}  \\
	\hline
	Ref.\ By & \ddref{PSF}, \iref{normalization}, \iref{deconvolution}\\
	\hline
\end{tabular}
\end{minipage}\\

~\newline

\noindent
\begin{minipage}{\textwidth}
	\renewcommand*{\arraystretch}{1.5}
	\begin{tabular}{| p{\colAwidth} | p{\colBwidth}|}
		\hline
		\rowcolor[gray]{0.9}
		Number& DD\refstepcounter{datadefnum}\thedatadefnum \label{PSF}\\
		\hline
		Label& \bf Point Spread Function\\
		\hline
		Symbol &$I_{PSF}(E)$\\
		\hline
		Units & Electron counts or arbitrary units (a.u.) (EELS)\\
		  \hline
		  Equation&$I_{PSF}(E) = \zeta(k={0..K}) = g(k) S'(k) + b(k) +
                            N(k)$ \wss{defining the limits on $k$ inside the
                            function seems odd to me.  Maybe you could just say
                            $\zeta(k)$ and then follow the equation with a comma
          and $k \in [0..K]$?}\\
		  \hline
		  Description & $I_{PSF}(E)$ is a reference spectrum ($I(E)$,
\ddref{Spectrum}) representing the point spread function (see \tref{signal}) on
the collected spectra in the spectrum image (\ddref{SI}). It is a spectrum
collected, ideally, with no interaction with a sample; the variables in the
equation are defined as in \tref{intensity}, with the exception of $S'(k)$,
which is an example of $S(k)$ with no sample interaction with the beam.
		  \\
		  \hline
		  Sources&~\cite{zuo_electron_2000, jeanguillaume_spectrum-image:_1989,
gloter_improving_2003}  \\
		  \hline
		  Ref.\ By & \tref{signal}, \iref{deconvolution}\\
		  \hline
	\end{tabular}
\end{minipage}\\

\subsubsection{Instance Models} \label{sec_instance}    

This section details the instance models, which are the algorithms that the code
must perform to process spectrum imaging data, given the data types defined in
Data Definitions above.

The data processing goal \gsref{G_Processing} is solved by the following
instance models (\iref{normalization}, \iref{deconvolution}, \iref{background}
\iref{gain}).

~\newline

%Instance Model 1

\noindent
\begin{minipage}{\textwidth}
	\renewcommand*{\arraystretch}{1.5}
	\begin{tabular}{| p{\colAwidth} | p{\colBwidth}|}
		  \hline
		  \rowcolor[gray]{0.9}
		  Number& IM\refstepcounter{instnum}\theinstnum \label{normalization}\\
		  \hline
		  Label& \bf Normalization to the integral\\
		  \hline
		  Input& $k_1$  is the index of beginning of the spectrum range (spectrum
axis), user input\\
		  & $k_2$ is the index of end of the spectrum range (spectrum axis), user
input\\
		  & $I(E)$ is the spectrum, user input\\
		  & The input is constrained such that $k_1 >= 0$, and $k_2 <= N$\\
		  \hline
		  Equation &
                             $I_{norm}(E)=\frac{I(E)}{\sum_{E(k=k_1)}^{E(k=k_2)}}
                             I(E(k))$ \wss{Technically, LaTeX will interpret
                             norm in the equation as n times o times r ... in
                             terms of spacing.  You probably actually want
                             $I_{\text{norm}}$.  Your reviewer (Keshav) noted that the
                             denominator looks odd for this equation.  I agree.
          :-)}\\
		  \hline
		  Output & Normalized spectrum\\
		  \hline
		  Description&$I(E)$ is a spectrum (\ddref{Spectrum}), user input.\\
		  &$k_1$: Index of beginning of spectrum range (spectrum axis), user input.\\
		  &$k_2$: Index of end of spectrum range (spectrum axis), user input.\\
		  &$K$: Last index along the spectral axis.\\
		  &$x$: Location along first spatial axis.\\
		  &$y$: Location along second spatial axis.\\
		  &$I_{norm}(E)$: Spectrum normalized along the spectral axis to the integral
defined by a user.\\
		  & Note that this algorithm can be applied to a 3D spectrum image
(\ddref{SI}) by applying the algorithm to each $(x,y)$ pixel independently.\\
		  \hline
		  Sources & -- \\
		  \hline
		  Ref.\ By & \iref{deconvolution}\\
		  \hline
	\end{tabular}
\end{minipage}\\

~\newline

%Instance Model 2

\noindent
\begin{minipage}{\textwidth}
	\renewcommand*{\arraystretch}{1.5}
	\begin{tabular}{| p{\colAwidth} | p{\colBwidth}|}
		\hline
		\rowcolor[gray]{0.9}
		Number & IM\refstepcounter{instnum}\theinstnum \label{deconvolution}\\
		\hline
		Label & \bf Richardson-Lucy Deconvolution\\
		\hline
		Input& $I_{PSF}(E)$ is the point spread function of the instrument (see
\ddref{PSF}).\\
		& $I_{real}^0(E)$ is the initial guess for the ``real'' spectrum
(\ddref{Spectrum}).\\
		& $I_{measured}(E)$ is the measured spectrum, as obtained from the
spectrometer software (\ddref{Spectrum}).\\
		& $R$ is the total (non-negative integer) number of iterations of the
deconvolution algorithm applied to the spectrum. \\
		& $c$ is the number of the current iteration (at iteration 0, $c$ = 0).\\
		& $i$, $j$, $l$ are each indices of pixels in the spectrum. Three different
variables are used for the pixel index to make the effect of the different
summations clear.\\
		\hline
		Equation & $I_{real (j)}^{c+1}(E)=I_{real
(j)}^c(E)\sum_i{\frac{I_{PSF}(E)I_{measured}(E)}{\sum_l{I_{PSF}(E)I_{real
(l)}^c(E)}}}$, while $c < R$\\
		\hline
		Output& $I_{deconvolved}(E_k)=I_{real (j)}^{R}(E)$\\
		\hline
		Description & The purpose of the RL deconvolution algorithm is to separate the
effects of the spectrometer point spread function from the ``real'' signal of
the sample. \textit{I.e.} in \tref{signal}, the signal is described as the
convolution of the point spread function with the signal from the microscope.
Deconvolution is usually run after normalization of the spectra
(\iref{normalization}) to eliminate effects of changes in the beam current.\\
		& This algorithm can also  be expanded to apply to the 3D spectrum image
(\ddref{SI})) by applying it at every $(x,y)$ pixel in the spectrum image.\\
		\hline
		Sources&~\cite{gloter_improving_2003, bellido_toward_2014} \ \\
		\hline
		Ref.\ By & --\\
		\hline
	\end{tabular}
\end{minipage}\\

~\newline

%Instance Model 3

\noindent
\begin{minipage}{\textwidth}
	\renewcommand*{\arraystretch}{1.5}
	\begin{tabular}{| p{\colAwidth} | p{\colBwidth}|}
		\hline
		\rowcolor[gray]{0.9}
		Number& IM\refstepcounter{instnum}\theinstnum \label{background}\\
		\hline
		Label& \bf Background subtraction\\
		\hline
		Input & $I_{collected}$ is the spectrum as collected by the spectrometer.\\
		& $I_{background}(E)$ is a spectrum representing the background to be
subtracted (this could be associated with the dark signal or with the signal
coming from unwanted sources: \textit{e.g.} outside noise, substrate
signal...).\\
		\hline
		Equation & $I_{sample}=I_{collected}-I_{background}$\\
		\hline
		Output& $I_{sample}(E)$\\
		\hline
		Description & This model addresses the removal of background contributions to
the signal ($b(k)$ in \tref{intensity}), where $b(k)$ is transformed to a
spectrum (\ddref{Spectrum}) as a function of energy.\\
		& The background spectrum must be of the same dimensions as the spectrum it is
to be removed from. For the spectrum image, the background spectrum may be
applied independently to each pixel.\\
		\hline
		Sources & -- \\
		\hline
		Ref.\ By & --\\
		\hline
	\end{tabular}
\end{minipage}\\

~\newline

%Instance Model 4

\noindent
\begin{minipage}{\textwidth}
	\renewcommand*{\arraystretch}{1.5}
	\begin{tabular}{| p{\colAwidth} | p{\colBwidth}|}
		\hline
		\rowcolor[gray]{0.9}
		Number& IM\refstepcounter{instnum}\theinstnum \label{gain}\\
		\hline
		Label& \bf Gain correction\\
		\hline
		Input & $I_{collected}$ is the spectrum (\ddref{Spectrum}) of the signal as
collected by the spectrometer (see \tref{intensity}).\\
		& $g(E)$ is the gain ``spectrum'' of the spectrometer $g(k)$ transformed into
a function of energy (\ddref{Spectrum}).\\
		\hline
		Equation & $I_{corrected}(E)=\frac{I_{collected}(E)}{g(E)}$\\
		\hline
		Output& $I_{corrected}(E)$ is the spectrum after the gain correction from the
spectrometer CCD is applied.\\
		\hline
		Description & This model addresses the removal of the gain ($g(k)$ in
\tref{intensity}) from the collected signal, where $g(k)$ is transformed to a
``spectrum'' (\ddref{Spectrum}) as a function of energy.\\
		\hline
		Sources & --\\
		\hline
		Ref.\ By & --\\
		\hline
	\end{tabular}
\end{minipage}\\

\subsubsection{Data Constraints} \label{sec_DataConstraints}    
This section details the constraints on the data that the software must process,
including the input, processing steps, and output from the software. Physical
constraints based on the expectations of the dataset are left out to allow the
user more freedom in the input.

\begin{itemize}
	\item The input spectrum image (\ddref{SI}) must be a 3D structure, or a 3D
structure contained within a .dm3 or .h5 file;
	\item The user input number of iterations to use for the deconvolution
algorithm must be an integer number greater than or equal to 0;
	\item For the normalization model, the user input of the number of bins to
normalize must not be empty;
	\item The user input of the background signal (\iref{background}) must be of
the same length along the spectral axis as the spectrum image it will be applied
to;
	\item The user input of the gain signal (\iref{gain}) must be of the same
length along the spectral axis as the spectrum image it will be applied to;
	\item The normalization ({\iref{normalization}}) and deconvolution
({\iref{deconvolution}}) data processing algorithms both include a division by
experimentally-derived values. The denominator in these steps must not be 0.
\end{itemize}

\subsubsection{Properties of a Correct Solution} \label{sec_CorrectSolution}

\noindent
A correct solution must be acceptable to a user experienced with EELS and/or CL
data processing, as is relevant to the EELS/CL data being processed.

\section{Requirements}

This section provides the functional requirements, the business tasks that the
software is expected to complete, and the nonfunctional requirements, the
qualities that the software is expected to exhibit.

\subsection{Functional Requirements}

\noindent \begin{itemize}

	\item[R\refstepcounter{reqnum}\thereqnum \label{R_SI_inputs}:] Accept and read
the following as Spectrum Image inputs:
	
	\begin{itemize}
		\item 3D data array (\textit{e.g.}, already stored in computer memory)
		\item File in format .dm3 containing 3D data array
		\item File in format .h5 containing 3D data array
	\end{itemize}
	
	\item[R\refstepcounter{reqnum}\thereqnum \label{R_spectrum_inputs}:] Accept and
read the following as Spectrum inputs:
	
	\begin{itemize}
		\item 1D data array
		\item File containing 1D data array in .csv format
	\end{itemize}

	\item[R\refstepcounter{reqnum}\thereqnum \label{R_Input_dimension}:] Verify
that all data inputs are of the correct dimensionality and composed of Real
numbers
	
	\item[R\refstepcounter{reqnum}\thereqnum \label{R_SI_slicing}:] Accept user
input to select a spectral range and extract $(x,y)$ slices averaged over the
selected spectral range and display the $(x,y)$ slice as an image. The software
should also export this image to an image file, as desired by the user.
	
	\item[R\refstepcounter{reqnum}\thereqnum \label{R_SI_area}:] Accept user input
to define an area on an $(x,y)$ image display and extract a spectrum averaged
over selected pixels. The software should also export this spectrum as a *.csv
file, as desired by the user.
	
	\item[R\refstepcounter{reqnum}\thereqnum \label{R_deconvolution}:] Given a 1D
spectrum array and an integer number, the software should perform that number of
deconvolution iterations on a 3D EELS dataset using the 1D spectrum array as a
reference of the system response. The software should display the output of the
deconvolution for navigation by the user. Output of the deconvolution algorithm
will be validated by an expert user. 
	
	\item[R\refstepcounter{reqnum}\thereqnum \label{R_normalization}:] The software
should be able to normalize a 3D spectrum image upon user command: either to an
integrated portion of a chosen range along the spectral axis or to a single
chosen channel.
	
	\item[R\refstepcounter{reqnum}\thereqnum \label{R_background}:] The software
should be able to perform a background subtraction, given the appropriate inputs
of a spectrum or spectrum image and a background reference spectrum.
	
	\item[R\refstepcounter{reqnum}\thereqnum \label{R_gain}:] The software should
be able to perform a gain correction, given the appropriate inputs of a spectrum
or spectrum image and a gain reference spectrum.
\end{itemize}

\wss{Your requirements should reference explicitly the instance models that
  apply.  This information is in your traceability matrices, but it should also
  be explicit in the definition of the requirements themselves.}

\subsection{Nonfunctional Requirements}

The nonfunctional priorities for the software are the correctness, useability,
speed, maintainability, expandability, and understandability. The user must be
able to easily use the software to extract the data that they want and do the
processing operations that they want, given the user characteristics defined
elsewhere. The speed of performance is of interest in the interactive display of
the 3D datasets and in the data processing operations, which can be
computationally heavy with large datasets. To improve the software over time and
add new functionality as the data processing field expands, the software must be
both maintainable and understandable by the original programmer and future
programmers. In addition, a more advanced user may wish to add their own
functions into the software data processing algorithms; the software should be
written in such a way to make expanding its functionality simple and
understandable. A user may also wish to perform their own tests of the software
functions, emphasizing the need to have transparent and understandable software.
To reach a wide user base, the program should be usable on many different
operating systems. The nonfunctional requirements are listed below:

\begin{itemize}
	\item The software should be maintainable by the original programmer or future
programmers to keep up to date with advances in data processing practices;
	\item The software should allow the addition of more data processing functions
as required by an advanced user experienced in programming;
	\item The software should not display significant lag when interacting with the
user, except in the case of data processing steps, which may be computationally
expensive;
	\item Use of the interactive display and of data processing functions should be
intuitive for a user experienced with the processing techniques for 3D datasets;
	\item The coding practices for the software should be open and transparent to
facilitate inspection by an experienced user to encourage trust in the data
processing steps performed;
	\item The display for the user to interact with should be readable:
\textit{i.e.} the text should be large enough to read by a user with average
eyesight, with high contrast against the background for visibility;
	\item The colours used in the display should be clearly distinguishable from
each other;
	\item The software should function on multiple operating systems, including
Windows, Linux, and MacOS, in a laptop or desktop environment.
\end{itemize}

\section{Likely Changes}
	This section details some changes which are likely to be made to the software
once the initial requirements are met, in the interests of continuous
improvement of the software's capabilities.
	\noindent \begin{itemize}
		\item[LC\refstepcounter{lcnum}\thelcnum\label{LC_identicalPSF}:] A PSF unique
to each pixel in the spectrum image (\ddref{SI}) will be used instead of
assuming one PSF is valid for all pixels within a spectrum image
(\aref{EELS_PSF_variability})
		\item[LC\refstepcounter{lcnum}\thelcnum\label{LC_identicalPSF}:] Cosmic rays
appear as high intensity single pixel spikes in a spectrum and can be removed
via 3D image filtering techniques. A future change will implement this filtering
option for the user. 
	\end{itemize}

\section{Traceability Matrices and Graphs}

The purpose of the traceability matrices is to provide easy references on what
has to be additionally modified if a certain component is changed.  Every time a
component is changed, the items in the column of that component that are marked
with an ``X'' may have to be modified as well.  Table~\ref{Table:trace} shows
the
dependencies of theoretical models, general definitions, data definitions, and
instance models with each other. Table~\ref{Table:R_trace} shows the
dependencies of instance models, requirements, and data constraints on each
other. Table~\ref{Table:A_trace} shows the dependencies of theoretical models,
general definitions, data definitions, instance models, and likely changes on
the assumptions.

\begin{table}[H]
	\centering
	\begin{tabular}{|c|c|c|c|c|c|c|c|c|c|c|c|c|c|c|c|c|c|c|c|}
		\hline
			& \aref{EELS_System_Response}& \aref{EELS_PSF_variability}&
\aref{EELS_Intensity_Fluctuations}& \aref{CL_Background}&
\aref{CL_System_Response}\\
		\hline
		\tref{intensity}     &  &  &  & X& X\\ \hline
		\tref{signal}        & X&  & X&  &  \\ \hline
		\dref{grid}          &  & X&  &  &  \\ \hline
		\ddref{Spectrum}     & X&  &  &  &  \\ \hline
		\ddref{SI}           & X& X&  &  &  \\ \hline
		\ddref{PSF}          & X& X&  &  &  \\ \hline
		\iref{normalization} &  &  & X&  &  \\ \hline
		\iref{deconvolution} & X& X&  &  &  \\ \hline
		\iref{background}    &  &  &  & X&  \\ \hline
		\iref{gain}          &  &  &  &  & X\\ \hline
	\end{tabular}
	\caption{Traceability Matrix Showing the Connections Between Assumptions and
Other Items}
	\label{Table:A_trace}
\end{table}

\begin{table}[H]
	\centering
	\begin{tabular}{|c|c|c|c|c|c|c|c|c|c|c|c|c|c|c|c|c|c|c|c|c|c|c|c|}
		\hline 
		& \tref{intensity} & \tref{signal} & \dref{grid} & \ddref{Spectrum}&
\ddref{SI} & \ddref{PSF}& \iref{normalization}& \iref{deconvolution}&
\iref{background}& \iref{gain} \\
		\hline
		\tref{intensity}     &  & X&  &  &  &  &  &  & X& X\\ \hline
		\tref{signal}        &  &  &  &  &  &  &  &  &  &  \\ \hline
		\dref{grid}          &  &  &  &  &  &  &  &  &  &  \\ \hline
		\ddref{Spectrum}     & X&  &  &  &  &  &  &  & X&  \\ \hline
		\ddref{SI}           &  &  & X& X&  &  &  &  &  &  \\ \hline
		\ddref{PSF}          & X&  &  & X&  &  &  & X&  &  \\ \hline
		\iref{normalization} &  & X&  &  &  &  &  &  &  &  \\ \hline
		\iref{deconvolution} & X& X&  &  &  & X&  &  &  &  \\ \hline
		\iref{background}    & X&  &  &  &  &  &  &  &  &  \\ \hline
		\iref{gain}          & X&  &  &  &  &  &  &  &  &  \\
		\hline
	\end{tabular}
	\caption{Traceability Matrix Showing the Connections Between Items of Different
Sections}
	\label{Table:trace}
\end{table}

\begin{table}[H]
\centering
\begin{tabular}{|c|c|c|c|c|c|c|c|c|c|c|c|c|c|c|}
\hline
	& \iref{normalization}& \iref{deconvolution}& \iref{background}& \iref{gain}&
\ref{sec_DataConstraints}& \rref{R_SI_inputs}& \rref{R_spectrum_inputs} &
\rref{R_Input_dimension} & \rref{R_SI_slicing} & \rref{R_SI_area} &
\rref{R_deconvolution} & \rref{R_normalization} & \rref{R_background} &
\rref{R_gain}\\
\hline
\iref{normalization}     &  &  &  &  &  & X&  &  &  &  &  & X&  &  \\ \hline
\iref{deconvolution}     & X&  &  &  &  & X&  & X&  &  & X&  &  &  \\ \hline
\iref{background}        &  &  &  &  &  & X& X& X&  &  &  &  & X&  \\ \hline
\iref{gain}              &  &  &  &  &  & X& X& X&  &  &  &  &  & X\\ \hline
\rref{R_SI_inputs}       &  &  &  &  & X& X&  & X&  & X&  &  &  &  \\ \hline
\rref{R_spectrum_inputs} &  &  &  &  & X&  &  & X&  &  & X&  &  &  \\ \hline
\rref{R_Input_dimension} &  &  &  &  & X& X& X&  & X& X&  &  & X& X\\ \hline
\rref{R_SI_slicing}      &  &  &  &  &  &  &  &  &  &  &  &  &  &  \\ \hline
\rref{R_SI_area}         &  &  &  &  &  &  &  &  &  &  &  &  &  &  \\ \hline 
\rref{R_deconvolution}   & X& X&  &  &  &  & X&  &  &  &  & X&  &  \\ \hline
\rref{R_normalization}   & X&  &  &  &  &  &  &  &  &  &  &  &  &  \\ \hline
\rref{R_background}      &  &  & X&  &  &  & X&  &  &  &  &  &  &  \\ \hline
\rref{R_gain}            &  &  &  & X&  &  & X&  &  &  &  &  &  &  \\ \hline

\hline
\end{tabular}
\caption{Traceability Matrix Showing the Connections Between Requirements and
Instance Models}
\label{Table:R_trace}
\end{table}

\newpage

\bibliography {SRS}

\end{document}
