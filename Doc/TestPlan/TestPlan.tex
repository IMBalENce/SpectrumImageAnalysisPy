\documentclass[12pt, titlepage]{article}

\usepackage{booktabs}
\usepackage{siunitx}
\usepackage{tabularx}
\usepackage{xr-hyper}

\usepackage{hyperref}
\hypersetup{
    colorlinks,
    citecolor=black,
    filecolor=black,
    linkcolor=red,
    urlcolor=blue
}

\externaldocument{../SRS/SRS}

\newcommand{\progname}{SpectrumImageAnalysisPy}
\newcommand{\rthereqnum}{P\thereqnum}
\newcommand{\ddref}[1]{DD\ref{#1}}

%% Comments

\usepackage{color}

\newif\ifcomments\commentstrue

\ifcomments
\newcommand{\authornote}[3]{\textcolor{#1}{[#3 ---#2]}}
\newcommand{\todo}[1]{\textcolor{red}{[TODO: #1]}}
\else
\newcommand{\authornote}[3]{}
\newcommand{\todo}[1]{}
\fi

\newcommand{\wss}[1]{\authornote{blue}{SS}{#1}}
\newcommand{\an}[1]{\authornote{magenta}{Author}{#1}}


\begin{document}
\bibliographystyle{ieeetr}
\title{SpectrumImageAnalysisPy} 
\author{Isobel Bicket}
\date{\today}
	
\maketitle

\pagenumbering{roman}

\section{Revision History}

\begin{tabularx}{\textwidth}{p{4cm}p{2cm}X}
\toprule {\bf Date} & {\bf Version} & {\bf Notes}\\
\midrule
October 31, 2017 & 1.0 & Initial draft\\
\bottomrule
\end{tabularx}

~\newpage

\section{Symbols, Abbreviations and Acronyms}

\renewcommand{\arraystretch}{1.2}
\begin{tabular}{l l} 
  \toprule		
  \textbf{symbol} & \textbf{description}\\
  \midrule 
  GUI & Graphical User Interface\\
  RMS & Root Mean Square\\
  T & Test\\
  \bottomrule
\end{tabular}\\

See the SRS \hyperref[SRS:RefMat]{Reference Material} for further definitions. 

\newpage

\tableofcontents

\listoftables

\newpage

\pagenumbering{arabic}

This document outlines the different tests and assessment tools which will be
used for assessing the performance of \progname{}. The testing will be done to
build confidence in the performance of the program and allow the end user to use
it with more assurity that it is behaving correctly and any artifacts seen in
the data are either a known result of the program, or independent of the use of
\progname{} to process the data.

\section{General Information}
The testing outlined in this document gives valuable contributions to the level
of trust in the performance of the software. Not only will testing the software
build confidence that it is performing correctly, but the results of the tests
will also tell the user what to expect when running the software, for instance
in terms of performance milestones. Testing not only ensures that the software
will, within predictive power, fulfill its \hyperref[subsec:FR]{functional
requirements}, but gives the progress and effectiveness of the software in
fulfulling the \hyperref[subsec:NfR]{non-functional
  requirements}. \wss{Neat idea to have hyperrefs.  Unfortunately they
  didn't work when I clicked them.}

\subsection{Purpose}
The purpose of this document is to describe the tests which will be carried out
on \progname{} to verify that it meets the requirements described in the
\hyperref[doc:SRS]{SRS}.

\wss{An explicit web-link to your GitHub repo would be nice.}

\subsection{Scope}
This document will cover the verification plan for testing the software
\progname{}, including the system tests towards verifying that the software
meets the functional and non-functional requirements; and the unit tests, of
which a non-exhaustive list is given. The document does not cover the validation
of the software algorithms, but will include an estimate of the combined error
of the software and deconvolution algorithm used. 

\subsection{Overview of Document}
 It begins by outlining the structure of the document, followed by an overview
of the testing arrangements. Specific details follow on the tests which will be
performed to test whether or not \progname{} satisfies the
\hyperref[subsec:FuncReqTest]{functional} and
\hyperref[subsec:NonfuncReqTest]{non-functional} requirements, as well as
details of the unit tests which will be written. The unit tests themselves may
satisfy some of the functional requirements without needing system testing,
these will be detailed under \hyperref[sec:UnitTest]{Unit Testing Plan} as
appropriate.

\section{Plan}
This section details the plan to be followed when testing the software,
including those involved in the testing, the testing approach, and the
verification tools which will be used.
	
\subsection{Software Description}
The software, \progname{}, is designed as a data analysis tool for electron
energy loss spectroscopy and cathodoluminescence spectrum imaging data. It
provides several data processing routines to the user and a graphical user
interface to navigate the 3D data set and export desired spectra or image
slices. The requirements and expectations for \progname{} are detailed in the
\hyperref[doc:SRS]{Software Requirements Specification}.

\subsection{Test Team}
The test team has one member: Isobel Bicket.

\subsection{Automated Testing Approach}

Unit testing will be performed on the functions within the code, as an automated
task to be done with every update of the code. The running of automated tests
also provides regression testing and integration testing of new features and
updates. The goal of testing is 100\% code coverage.\\
The testing plan will not cover testing of the GUI, only the functions called by
the GUI after user interaction. Proper user interfacing is expected to be done
by the library used to build the GUI.\\

\subsection{Verification Tools}
\label{subsec:VerificationTools}
The following verification tools will be used:
\begin{itemize}
	\item Python unittest library: for performing unit tests on the code;
	\item Python coverage library: to determine the coverage of the tests;
	\item Deconvo.m: a Richardson-Lucy deconvolution algorithm implementation in
Matlab, written by Dr. E.P. Bellido \cite{bellido_toward_2014};
	\item Python HyperSpy library: a Python-based library for spectrum processing
for EELS and other techniques \cite{francisco_de_la_pena_2017_583693}.
\end{itemize}

\subsection{Non-Testing Based Verification}

The following non-testing based verification methods will be used to assess the
performance of \progname{}

\begin{itemize}
	\item Code review: a detailed code review will be performed by an external
party and feedback provided;
	\item User survey: a group of qualified users will be asked to process a sample
dataset using the software and will be polled on their user experience. The
group should include a number of people with different levels of experience in
data processing and programming, as well as different computer systems.
\end{itemize}

\section{System Test Description}
\label{sec:SysTest}
This section lists the system tests to be performed to verify whether or not the
program fulfills the functional requirements and to test how well it meets the
non-functional requirements. Note that some of the tests for the functional
requirements are unit tests and can be found in the \hyperref[sec:UnitTest]{Unit
Test} section.

\subsection{Tests for Functional Requirements}
\label{subsec:FuncReqTest}
Here the tests to verify whether or not \progname{} meets the functional
requirements are listed. In addition, some of the tests validate the algorithms
chosen for use in \progname{}.

\subsubsection{Deconvolution}
In this subsection, the tests for the performance of the Richardson-Lucy
deconvolution algorithm will be explained.
\paragraph{}

\begin{enumerate}

\item{Deconvolution error estimation - blank PSF}
\label{TFR:RLblankPSF}

Type: Functional, Dynamic, Automatic

Initial State: Spectrum loaded into \progname{}.

Input: A known spectrum (for instance, the data used in the paper by Bellido
\textit{et al} \cite{bellido_toward_2014}, if it is available from the
author). \wss{Don't wait to ask for the information from Bellido.  You
  should contact him or her now.  This sounds like a nice test case.}
Otherwise a spectrum will be simulated. A range of integers will be used as
input for the number of deconvolution iterations. The input of a point spread
function to the deconvolution will be an array of ones.

Output: A series of spectra deconvolved using the range of iterations given in
the input, and a plot of the RMS error vs the number of iterations used.
					
How test will be performed: A series of deconvolutions will be run on the
artificial spectrum, with varying numbers of iterations to produce a set of
deconvolved spectra. The difference between the deconvolved spectra and the
original simulated spectrum will be calculated and the RMS error plotted as a
function of number of iterations. Ideally the deconvolved spectra will be
identical to the input spectrum, because we are using a null function as the PSF
reference.\\


\item{Deconvolution error estimation}
\label{TFR:RLerror}

Type: Functional, Dynamic, Automatic
					
Initial State: Spectrum loaded into \progname{}.

Input: A known spectrum convolved with a broadening function (for instance, the
data used in the paper by Bellido \textit{et al} \cite{bellido_toward_2014}, if
it is available from the author). Otherwise a spectrum will be simulated and
convolved with a broadening function (such as a Lorentzian peak, a Gaussian
peak, or an experimental point spread function). A range of integers will be
used as input for the number of deconvolution iterations. The broadening
function itself is also an input to the deconvolution.

Output: A series of spectra deconvolved using the range of iterations given in
the input, and a plot of the RMS error vs the number of iterations used.
					
How test will be performed: A series of deconvolutions will be run on the
artificially broadened spectrum, using the broadening function as the point
spread function, with varying numbers of iterations to produce a set of
deconvolved spectra. The difference between the deconvolved spectra and the
original simulated spectrum (before convolution with a broadening function) will
be calculated and the root mean square error plotted as a function of number of
iterations. The same data will be deconvolved using the HyperSpy toolbox and the
Matlab code listed in the \hyperref[subsec:VerificationTools]{Verification
Tools} section.\\


\item{Deconvolution signal-to-noise ratio estimation}
\label{TFR:RLSNR}

Type: Functional, Dynamic, Automatic
					
Initial State: Spectrum loaded into \progname{}.

Input: A known spectrum convolved with a broadening function (for instance, the
data used in the paper by Bellido \textit{et al} \cite{bellido_toward_2014}, if
it is available from the author). Otherwise a spectrum will be simulated and
convolved with a broadening function (such as a Lorentzian peak, a Gaussian
peak, or an experimental point spread function). A range of integers will be
used as input for the number of deconvolution iterations. The broadening
function itself is also an input to the deconvolution.

Output: A series of spectra deconvolved using the range of iterations given in
the input, and a plot of the SNR vs the number of iterations used.
					
How test will be performed: A series of deconvolutions will be run on the
artificially broadened spectrum, using the broadening function as the point
spread function, with varying numbers of iterations to produce a set of
deconvolved spectra. The difference between the deconvolved spectra and the
original simulated spectrum (before convolution with a broadening function) will
be calculated and the signal-to-noise ratio plotted as a function of number of
iterations. The test will follow the SNR estimation used in
\cite{bellido_toward_2014}, taking the variance of the signal every five data
points.\\

\end{enumerate}

\subsubsection{File Export}
In this subsection, the tests for checking that the program writes files
correctly will be explained.

\begin{enumerate}

\item{Image export}
\label{TFR:ImgExport}

Type: Functional, Dynamic, Manual

Initial State: Spectrum image loaded into \progname{}. The spectrum image to be
used for this test may, for example, consist of float values from 0 to 1,
reshaped into the size $(2,2,10)$, with an energy range from $0-9\
\si{\electronvolt}$.

Input: A selected range in the spectrum (eg, $1$ to $2\ \si{\electronvolt}$).

Output: An \textit{Image.png} image written to the filesystem corresponding to
the signal averaged over the selected energy range for each pixel.

How test will be performed: Using a 3D dataset, the test will request the return
value of an $(x,y)$ image averaged over the chosen spectral range and export
this to an \textit{Image.png} file. No errors should be raised. The user must
check that the range averaged over is the same as the range selected within the
data limits.\\
\an{Not sure if I can change these tests into automatic tests. I could read the
file back into the program, but then I'm testing both the export and import
functions and can't separate errors from one, so I made the test
manual.}
\wss{You could do this test the first time as a manual test, and then
  change it into an automated test.  It could become part of your
  regression testing.  The idea is that you would determine with
  confidence that Image.png is very likely correct.  You can then test
  any future executions to make sure that the generated Image.png
  matches the ``golden'' version.  This kind of test doesn't prove that
  the future versions of your software are correct, but it does show
  whether something you have done has altered the output.  This is
  what is done with VTK.  They have nightly builds and verify that any
changes have not altered their outputs.  (If the output is altered,
I'm sure they consider the possibility that the ``golden'' version had
an error.  After a quick google search, I found some code that vtk
uses for their unit tests.  This might be too much work for you to
take on in the scope of this course, but it is worth thinking about.}

\item{Image export - GUI}
\label{TFR:ImgExportGUI}

Type: Functional, Dynamic, Manual

Initial State: Spectrum image loaded into \progname{}. The spectrum image to be
used for this test may, for example, consist of float values from 0 to 1,
reshaped into the size $(10,10,25)$, with an energy range from $0-9\
\si{\electronvolt}$.

Input: The user must interact with the GUI to select an energy range in the
spectrum from which to create and export an image.

Output: An \textit{Image.png} image written to the filesystem corresponding to
the image averaged over the selected spectrum energy range.

How test will be performed: Using a 3D dataset, the test will display the GUI
and ask the user to select an energy range in the spectrum and run the command
to export an image to file. The user must inspect the \textit{Image.png} file to
ensure that the correct image was exported from the correct pixel.\\

\item{Spectrum export}
\label{TFR:SpecExport}

Type: Functional, Dynamic, Manual

Initial State: Spectrum image loaded into \progname{}. The spectrum image to be
used for this test may, for example, consist of float values from 0 to 1,
reshaped into the size $(2,2,10)$, with an energy range from $0-9\
\si{\electronvolt}$.

Input: A selected range in the image (\textit{e.g.}, pixel $(0,0)$).

Output: A \textit{Spectrum.csv} file written to the filesystem corresponding to
the spectrum averaged over the selected area.

How test will be performed: Using a 3D dataset, the test will request the return
value of a spectrum averaged over the chosen area. No errors should be raised.
The user must inspect the \textit{Spectrum.csv} file to ensure that the correct
spectrum was exported from the correct pixel, with the correct energy range.

\item{Spectrum export - GUI}
\label{TFR:SpecExportGUI}

Type: Functional, Dynamic, Manual

Initial State: Spectrum image loaded into \progname{}. The spectrum image to be
used for this test may, for example, consist of float values from 0 to 1,
reshaped into the size $(10,10,25)$, with an energy range from $0-9\
\si{\electronvolt}$.

Input: The user must interact with the GUI to select an area on the image from
which to create and export the spectrum.

Output: A \textit{Spectrum.csv} image written to the filesystem corresponding to
the spectrum averaged over the selected area.

How test will be performed: Using a 3D dataset, the test will display the GUI
and ask the user to select an area in the image and run the command to export a
spectrum to file. The user must inspect the \textit{Spectrum.csv} file to ensure
that the correct spectrum was exported from the correct pixel.\\

\end{enumerate}


\subsection{Tests for Nonfunctional Requirements}
\label{subsec:NonfuncReqTest}

\subsubsection{Program speed}

\begin{enumerate}

\paragraph{Deconvolution}
\item{Deconvolution calculation time - SI size}

Type: Performance, Dynamic, Automatic
					
Initial State: Spectrum loaded into \progname{}.

Input: A known spectrum convolved with a broadening function (for instance, the
data used in the paper by Bellido \textit{et al} \cite{bellido_toward_2014}, if
it is available from the author), and transformed into a spectrum image data
format (of 3D shape). Otherwise a spectrum will be simulated and convolved with
a broadening function (such as a Lorentzian peak, a Gaussian peak, or an
experimental point spread function). The broadening function itself is also an
input to the deconvolution. The spectrum image size will be varied from
$(1\text{ by }1\text{ by }K)$ to $(1000 \text{ by } 1000 \text{ by } K)$ with a
minimum of 10 intermediate values. 25 iterations will be performed for each
spectrum image size. A typical value for $K$ might be 2048 or 1024, depending on
the dataset used.

Output: A plot of the calculation time required vs the size of the spectrum
image used.

How test will be performed: The Richardson-Lucy deconvolution algorithm will be
run on a series of spectrum images of varying size. 25 iterations will be run on
each spectrum image and the time required to finish the calculations recorded
and plotted against the number of $(x, y)$ pixels in the spectrum image. The
test will be repeated for the deconvolution algorithm written in the HyperSpy
toolbox, for comparison of the plots.\\


\item{Deconvolution calculation time - iterations}

Type: Performance, Dynamic, Automatic

Initial State: Spectrum loaded into \progname{}.

Input: A known spectrum convolved with a broadening function (for instance, the
data used in the paper by Bellido \textit{et al} \cite{bellido_toward_2014}, if
it is available from the author), and transformed into a spectrum image data
format (of 3D shape). Otherwise a spectrum will be simulated and convolved with
a broadening function (such as a Lorentzian peak, a Gaussian peak, or an
experimental point spread function). The broadening function itself is also an
input to the deconvolution. The spectrum image size will be set at $(500 \text{
by } 500 \text{ by } E)$ and a the number of iterations will be changed from 1
to 500 with at least 10 intermediate steps. A typical value for E might be 2048
or 1024, depending on the dataset used.

Output: A plot of the calculation time required vs the number of iterations
used.

How test will be performed: The Richardson-Lucy deconvolution algorithm will be
run repeatedly on a spectrum image with varying numbers of iterations. The time
required to finish the calculations will be recorded and plotted against the
number of $(x, y)$ pixels in the spectrum image.  The test will be repeated for
the deconvolution algorithm written in the HyperSpy toolbox, for comparison of
the plots.\\

\paragraph{Normalization}
\item{Normalization calculation time}

Type: Performance, Dynamic, Automatic

Initial State: Spectrum loaded into \progname{}.

Input: A known spectrum (for instance, an artificially created spectrum, a range
of values from 0 to the number of pixels used in the spectrum image),
transformed into a spectrum image data format (of 3D shape).The spectrum image
size will be varied from $(1\text{ by }1\text{ by }K)$ to $(1000 \text{ by }
1000 \text{ by }K)$ with a minimum of 10 intermediate values. A typical value
for $K$ might be 2048 or 1024, depending on the dataset used.

Output: A plot of the calculation time required vs the size of the spectrum
image used.

How test will be performed: The normalization algorithm will be run on a series
of spectrum images of varying size, normalizing to the integral of the full
spectrum. The time required to finish the calculations will be recorded and
plotted against the number of $(x, y)$ pixels in the spectrum image.\\

\paragraph{Gain Correction}
\item{Gain correction calculation time}

Type: Performance, Dynamic, Automatic

Initial State: Spectrum loaded into \progname{}.

Input: A known spectrum (for instance, an artificially created spectrum, a range
of values from 0 to the number of pixels used in the spectrum image),
transformed into a spectrum image data format (of 3D shape).The spectrum image
size will be varied from $(1\text{ by }1\text{ by }K)$ to $(1000 \text{ by }
1000 \text{ by }K)$ with a minimum of 10 intermediate values. A typical value
for $K$ might be 2048 or 1024, depending on the dataset used. The gain
correction factor will be given as a range of integers from 0 to $K-1$.

Output: A plot of the calculation time required vs the size of the spectrum
image used.

How test will be performed: The gain correction function will be run on a series
of spectrum images of varying size. The time required to finish the calculations
will be recorded and plotted against the number of $(x, y)$ pixels in the
spectrum image.\\

\paragraph{Background subtraction}
\item{Background subtraction calculation time}

Type: Performance, Dynamic, Automatic

Initial State: Spectrum loaded into \progname{}.

Input: A known spectrum (for instance, the data used in the paper by
Bellido \textit{et al} \cite{bellido_toward_2014}, if it is available
from the author), transformed into a spectrum image data format (of 3D
shape).The spectrum image size will be varied from
$(1\text{ by }1\text{ by }K)$ to
$(1000 \text{ by } 1000 \text{ by }K)$ with a minimum of 10
intermediate values. A typical value for $K$ might be 2048 or 1024,
depending on the dataset used. The background correction will be given
as a an array of ones of size $K$.

Output: A plot of the calculation time required vs the size of the spectrum
image used.

How test will be performed: The gain correction function will be run on a series
of spectrum images of varying size. The time required to finish the calculations
will be recorded and plotted against the number of $(x, y)$ pixels in the
spectrum image.\\

\end{enumerate}

\wss{I did not see the Usability NFR explicitly addressed.  You have
  information on this throughout your document, but you really should
  have a specific test plan related to it.  You give a great user
  survey and some information on who will be recruited, but  it would
  be nice to see the specific task that they will be asked to
  perform.  It would also be nice to know what resources they will be given, in terms of
  documentation, assistance from you etc.}

\subsection{Traceability Between Test Cases and Requirements}
This section provides a matrix with the relationships between the
\hyperref[subsec:FR]{functional requirements} in the SRS and the
\hyperref[subsec:FuncReqTest]{System Tests} outlined above. The non-functional
requirements are addressed in the section \hyperref[subsec:NonfuncReqTest]{Tests
for Nonfunctional Requirements}. Note that testing for some of the
non-functional requirements will be addressed with the user survey and code
review.
\begin{table}[h]
	\centering
	\begin{tabular}{|c|c|c|c|c|c|c|c|c|c|c|c|c|c|c|c|c|c|c|c|}
		\hline 
		& \hyperref[TFR:RLblankPSF]{1} & \hyperref[TFR:RLerror]{2} &
\hyperref[TFR:RLSNR]{3} &  \hyperref[TFR:ImgExport]{4} &
\hyperref[TFR:ImgExportGUI]{5} & \hyperref[TFR:SpecExport]{6}&
\hyperref[TFR:SpecExportGUI]{7} & \hyperref[sec:UnitTest]{Unit Test} \\
		\hline
		\hyperref[R_SI_inputs]{SI input}     			&  &  &  &  &  &  &  & X\\ 
		\hline
		\hyperref[R_spectrum_inputs]{Spectrum input}    &  &  &  &  &  &  &  & X\\ 
		\hline
		\hyperref[R_Input_dimension]{Input dimensions}  &  &  &  &  &  &  &  & X\\
		\hline
		\hyperref[R_SI_slicing]{SI slicing}				&  &  &  & X& X&  &  & X\\ 
		\hline
		\hyperref[R_SI_area]{SI area}          			&  &  &  &  &  & X& X& X\\ 
		\hline
		\hyperref[R_deconvolution]{Deconvolution}       & X& X& X&  &  &  &  & X\\ 
		\hline
		\hyperref[R_normalization]{Normalization} 		&  &  &  &  &  &  &  & X\\ 
		\hline
		\hyperref[R_background]{Background} 			&  &  &  &  &  &  &  & X\\ 
		\hline
		\hyperref[R_gain]{Gain}  						&  &  &  &  &  &  &  & X\\ 
		\hline
	\end{tabular}
	\caption{Traceability Matrix Showing the Connections Between Test Cases and
Requirements}
	\label{Table:A_trace}
\end{table}

				
\section{Unit Testing Plan}
\label{sec:UnitTest}

\subsection{Inputs}
This section details the unit tests concerning the import of spectrum
images and spectrum datasets (\hyperref[R_SI_inputs]{R1},
\hyperref[R_spectrum_inputs]{R2}, \hyperref[R_Input_dimension]{R3})
\wss{Very nice to have explicit cross-references between documents.
  To keep the information up to date, consider using make to build
  your documentation.}

\paragraph{Spectrum Image Input}

\begin{enumerate}

\item{SI Data array with energy range (EELS)}

Type: Functional, Dynamic, Unit

Initial State: None

Input: 3D data array, present in memory; use standard calibration
variabilities\\
For example, the following values might be used to create an SI: 
\begin{itemize}
	\item $x = [0, 1]$
	\item $y = [0, 1]$
	\item $\text{Intensity} = \text{ones}(\text{size}(2,2,10))$
	\item $\text{x calibration} = 1\ \si{\nano\metre}/\text{pixel}$
	\item $\text{energy range} = \text{range}(0, 10)\ \si{\electronvolt}$
\end{itemize}

Output: Spectrum Image stored within \progname{}; the first and second axes
should correspond to spatial dimensions, while the third axis should correspond
to a spectral dimension.

How test will be performed: Create a 3D data array and attempt to initialize a
spectrum image in \progname{}. No errors should be raised. The spectrum image
axes should be read in the correct order (x, y, E), and the calibrations should
be applied to the correct axes. Addresses \hyperref[R_SI_inputs]{Functional
Requirement}.


\item{SI Data array with dispersion (EELS)}

Type: Functional, Dynamic, Unit

Initial State: None

Input: 3D data array, present in memory; use standard calibration
variabilities\\
For example, the following values might be used to create an SI: 
\begin{itemize}
	\item $x = [0, 1]$
	\item $y = [0, 1]$
	\item $\text{Intensity} = \text{ones}(\text{size}(2,2,10))$
	\item $\text{x calibration} = 1\ \si{\nano\metre}/\text{pixel}$
	\item $\text{dispersion} = 0.01\ \si{\electronvolt}/\text{pixel}$
	\item $E(0) = -2\ \si{\electronvolt}$
\end{itemize}

Output: Spectrum Image stored within \progname{}; the first and second axes
should correspond to spatial dimensions, while the third axis should correspond
to a spectral dimension.

How test will be performed: Create a 3D data array and attempt to initialize a
spectrum image in \progname{}. No errors should be raised. The spectrum image
axes should be read in the correct order (x, y, E), and the calibrations should
be applied to the correct axes.


\item{SI Data array with wavelength range (CL)}

Type: Functional, Dynamic, Unit

Initial State: None

Input: 3D data array, present in memory; use standard calibration
variabilities\\
For example, the following values might be used to create an SI: 
\begin{itemize}
	\item $x = [0, 1]$
	\item $y = [0, 1]$
	\item $\text{Intensity} = \text{ones}(\text{size}(2,2,10))$
	\item $\text{x calibration} = 1\ \si{\nano\metre}/\text{pixel}$
	\item $\text{wavelength range} = \text{range}(0, 10)\ \si{\nano\metre}$
\end{itemize}

Output: Spectrum Image stored within \progname{}; the first and second axes
should correspond to spatial dimensions, while the third axis should correspond
to a spectral dimension.

How test will be performed: Create a 3D data array and attempt to initialize a
spectrum image in \progname{}. No errors should be raised. The spectrum image
axes should be read in the correct order (x, y, E), and the calibrations should
be applied to the correct axes.


\item{SI .dm3 file}

Type: Functional, Dynamic, Unit

Initial State: None

Input: A .dm3 file, containing a fabricated spectrum image. The .dm3 file will
be created using Digital Micrograph (Gatan Microscopy Suite Software) software
\cite{gatan_inc._gatan_nodate} with the following parameters:
\begin{itemize}
	\item $x = [0, 1]$
	\item $y = [0, 1]$
	\item $\text{Intensity} = \text{ones}(\text{size}(2,2,10))$
	\item $\text{x calibration} = 1\ \si{\nano\metre}/\text{pixel}$
	\item $\text{dispersion} = 0.01\ \si{\electronvolt}/\text{pixel}$
	\item $E(0) = -2\ \si{\electronvolt}$
\end{itemize}

Output: EELS Spectrum Image stored within \progname{}, with metadata assigned to
appropriate values

How test will be performed: Read in the .dm3 file with \progname{}. Check that
no errors are raised. Display the spectrum image to manually check that the data
was read in correctly. Check spatial and spectral calibration to verify metadata
was assigned correctly.

\an{There is a potential problem with the .dm3 file, since there are many
version of Digital Micrograph out there and many different set-ups, all of which
may format their .dm3 slightly differently. Not sure on the best way to test
this, without asking several other labs for pieces of data (which are not
usually shared readily) and trying out all of them.}

\wss{Can you simplify your life and, at least for the time being,
  select one dm3 standard and make that the one you support and test?}

\item{SI .h5 file}

Type: Functional, Dynamic, Unit.

Initial State: None

Input: .h5 file, containing an acquired or fabricated spectrum image. The .h5
spectrum image file originates from Odemis software \cite{bv_odemis:_nodate}, it
may be possible to fabricate a spectrum image using the scripting interface,
using the following values for the calibration parameters, otherwise it may be
necessary to use an acquired data file.

\begin{itemize}
	\item $x = [0, 1]$
	\item $y = [0, 1]$
	\item $\text{Intensity} = \text{ones}(\text{size}(2,2,10))$
	\item $\text{x calibration} = 1\ \si{\nano\metre}/\text{pixel}$
	\item $\text{wavelength range} = \text{range}(0, 10)\ \si{\nano\metre}$
\end{itemize}

Output: CL Spectrum Image stored within \progname, with metadata assigned to
appropriate values.
					
How test will be performed: Read in the .h5 file with \progname{}. Check that no
errors are raised. Check spatial and spectral calibration to verify metadata was
assigned correctly.

\item{SI Data array with dispersion (EELS) - complex numbers}

Type: Functional, Dynamic, Unit

Initial State: None

Input: 3D data array, present in memory; use standard calibration
variabilities\\
For example, the following values might be used to create an SI: 
\begin{itemize}
	\item $x = [0, 1]$
	\item $y = [0, 1]$
	\item $\text{Intensity} = \text{ones}(\text{size}(2,2,10)) + j *
\text{ones}(\text{size}(2,2,10))$
	\item $\text{x calibration} = 1\ \si{\nano\metre}/\text{pixel}$
	\item $\text{dispersion} = 0.01\ \si{\electronvolt}/\text{pixel}$
	\item $E(0) = -2\ \si{\electronvolt}$
\end{itemize}

Output: Spectrum Image stored within \progname{}; the first and second axes
should correspond to spatial dimensions, while the third axis should correspond
to a spectral dimension.

How test will be performed: Create a 3D data array and attempt to initialize a
spectrum image in \progname{}. No errors should be raised. The spectrum image
axes should be read in the correct order (x, y, E), and the calibrations should
be applied to the correct axes.

\end{enumerate}

\paragraph{Spectrum Input}

\begin{enumerate}

\item{Spectrum Data array}

Type: Functional, Dynamic, Unit.
					
Initial State: 
					
Input: 1D data array, present in memory. For example, the following data may be
used to initialize a Spectrum:
\begin{itemize}
	\item $\text{intensity} = \text{ones}(\text{size}(10))\ \text{counts}$
	\item $\text{energy range} = \text{range}(0, 10)\ \si{\electronvolt}$
\end{itemize}

Output: Spectrum stored within \progname{}, with data assigned to appropriate
values.
					
How test will be performed: Create a 1D data array and attempt to read it with
\progname{}. No errors should be raised. The unit test will check that the
intensity and energy range are assigned to the appropriate variables, as well as
the spectral calibration.


\item{Spectrum .csv file}

Type: Functional, Dynamic, Unit.

Initial State: 

Input: .csv file, containing a 1D spectrum. The following data may be used to
create a spectrum csv file:
\begin{itemize}
	\item $\text{Intensity} = \text{ones}(\text{size}(10))\ \text{counts}$
	\item $\text{energy range} = \text{range}(0, 10)\ \si{\electronvolt}$
\end{itemize}

Output: Spectrum stored within \progname{}, with data assigned to appropriate
values

How test will be performed: A csv file containing 1D spectrum data will be
imported. No errors should be raised. The unit test will check that the
intensity and energy range are assigned to the appropriate variables, as well as
the spectral calibration.


\item{Spectrum Data array - complex numbers}

Type: Functional, Dynamic, Unit.

Initial State: 

Input: 1D data array, present in memory. For example, the following data may be
used to initialize a Spectrum:
\begin{itemize}
	\item $\text{energy values} = \text{ones}(\text{size}(10)) +
j*\text{ones}(\text{size}(10))\ \text{counts}$
	\item $\text{energy range} = \text{range}(0, 10)\ \si{\electronvolt}$
\end{itemize}

Output: Error message - input must be real

How test will be performed: Create a 1D data array and attempt to read it with
\progname{}. An error should be raised.


\item{Spectrum Data array - 2D array}

Type: Functional, Dynamic, Unit.

Initial State: 

Input: 2D data array, present in memory. For example, the following data may be
used to initialize a Spectrum:
\begin{itemize}
	\item $\text{energy values} = \text{ones}(\text{size}(2, 2))\ \text{counts}$
	\item $\text{energy range} = \text{range}(0, 2)\ \si{\electronvolt}$
\end{itemize}

Output: Error message - input must be 1D

How test will be performed: Create a 1D data array and attempt to read it with
\progname{}. An error should be raised.

\end{enumerate}

\subsection{Data Extraction}
This section details the unit tests concerning the extraction of images or
spectra from a spectrum image (\hyperref[R_SI_slicing]{R4},
\hyperref[R_SI_area]{R5})
\paragraph{Energy range slicing}

The spectrum image to be used for these tests will be the same for all tests in
this section and may, for example, consist of float values from 0 to 1, reshaped
into the size $(2,2,10)$, with an energy range from $0-9\ \si{\electronvolt}$.

\begin{enumerate}

\item{Extract $(x,y)$ slice from spectrum}

Type: Functional, Dynamic, Unit

Initial State: Spectrum image exists in \progname{}.

Input: Two values within the spectrum range calibration values. For example, in
a spectrum image with an energy range from $0-9\ \si{\electronvolt}$, the input
values might be 5 and 7. the existing spectrum image must have different values
of intensity at each pixel and each channel along the energy axis.

Output: An $(x,y)$ image averaged over the spectral range specified by the input
to the test.

How test will be performed: Using a 3D dataset, such as those used in the
Spectrum image 3D dataset unit test, the test will request the return value of
an $(x,y)$ image averaged over the chosen spectral range. No errors should be
raised. Check that the range averaged over is the same as the range selected.


\item{Extract $(x,y)$ slice from spectrum, slice less than any value in the
energy range}

Type: Functional, Dynamic, Unit

Initial State: Spectrum image exists in \progname{}.

Input: Two values within the spectrum range calibration values. For example, in
a spectrum image with an energy range from $0-9\ \si{\electronvolt}$, the input
values might be -2 and 5. the existing spectrum image must have different values
of intensity at each pixel and each channel along the energy axis.

Output: An $(x,y)$ image averaged over the spectral range specified by the input
to the test and the minimum value of the energy range in the spectrum image.

How test will be performed: Using a 3D dataset, such as those used in the
Spectrum image 3D dataset unit test, the test will request the return value of
an $(x,y)$ image averaged over the chosen spectral range. No errors should be
raised. Check that the range averaged over is the same as the range selected
within the data limits.


\item{Extract $(x,y)$ slice from spectrum, slice more than any value in the
energy range}

Type: Functional, Dynamic, Unit

Initial State: Spectrum image exists in \progname{}.

Input: Two values within the spectrum range calibration values. For example, in
a spectrum image with an energy range from $0-9\ \si{\electronvolt}$, the input
values might be 5 and 12. the existing spectrum image must have different values
of intensity at each pixel and each channel along the energy axis.

Output: An $(x,y)$ image averaged over the spectral range specified by the input
to the test and the maximum value of the energy range in the spectrum image.

How test will be performed: Using a 3D dataset, such as those used in the
Spectrum image 3D dataset unit test, the test will request the return value of
an $(x,y)$ image averaged over the chosen spectral range. No errors should be
raised. Check that the range averaged over is the same as the range selected
within the data limits.


\item{Extract $(x,y)$ slice from spectrum, slice completely outside the energy
range}

Type: Functional, Dynamic, Unit

Initial State: Spectrum image exists in \progname{}.

Input: Two values within the spectrum range calibration values. For example, in
a spectrum image with an energy range from $0-9\ \si{\electronvolt}$, the input
values might be -5 and -2. the existing spectrum image must have different
values of intensity at each pixel and each channel along the energy axis.

Output: A warning should be raised that the range selected is outside the range
for which there is data available.

How test will be performed: Using a 3D dataset, such as those used in the
Spectrum image 3D dataset unit test, the test will request the return value of
an $(x,y)$ image averaged over the chosen spectral range. A warning should be
raised and no image returned.


\item{Extract $(x,y)$ slice from spectrum, backwards slice}

Type: Functional, Dynamic, Unit

Initial State: Spectrum image exists in \progname{}.

Input: Two values within the spectrum range calibration values. For example, in
a spectrum image with an energy range from $0-9\ \si{\electronvolt}$, the input
values might be 7 and 5. the existing spectrum image must have different values
of intensity at each pixel and each channel along the energy axis.

Output: An $(x,y)$ image averaged over the spectral range specified by the input
to the test.

How test will be performed: Using a 3D dataset, such as those used in the
Spectrum image 3D dataset unit test, the test will request the return value of
an $(x,y)$ image averaged over the chosen spectral range (regardless of the
order of selection, the program should behave exactly the same as if the input
range was 5 to 7). No errors should be raised. Check that the range averaged
over is the same as the range selected (ie, by knowing what data the spectrum
image contains).

\end{enumerate}

\paragraph{Area spectrum extraction}

The spectrum image to be used for these tests will be the same for all tests in
this section and may, for example, consist of float values from 0 to 1, reshaped
into the size (100, 100, 10), with an energy range from $0-9\
\si{\electronvolt}$.

\begin{enumerate}

\item{Extract area - square}

Type: Functional, Dynamic, Unit

Initial State: Spectrum image exists in \progname{}.

Input: Four $(x,y)$ values defining the vertices of a square (for instance,
$(20, 20)$, $(20, 35)$, $(35, 35)$, $(35, 20)$). A spectrum image, as defined
above for this section.

Output: An spectrum, averaged over the $(x,y)$ area defined.

How test will be performed: Using a 3D dataset, the test will request the return
value of a spectrum averaged over the chosen area. No errors should be raised.
Check that the correct area was used in the average (correct solution is known
by manual deduction from the spectrum image).

\item{Extract area - triangle}

Type: Functional, Dynamic, Unit

Initial State: Spectrum image exists in \progname{}.

Input: A range of $(x,y)$ values defining a triangle (for instance, $(20, 20)$,
$(35, 35)$, $(35, 20)$). A spectrum image, as defined above for this section.

Output: A spectrum, averaged over the $(x,y)$ values defined.

How test will be performed: Using a 3D dataset, the test will request the return
value of a spectrum averaged over the chosen area. No errors should be raised.
Check that the correct area was used in the average (correct solution is known
by manual deduction from the spectrum image).

\item{Extract area - line}

Type: Functional, Dynamic, Unit

Initial State: Spectrum image exists in \progname{}.

Input: A range of $(x,y)$ values defining a line (for instance, $(20, 20)$,
$(35, 35)$). A spectrum image, as defined above for this section.

Output: A warning that a 2D area does not exist when only two points are used to
define the area.

How test will be performed: Using a 3D dataset, the test will request the return
value of a spectrum averaged over the chosen area. A warning should be raised
and nothing returned.

\item{Extract area - outside image}

Type: Functional, Dynamic, Unit

Initial State: Spectrum image exists in \progname{}.

Input: A range of $(x,y)$ values defining a line (for instance, $(-20, 20)$,
$(35, 135)$). A spectrum image, as defined above for this section.

Output: A spectrum, averaged over the $(x,y)$ values defined, within the image
dimensions. Dimensions outside the image should be rounded to the closest value
within the image range (0-99).

How test will be performed: Using a 3D dataset, the test will request the return
value of a spectrum averaged over the chosen area.  No errors should be raised.
Check that the correct area was used in the average (correct solution is known
by manual deduction from the spectrum image). The program should behave as
though any image ranges outside the available range are at the closest border of
the image (\textit{e.g.} -20 rounds to 0, 135 rounds to 99).

\end{enumerate}

\subsection{Normalization}
This section details the unit tests concerning the normalization of a spectrum
(\hyperref[normalization]{IM1}, \hyperref[R_normalization]{R7})
		
\paragraph{Spectrum normalization}

\begin{enumerate}

\item{Normalization - integral}

Type: Functional, Dynamic, Unit

Initial State: None

Input: 1D spectrum object, with an energy range from $0-9\ \si{\electronvolt}$
and intensity values of 1 at each spectrum channel.

Output: Normalized spectrum, with an intensity value of 0.1 in each spectrum
channel.

How test will be performed: Feed the spectrum object into the normalization
function and request normalization to the full integral. No errors should be
raised and a normalized spectrum should be returned.


\item{Normalization - select channel}

Type: Functional, Dynamic, Unit

Initial State: None

Input: 1D spectrum object, with an energy range from $0-9\ \si{\electronvolt}$
and intensity values of 1 at each spectrum channel, except for the fifth
channel, which will have a value of 10. 

Output: Normalized spectrum, with an intensity value of 1 in the fifth channel
and a value of 0.1 everywhere else.

How test will be performed: Feed the spectrum object into the normalization
function and request normalization to the fifth channel. No errors should be
raised and a normalized spectrum should be returned.


\item{Normalization - integral is zero}

Type: Functional, Dynamic, Unit

Initial State: None

Input: 1D spectrum object, with an energy range from $0-9\ \si{\electronvolt}$
and intensity values of 0 at each spectrum channel. 

Output: Error message, stating that normalization cannot be performed on this
spectrum because of a divide-by-zero error.

How test will be performed: Feed the spectrum object into the normalization
function and request normalization to the fifth channel. An error should be
raised.

\end{enumerate}

\subsection{Background Correction}
This section details the unit tests concerning the background correction of a
spectrum (\hyperref[background]{IM3}, \hyperref[R_background]{R8})
\paragraph{Spectrum-background correction}

\begin{enumerate}

\item{Background subtraction}

Type: Functional, Dynamic, Unit

Initial State: None

Input: 1D spectrum object, with an energy range from $0-9\ \si{\electronvolt}$
and intensity values of 1 at each spectrum channel. Background spectrum, with
the same energy range and intensity values of 0.1 at each spectrum channel.

Output: Corrected spectrum, with an intensity value of 0.9 in each spectrum
channel.

How test will be performed: Feed the spectrum object and background spectrum
object into the background correction function. No errors should be raised and a
background-corrected spectrum should be returned.

\item{Background subtraction - different size}

Type: Functional, Dynamic, Unit

Initial State: None

Input: 1D spectrum object, with an energy range from $0-9\ \si{\electronvolt}$
and intensity values of 1 at each spectrum channel. Background spectrum, with an
energy range of $0-8\ \si{\electronvolt}$ and intensity values of 0.1 at each
spectrum channel (one channel shorter than the 1D spectrum object).

Output: An error message, stating that the background correction spectrum is not
the right size and the correction cannot be applied.

How test will be performed: Feed the spectrum object and background spectrum
object into the background correction function. An error should be raised.

\end{enumerate}

\subsection{Gain Correction}
This section details the unit tests concerning the gain correction of a spectrum
(\hyperref[gain]{IM4}, \hyperref[R_gain]{R9})

\paragraph{Spectrum-gain correction}

\begin{enumerate}

\item{Gain correction}

Type: Functional, Dynamic, Unit

Initial State: None

Input: 1D spectrum object, with an energy range from $0-9 \si{\electronvolt}$
and intensity values of 1 at each spectrum channel. Gain correction spectrum,
with the same energy range and intensity values of 0.1 at each spectrum channel.

Output: Corrected spectrum, with an intensity value of 0.1 in each spectrum
channel.

How test will be performed: Feed the spectrum object and gain correction
spectrum object into the gain correction function. No errors should be raised
and a gain-corrected spectrum should be returned.


\item{Gain correction - zeros}

Type: Functional, Dynamic, Unit

Initial State: None

Input: 1D spectrum object, with an energy range from $0-9 \si{\electronvolt}$
and intensity values of 1 at each spectrum channel. Gain correction spectrum,
with the same energy range and intensity values of 0 at each spectrum channel.

Output: Corrected spectrum, with an intensity value of 0 in each spectrum
channel and a Warning that the gain correction is composed of zeros and so
nullifies the whole spectrum.

How test will be performed: Feed the spectrum object and gain correction
spectrum object into the gain correction function. A warning should be raised
and a gain-corrected spectrum should be returned.

\end{enumerate}


\bibliography {TestPlan}

\newpage

\section{Appendix}

This appendix contains questions for a Code Review and questions for a user
survey.

\subsection{Code Review Checklist}
This section contains some questions to be covered during a Code Review process.
An effort was made to keep the number of items low to allow more time for a
thorough review of these items without requiring an overly lengthy time
commitment from the reviewer. Questions were inspired by
\cite{office_of_safety_and_mission_assurance_software_1993,
karl_e._wiegers_generic_2001, erik_dietrich_creating_2015}.
\begin{itemize}
\item Summarize in your own words what each function or unit of code does, based
on inspection of the code. 
\item Is the documentation clear and readable for each function or unit of code?
\item Does the described functionality match with your written description?
\item Are all variables and functions named clearly and meaningfully?
\item Are all cases covered in IF/ELSEIF loops?
\item Do all branches in conditionals make sense (logic and resulting action is
correct)?
\item Are the specifications and requirements complete and correctly
implemented?
\item Is the user interface clear and easy to use?
\item Does each unit of code have a single purpose, or can it be split into
multiple functions?
\item What is the expected memory usage for each function? Can it be reduced?
(\textit{e.g.} are unnecessary copies of arrays being created?)
\item Are array slicing references in the correct order? (\textit{e.g.}, check
correct dimension is used)
\item Does it appear to be easy to add new modules or functions?
\end{itemize}

\subsection{User Survey}
This section contains some sample questions which may be used to create a user
survey of the software. Effort was made to keep the number of questions small
and to address non-functional requirements.
\begin{itemize}
\item What operating system do you use?
\item Did the software experience any noticeable lag while you were using the
GUI interface?
\item Where you able to use the interactive display easily, or did you find the
commands difficult to remember and use?
\item Could you easily read everything you needed to on the data display?
(please consider font, font size, colour, plot lines, colour maps, etc.)
\item What aspects of using the software did you find most intuitive? Least
intuitive?
\item Do you think you will be able to use this software given only the
documentation available and no other training?
\item On a scale of 1-10, did you find the software easy to use? (0: impossible
to use; 10: very simple and intuitive)
\item On a scale of 1-10, did you find the software enjoyable to use? (0: hated
the software and resented the waste of time; 10: fully enjoyed the experience
and I look forward to using this to process my data again)
\item Do you think the software will be useful to you in your work?
\item Do you have any comments or suggestions for further improvements? What
would make this software more useful to you?
\end{itemize}

\wss{Great work!}

\end{document}
