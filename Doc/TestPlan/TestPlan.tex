\documentclass[12pt, titlepage]{article}

\usepackage{booktabs}
\usepackage{siunitx}
\usepackage{tabularx}
\usepackage{xr-hyper}

\usepackage{hyperref}
\hypersetup{
    colorlinks,
    citecolor=black,
    filecolor=black,
    linkcolor=red,
    urlcolor=blue
}

\externaldocument{../SRS/SRS}

\newcommand{\progname}{SpectrumImageAnalysisPy}
\newcommand{\rthereqnum}{P\thereqnum}
\newcommand{\ddref}[1]{DD\ref{#1}}

%% Comments

\usepackage{color}

\newif\ifcomments\commentstrue

\ifcomments
\newcommand{\authornote}[3]{\textcolor{#1}{[#3 ---#2]}}
\newcommand{\todo}[1]{\textcolor{red}{[TODO: #1]}}
\else
\newcommand{\authornote}[3]{}
\newcommand{\todo}[1]{}
\fi

\newcommand{\wss}[1]{\authornote{blue}{SS}{#1}}
\newcommand{\an}[1]{\authornote{magenta}{Author}{#1}}


\begin{document}
\bibliographystyle{ieeetr}
\title{SpectrumImageAnalysisPy} 
\author{Isobel Bicket}
\date{\today}
	
\maketitle

\pagenumbering{roman}

\section{Revision History}

\begin{tabularx}{\textwidth}{p{4cm}p{2cm}X}
\toprule {\bf Date} & {\bf Version} & {\bf Notes}\\
\midrule
\today & 1.0 & Initial draft\\
\bottomrule
\end{tabularx}

~\newpage

\section{Symbols, Abbreviations and Acronyms}

\renewcommand{\arraystretch}{1.2}
\begin{tabular}{l l} 
  \toprule		
  \textbf{symbol} & \textbf{description}\\
  \midrule 
  T & Test\\
  \bottomrule
\end{tabular}\\

\wss{symbols, abbreviations or acronyms -- you can reference the SRS tables if needed}

\newpage

\tableofcontents

\listoftables

\listoffigures

\newpage

\pagenumbering{arabic}

This document ...

\section{General Information}

\subsection{Purpose}

\subsection{Scope}

\subsection{Overview of Document}

\section{Plan}
	
\subsection{Software Description}

\subsection{Test Team}

The test team has one member: Isobel Bicket.

\subsection{Automated Testing Approach}

Unit testing will be performed on the functions within the code, as an automated task to be done with every update of the code. The running of automated tests also provides regression testing and integration testing of new features and updates. The goal of testing is 100\% code coverage.

\subsection{Verification Tools}

The following verification tools will be used
\begin{itemize}
	\item Python unittest library: for performing unit tests on the code;
	\item Python coverage library: to determine the coverage of the tests;
	\item Deconvo.m: a Richardson-Lucy deconvolution algorithm implementation in Matlab, written by Dr. E.P. Bellido \cite{bellido_toward_2014};
	\item Python HypersPy library: a Python-based library for EELS and EDX spectrum processing.
\end{itemize}
\wss{Thoughts on what tools to use, such as the following: unit testing
  framework, valgrind, static analyzer, make, continuous integration, test
  coverage tool, etc.}

% \subsection{Testing Schedule}
		
% See Gantt Chart at the following url ...

\subsection{Non-Testing Based Verification}

The following non-testing based verification methods will be used to assess the performance of \progname{}

\begin{itemize}
	\item Code review: a detailed code review will be performed by an external party and feedback provided;
	\item User survey: a group of qualified users will be asked to process a sample dataset using the software and will be polled on their user experience.
\end{itemize}

\wss{List any approaches like code inspection, code walkthrough, symbolic
  execution etc.  Enter not applicable if that is the case.}

\section{System Test Description}
	
\subsection{Tests for Functional Requirements}

\subsubsection{Inputs}
		
\paragraph{Spectrum Image Input}

\begin{enumerate}

\item{SI Data array\\}

Type: Functional, Dynamic, Manual, Static etc.
					
Initial State: 
					
Input: 
					
Output: 
					
How test will be performed: 


\end{enumerate}

\paragraph{}

\begin{enumerate}

\item{}

Type: Functional, Dynamic, Manual, Static etc.
					
Initial State: \ddref{Spectrum}
					
Input: 
					
Output: 
					
How test will be performed: 



\end{enumerate}


\subsubsection{Area of Testing2}

...

\subsection{Tests for Nonfunctional Requirements}

\subsubsection{Area of Testing1}
		
\paragraph{Title for Test}

\begin{enumerate}

\item{test-id1\\}

Type: 
					
Initial State: 
					
Input/Condition: 
					
Output/Result: 
					
How test will be performed: 
					
\item{test-id2\\}

Type: Functional, Dynamic, Manual, Static etc.
					
Initial State: 
					
Input: 
					
Output: 
					
How test will be performed: 

\end{enumerate}

\subsubsection{Area of Testing2}

...

\subsection{Traceability Between Test Cases and Requirements}

% \section{Tests for Proof of Concept}

% \subsection{Area of Testing1}
		
% \paragraph{Title for Test}

% \begin{enumerate}

% \item{test-id1\\}

% Type: Functional, Dynamic, Manual, Static etc.
					
% Initial State: 
					
% Input: 
					
% Output: 
					
% How test will be performed: 
					
% \item{test-id2\\}

% Type: Functional, Dynamic, Manual, Static etc.
					
% Initial State: 

% Input: 
					
% Output: 
					
% How test will be performed: 

% \end{enumerate}

% \subsection{Area of Testing2}

% ...
				
\section{Unit Testing Plan}

\subsection{Inputs}
		
\paragraph{Spectrum Image Input}

\begin{enumerate}

\item{SI Data array with energy range (EELS)}

Type: Functional, Dynamic, Unit

Initial State: None

Input: 3D data array, present in memory; use standard calibration variabilities\\
For example, the following values might be used to create an SI: 
\begin{itemize}
	\item $x = [0, 1]$
	\item $y = [0, 1]$
	\item $\text{x calibration} = 1\ \si{\nano\metre}/\text{pixel}$
	\item $\text{energy range} = range(0, 10) \si{\electronvolt}$
\end{itemize}

Output: Spectrum Image stored within \progname{}; the first and second axes should correspond to spatial dimensions, while the third axis should correspond to a spectral dimension.

How test will be performed: Create a 3D data array and attempt to initialize a spectrum image in \progname{} using a variety of calibration possibilities. No errors should be raised. The spectrum image axes should be read in the correct order (x, y, E), and the calibrations should be applied to the correct axes.

\item{SI Data array with dispersion (EELS)}

Type: Functional, Dynamic, Unit

Initial State: None

Input: 3D data array, present in memory; use standard calibration variabilities\\
For example, the following values might be used to create an SI: 
\begin{itemize}
	\item $x = [0, 1]$
	\item $y = [0, 1]$
	\item $\text{x calibration} = 1\ \si{\nano\metre}/\text{pixel}$
	\item $\text{dispersion} = 0.01\ \si{\electronvolt}/\text{pixel}$
	\item $E(0) = -2 \si{\electronvolt}$
\end{itemize}

Output: Spectrum Image stored within \progname{}; the first and second axes should correspond to spatial dimensions, while the third axis should correspond to a spectral dimension.

How test will be performed: Create a 3D data array and attempt to initialize a spectrum image in \progname{} using a variety of calibration possibilities. No errors should be raised. The spectrum image axes should be read in the correct order (x, y, E), and the calibrations should be applied to the correct axes.


\item{SI Data array with wavelength range (CL)}

Type: Functional, Dynamic, Unit

Initial State: None

Input: 3D data array, present in memory; use standard calibration variabilities\\
For example, the following values might be used to create an SI: 
\begin{itemize}
	\item $x = [0, 1]$
	\item $y = [0, 1]$
	\item $\text{x calibration} = 1\ \si{\nano\metre}/\text{pixel}$
	\item $\text{wavelength range} = range(0, 10) \si{\nano\metre}$
\end{itemize}

Output: Spectrum Image stored within \progname{}; the first and second axes should correspond to spatial dimensions, while the third axis should correspond to a spectral dimension.

How test will be performed: Create a 3D data array and attempt to initialize a spectrum image in \progname{} using a variety of calibration possibilities. No errors should be raised. The spectrum image axes should be read in the correct order (x, y, E), and the calibrations should be applied to the correct axes.


\item{SI .dm3 file}

Type: Functional, Dynamic, Unit

Initial State: None

Input: A .dm3 file, containing a fabricated spectrum image. The .dm3 file will be created using Digital Micrograph (Gatan Microscopy Suite Software) software \cite{noauthor_gatan_nodate} with the following parameters:
\begin{itemize}
	\item $x = [0, 1]$
	\item $y = [0, 1]$
	\item $\text{x calibration} = 1\ \si{\nano\metre}/\text{pixel}$
	\item $\text{dispersion} = 0.01\ \si{\electronvolt}/\text{pixel}$
	\item $E(0) = -2 \si{\electronvolt}$
\end{itemize}

Output: EELS Spectrum Image stored within \progname{}, with metadata assigned to appropriate values

How test will be performed: Read in the .dm3 file with \progname{}. Check that no errors are raised. Display the spectrum image to manually check that the data was read in correctly. Check spatial and spectral calibration to verify metadata was assigned correctly.

\an{There is a potential problem with the .dm3 file, since there are many version of Digital Micrograph out there and many different set-ups, all of which may format their .dm3 slightly differently. Not sure on the best way to test this, without asking several other labs for pieces of data (which are not usually shared readily) and trying out all of them.}

\item{SI .h5 file}

Type: Functional, Dynamic, Manual, Static etc.
					
Initial State: None
					
Input: .h5 file, containing an acquired or fabricated spectrum image. The .h5 spectrum image file originates from Odemis software \cite{bv_odemis:_nodate}, it may be possible to fabricate a spectrum image using the scripting interface, using the following values for the calibration parameters, otherwise it may be necessary to use an acquired data file.

\begin{itemize}
	\item $x = [0, 1]$
	\item $y = [0, 1]$
	\item $\text{x calibration} = 1\ \si{\nano\metre}/\text{pixel}$
	\item $\text{wavelength range} = range(0, 10) \si{\nano\metre}$
\end{itemize}
					
Output: CL Spectrum Image stored within \progname, with metadata assigned to appropriate values.
					
How test will be performed: Read in the .h5 file with \progname{}. Check that no errors are raised. Check spatial and spectral calibration to verify metadata was assigned correctly.

\end{enumerate}

\paragraph{Spectrum Input}

\begin{enumerate}

\item{Spectrum Data array\\}

Type: Functional, Dynamic, Manual, Static etc.
					
Initial State: 
					
Input: 3D data array, present in memory
					
Output: Spectrum Image stored within \progname{}; the first and second axes should correspond to spatial dimensions, while the third axis should correspond to a spectral dimension.
					
How test will be performed: Create a 3D data array and attempt to read it with \progname{}. Check that no errors are raised. Display the spectrum image to manually check that the data was read in correctly.


\item{Spectrum .csv file\\}

Type: Functional, Dynamic, Manual, Static etc.

Initial State: 

Input: .dm3 file, containing an acquired spectrum image

Output: Spectrum Image stored within \progname{}, with metadata assigned to appropriate values

How test will be performed: 

\end{enumerate}

\wss{Unit testing plans for internal functions and, if appropriate, output
  files}

\bibliography {TestPlan}

\newpage

\section{Appendix}

This is where you can place additional information.

\subsection{Symbolic Parameters}

The definition of the test cases will call for SYMBOLIC\_CONSTANTS.
Their values are defined in this section for easy maintenance.

\subsection{Usability Survey Questions?}

This is a section that would be appropriate for some teams.

\end{document}