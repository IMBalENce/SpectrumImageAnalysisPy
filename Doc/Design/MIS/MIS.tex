\documentclass[12pt, titlepage]{article}

\usepackage{amsmath, mathtools}

\usepackage{amsfonts}
\usepackage{amssymb}
\usepackage{graphicx}
\usepackage{colortbl}
\usepackage{xr}
\usepackage{xr-hyper}
\usepackage{hyperref}
\usepackage{longtable}
\usepackage{xfrac}
\usepackage{tabularx}
\usepackage{float}
\usepackage{siunitx}
\usepackage{booktabs}
\usepackage{multirow}
\usepackage[section]{placeins}
\usepackage{caption}
\usepackage{fullpage}


\hypersetup{
    bookmarks=true,     % show bookmarks bar?
    colorlinks=true,       % false: boxed links; true: colored links
    linkcolor=red,          % color of internal links (change box color with
    %linkbordercolor)
    citecolor=blue,      % color of links to bibliography
    filecolor=magenta,  % color of file links
    urlcolor=cyan          % color of external links
}
\externaldocument{../../SRS/SRS}
\usepackage{array}

%% Comments

\usepackage{color}

\newif\ifcomments\commentstrue

\ifcomments
\newcommand{\authornote}[3]{\textcolor{#1}{[#3 ---#2]}}
\newcommand{\todo}[1]{\textcolor{red}{[TODO: #1]}}
\else
\newcommand{\authornote}[3]{}
\newcommand{\todo}[1]{}
\fi

\newcommand{\wss}[1]{\authornote{blue}{SS}{#1}}
\newcommand{\an}[1]{\authornote{magenta}{Author}{#1}}


\newcommand{\progname}{SpectrumImageAnalysisPy}

\begin{document}
\label{doc:MIS}
\bibliographystyle{ieeetr}
\title{Module Interface Specification for \progname{}}

\author{Isobel Bicket}

\date{\today}

\maketitle

\pagenumbering{roman}

\section{Revision History}

\begin{tabularx}{\textwidth}{p{4cm}p{2cm}X}
\toprule {\bf Date} & {\bf Version} & {\bf Notes}\\
\midrule
	November 29, 2017 & 1.0 & Initial draft\\
\bottomrule
\end{tabularx}

~\newpage

\section{Symbols, Abbreviations and Acronyms}

See \hyperref[doc:SRS]{SRS} documentation at
\url{https://github.com/icbicket/SpectrumImageAnalysisPy/blob/SpectrumImageAnalysisPy_dev/Doc/SRS/SRS.pdf}.

\newpage

\tableofcontents

\newpage

\pagenumbering{arabic}

\section{Introduction}

The following document details the Module Interface Specifications for
\progname, a library created for the data processing of spectrum image datasets.

Complementary documents include the System Requirement Specifications
and Module Guide.  The full documentation and implementation can be
found at
\url{https://github.com/icbicket/SpectrumImageAnalysisPy/tree/SpectrumImageAnalysisPy_dev}.

All modules within \progname{} are accessible by the user from terminal
commands. Some modules will interface with each other, but the workflow of
\progname is driven by the user.

\section{Notation}
The structure of the MIS for modules comes from \cite{HoffmanAndStrooper1995},
with the addition that template modules have been adapted from
\cite{GhezziEtAl2003}.  The mathematical notation comes from Chapter 3 of
\cite{HoffmanAndStrooper1995}.  For instance, the symbol := is used for a
multiple assignment statement and conditional rules follow the form $(c_1
\Rightarrow r_1 | c_2 \Rightarrow r_2 | ... | c_n \Rightarrow r_n )$.

The following table summarizes the primitive data types used by \progname. 

\begin{center}
    \renewcommand{\arraystretch}{1.2}
    \noindent 
    \begin{tabular}{l l p{7.5cm}} 
        \toprule 
        \textbf{Data Type} & \textbf{Notation} & \textbf{Description}\\ 
        \midrule
        character & char & a single symbol or digit\\
        string & str & a sequence of characters\\
        integer & $\mathbb{Z}$ & a number without a fractional component in (-$\infty$,
        $\infty$) \\
        natural number & $\mathbb{N}$ & a number without a fractional component in [1,
        $\infty$) \\
        real & $\mathbb{R}$ & any number in (-$\infty$, $\infty$)\\
        complex & $\mathbb{C}$ & any combination of real and imaginary numbers, in the
        form $a+bi$, where $a$ and $b$ are real and $i$ is the imaginary number\\
        \bottomrule
    \end{tabular} 
\end{center}

\noindent
The specification of \progname \ uses some derived data types: sequences,
strings, and
tuples. Sequences are lists filled with elements of the same data type. Strings
are sequences of characters. Tuples contain a list of values, potentially of
different types. In addition, \progname \ uses functions, which
are defined by the data types of their inputs and outputs. Local functions are
described by giving their type signature followed by their specification. It
should be noted that each state variable is assumed to have a setter and getter
accessible through module.variable\_name.

\section{Module Decomposition}

The following table is taken directly from the Module Guide document for this
project. The modules in this document are listed in the order in which they
appear in this table.

\begin{table}[h!]
    \centering
    \begin{tabular}{p{0.25\textwidth} p{0.25\textwidth} p{0.4\textwidth}}
\toprule
\textbf{Level 1} & \textbf{Level 2} & \textbf{Level 3}\\
\midrule

{Hardware-Hiding Module} & ~ & ~ \\
\midrule

\multirow{18}{0.25\textwidth}{Behaviour-Hiding Module} &
\multirow{4}{0.25\textwidth}{Import} & csv\\
& & dm3\\
& & h5\\
& & rpl\\\cline{2-3}
& \multirow{4}{0.25\textwidth}{Export} & csv\\
& & h5\\
& & png\\
& & rpl\\\cline{2-3}
& \multirow{4}{0.25\textwidth}{Data processing} & Richardson-Lucy
Deconvolution\\
& & Normalization\\
& & Gain correction\\
& & Background correction\\\cline{2-3}
& \multirow{3}{0.25\textwidth}{Data extraction} & 1D slice\\
& & 2D mask\\
& & 3D mask\\\cline{2-3}
& \multirow{3}{0.25\textwidth}{Display} & 1D spectrum plot\\
& & 2D image plot\\
& & 3D spectrum image plot\\
\midrule

\multirow{2}{0.25\textwidth}{Software Decision Module} &
\multirow{3}{0.25\textwidth}{Data} & Spectrum\\
& & Image\\
& & Spectrum Image\\\cline{2-3}
& Array Data Structure\\\cline{2-3}
& Plotting Library\\

\bottomrule

\end{tabular}
    \caption{Module Hierarchy}
    \label{TblMH}
\end{table}

\newpage
~\newpage

\section{MIS of Hardware Hiding Module} \label{Mod:HH}

\subsection{Module}
HardwareHiding

\subsection{Uses}
N/A

\subsection{Syntax}

\subsubsection{Exported Access Programs}

\begin{center}
    \begin{tabular}{p{3cm} p{4cm} p{4cm} p{2cm}}
        \hline
        \textbf{Name} & \textbf{In} & \textbf{Out} & \textbf{Exceptions} \\
        \hline
        InputDevices & Hardware & Read out & - \\
        OutputDevices & Hardware & Write out & - \\
        \hline
    \end{tabular}
\end{center}

\subsection{Semantics}
This module handles the interface between the hardware being used and inputs to
the software

\subsubsection{State Variables}
N/A

\subsubsection{Environment Variables}
\begin{itemize}
    \item Keyboard
    \item Mouse
    \item Screen
    \item Long Term Storage
    \item Temporary storage
\end{itemize}

\subsubsection{Access Routine Semantics}

\noindent InputDevices():
\begin{itemize}
    \item input: Hardware allowing the user to input instructions to the
    computer software, \textit{e.g.} mouse, keyboard, long term or temporary memory
    \item transition: N/A
    \item output: Software instructions corresponding to the desire of the user
    (\textit{e.g.} registering a mouse click, reading a variable from memory,
    accessing a file on the harddrive)
    \item exception: N/A
\end{itemize}

\noindent OutputDevices():
\begin{itemize}
    \item input: Hardware allowing the user to see output from the computer
    software, \textit{e.g.} screen, storage
    \item transition: N/A
    \item output: Interface to allow software to communicate output to the user
    (\textit{e.g.}, it provides the capability for the software to output something
    onto the screen or write to a file on a harddrive, or write to memory)
    \item exception: N/A
\end{itemize}

\section{MIS of Import csv Module} \label{Mod:ImportCSV} 
\subsection{Module}
ImportCSV

\subsection{Uses}
\begin{itemize}
    \item \hyperref[Mod:Spectrum]{Data 1D Spectrum}
    \item \hyperref[Mod:Array]{Array data structure}
    \item \hyperref[Mod:HH]{Hardware-hiding}
\end{itemize}

\subsection{Syntax}

\subsubsection{Exported Access Programs}

\begin{center}
    \begin{tabular}{p{2cm} p{4cm} p{4cm} p{5cm}}
        \hline
        \textbf{Name} & \textbf{In} & \textbf{Out} & \textbf{Exceptions} \\
        \hline
        ReadCSV & fname: str & Spectrum & NO FILE, NOT CSV\\
        \hline
    \end{tabular}
\end{center}

\subsection{Semantics}
This module imports data from csv files and initializes a Spectrum object.

\subsubsection{State Variables}
N/A

\subsubsection{Environment Variables}
filesystem

\subsubsection{Access Routine Semantics}

\noindent ReadCSV():\\
ReadCSV reads a .csv file and creates a Spectrum object with the appropriate
assignations to intensity and energy range.
\begin{itemize}
    \item input: \textit{fname}: \textit{str}
    \item transition: N/A 
    \item output: \hyperref[Mod:Spectrum]{Spectrum}
    \item exceptions:
\end{itemize}
\begin{center}
    \begin{tabular}{p{3cm} p{12cm}}
        \toprule[0.15em]
        \textbf{Exception} & \textbf{Condition}\\
        \midrule[0.1em]
        \multirow{2}{0.25\textwidth}{NO FILE} & The filename does not correspond to
        any file in the filesystem\\ 
        & $fname \notin filesystem$\\ 
        \midrule[0.05em]
        \multirow{2}{0.25\textwidth}{NOT CSV} & The indicated file is not a *.csv
        format\\
        & $fname \notin \{files|files \in .csv\}$\\ 
        \bottomrule[0.15em]
    \end{tabular}
\end{center}


\section{MIS of Import dm3 Module} \label{Mod:ImportDM3}

\subsection{Module}

ImportDM3

\subsection{Uses}
\begin{itemize}
    \item \hyperref[Mod:Array]{Array data structure}
    \item \hyperref[Mod:HH]{Hardware hiding}
    \item \hyperref[Mod:SI]{Data Spectrum Image}
    \item \hyperref[Mod:Spectrum]{Data 1D Spectrum}
    \item \hyperref[Mod:Image]{Data 2D Image}
\end{itemize}

\subsection{Syntax}

\subsubsection{Exported Access Programs}

\begin{center}
    \begin{tabular}{p{2cm} p{4cm} p{4cm} p{4cm}}
        \hline
        \textbf{Name} & \textbf{In} & \textbf{Out} & \textbf{Exceptions} \\
        \hline
        ReadDM3 & fname: string & SI $|$ Spectrum $|$ Image & NO FILE, NOT DM3 \\
        \hline
    \end{tabular}
\end{center}

\subsection{Semantics}
This module imports data from .dm3 and initializes the appropriate data type.

\subsubsection{State Variables}
N/A

\subsubsection{Environment Variables}
\begin{itemize}
    \item filesystem: the filesystem of the computer on which \progname{} is being
    run
\end{itemize}
\subsubsection{Access Routine Semantics}

\noindent ImportDM3():
\begin{itemize}
    \item input: \textit{fname}: \textit{str}
    \item transition: N/A
    \item output: \hyperref[Mod:SI]{Spectrum Image} or
    \hyperref[Mod:Spectrum]{Spectrum} or \hyperref[Mod:Image]{Image}
    \item exception: 
    \begin{center}
        \begin{tabular}{p{3cm} p{12cm}}
            \toprule[0.15em]
            \textbf{Exception} & \textbf{Condition}\\
            \midrule[0.1em]
            \multirow{2}{0.25\textwidth}{NO FILE} & The filename does not correspond
            to any file in the filesystem\\ 
            & $fname \notin filesystem$\\ 
            \midrule[0.05em]
            NOT DM3 & The indicated file is not a *.dm3 format\\
            \bottomrule[0.15em]
        \end{tabular}
    \end{center}
    
\end{itemize}

\section{MIS of Import h5 Module} \label{Mod:ImportH5} 

\subsection{Module}
ImportH5

\subsection{Uses}
\begin{itemize}
    \item \hyperref[Mod:Array]{Array data structure}
    \item \hyperref[Mod:HH]{Hardware hiding}
    \item \hyperref[Mod:SI]{Data Spectrum Image}
    \item \hyperref[Mod:Spectrum]{Data 1D Spectrum}
    \item \hyperref[Mod:Image]{Data 2D Image}
\end{itemize}

\subsection{Syntax}

\subsubsection{Exported Access Programs}

\begin{center}
    \begin{tabular}{p{2cm} p{4cm} p{4cm} p{2cm}}
        \hline
        \textbf{Name} & \textbf{In} & \textbf{Out} & \textbf{Exceptions} \\
        \hline
        ReadH5 & fname: string & SI $|$ Spectrum $|$ Image & NO FILE, NOT H5 \\
        \hline
    \end{tabular}
\end{center}

\subsection{Semantics}
This module handles the reading of .h5 files and assignation of the data
contained therein to the appropriate data type.

\subsubsection{State Variables}
N/A

\subsubsection{Access Routine Semantics}

\noindent ImportDM3():
\begin{itemize}
    \item input: \textit{fname}: \textit{str}
    \item transition: N/A
    \item output: \hyperref[Mod:SI]{Spectrum Image} or
    \hyperref[Mod:Spectrum]{Spectrum} or \hyperref[Mod:Image]{Image}
    \item exception: 
    \begin{center}
        \begin{tabular}{p{3cm} p{12cm}}
            \toprule[0.15em]
            \textbf{Exception} & \textbf{Condition}\\
            \midrule[0.1em]
            \multirow{2}{0.25\textwidth}{NO FILE} & The filename does not
            correspond to any file in the filesystem\\ 
            & $fname \notin filesystem \Rightarrow$ NO\_FILE\\ 
            \midrule[0.05em]
            NOT H5 & The indicated file is not a *.h5 format\\
            \bottomrule[0.15em]
        \end{tabular}
    \end{center}
\end{itemize}

\section{MIS of Import rpl Module} \label{Mod:ImportRPL}

\subsection{Module}

ImportRPL

\subsection{Uses}
\begin{itemize}
    \item \hyperref[Mod:Array]{Array data structure}
    \item \hyperref[Mod:HH]{Hardware hiding}
    \item \hyperref[Mod:SI]{Data Spectrum Image}
    \item \hyperref[Mod:Spectrum]{Data 1D Spectrum}
    \item \hyperref[Mod:Image]{Data 2D Image}
\end{itemize}

\subsection{Syntax}

\subsubsection{Exported Access Programs}

\begin{center}
    \begin{tabular}{p{2cm} p{4cm} p{4cm} p{2cm}}
        \hline
        \textbf{Name} & \textbf{In} & \textbf{Out} & \textbf{Exceptions} \\
        \hline
        ReadRPL & fname: string & SI $|$ Spectrum $|$ Image & NO FILE, NOT RPL \\
        \hline
    \end{tabular}
\end{center}

\subsection{Semantics}
This module handles the reading of .rpl files and assigns the data contained
within to the appropriate data type.

\subsubsection{State Variables}
N/A

\subsubsection{Access Routine Semantics}

\noindent ImportRPL():
\begin{itemize}
    \item input: \textit{fname}: \textit{str}
    \item transition: N/A
    \item output: \hyperref[Mod:SI]{Spectrum Image} or
    \hyperref[Mod:Spectrum]{Spectrum} or \hyperref[Mod:Image]{Image}
    \item exception: 
    \begin{center}
        \begin{tabular}{p{3cm} p{12cm}}
            \toprule[0.15em]
            \textbf{Exception} & \textbf{Condition}\\
            \midrule[0.1em]
            \multirow{2}{0.25\textwidth}{NO FILE} & The filename does not
            correspond to any file in the filesystem\\ 
            & $fname \notin filesystem \Rightarrow$ NO\_FILE\\ 
            \midrule[0.05em]
            NOT RPL & The indicated file is not a *.rpl format\\
            \bottomrule[0.15em]
        \end{tabular}
    \end{center}
\end{itemize}

\section{MIS of Export csv Module} \label{Mod:ExportCSV}

\subsection{Module}

ExportCSV

\subsection{Uses}
\begin{itemize}
    \item \hyperref[Mod:Slice1D]{Data Extraction 1D Slice}
    \item \hyperref[Mod:Mask3D]{Data Extraction 3D Mask}
    \item \hyperref[Mod:Spectrum]{Data 1D Spectrum}
    \item \hyperref[Mod:Disp1D]{Display 1D Spectrum}
    \item \hyperref[Mod:HH]{Hardware Hiding}
\end{itemize}

\subsection{Syntax}

\subsubsection{Exported Access Programs}

\begin{center}
    \begin{tabular}{p{2cm} p{4cm} p{3cm} p{4cm}}
        \hline
        \textbf{Name} & \textbf{In} & \textbf{Out} & \textbf{Exceptions} \\
        \hline
        WriteCSV & Spectrum & file & FILE EXISTS \\
        \hline
    \end{tabular}
\end{center}

\subsection{Semantics}
This module writes Spectrum data to a csv file.

\subsubsection{State Variables}
N/A

\subsubsection{Environment Variables}
\begin{itemize}
    \item filesystem
\end{itemize}

\subsubsection{Access Routine Semantics}

\noindent WriteCSV():
\begin{itemize}
    \item input: \hyperref[Mod:Spectrum]{Spectrum}
    \item transition: N/A
    \item output: csv file containing spectrum data, written to filesystem
    \item exception:
    \begin{center}
        \begin{tabular}{p{3cm} p{12cm}}
            \toprule[0.15em]
            \textbf{Exception} & \textbf{Condition}\\
            \midrule[0.1em]
            \multirow{2}{0.25\textwidth}{FILE EXISTS} & The filename already
            exists in the filesystem\\ 
            & $fname \in filesystem \Rightarrow$ FILE\_EXISTS\\ 
            \bottomrule[0.15em]
        \end{tabular}
    \end{center}
\end{itemize}

\section{MIS of Export h5 Module} \label{Mod:ExportH5}

\subsection{Module}

ExportH5

\subsection{Uses}
\begin{itemize}
    \item \hyperref[Mod:Spectrum]{Data 1D Spectrum}
    \item \hyperref[Mod:Image]{Data 2D Image}
    \item \hyperref[Mod:SI]{Data 3D Spectrum Image}
    \item \hyperref[Mod:Slice1D]{Data Extraction 1D Slice}
    \item \hyperref[Mod:Mask2D]{Data Extraction 2D Mask}
    \item \hyperref[Mod:Mask3D]{Data Extraction 3D Mask}
    \item \hyperref[Mod:Disp1D]{Display 1D Spectrum}
    \item \hyperref[Mod:Disp2D]{Display 2D Image}
    \item \hyperref[Mod:HH]{Hardware Hiding}
\end{itemize}

\subsection{Syntax}

\subsubsection{Exported Access Programs}

\begin{center}
    \begin{tabular}{p{2cm} p{4cm} p{2cm} p{4cm}}
        \hline
        \textbf{Name} & \textbf{In} & \textbf{Out} & \textbf{Exceptions} \\
        \hline
        WriteH5 & Spectrum $|$ Image $|$ Spectrum Image & file & FILE EXISTS \\
        \hline
    \end{tabular}
\end{center}

\subsection{Semantics}
This module writes Spectrum data, Image data, or Spectrum Image data to an h5
file.

\subsubsection{State Variables}
N/A

\subsubsection{Environment Variables}
\begin{itemize}
    \item filesystem
\end{itemize}

\subsubsection{Access Routine Semantics}

\noindent WriteH5():
\begin{itemize}
    \item input: \hyperref[Mod:Spectrum]{Spectrum} $|$
    \hyperref[Mod:Image]{Image} $|$ \hyperref[Mod:SI]{Spectrum Image}
    \item transition: N/A
    \item output: h5 file containing spectrum data, image data, or spectrum
    image data (including any metadata), written to filesystem
    \item exception:
    \begin{center}
        \begin{tabular}{p{3cm} p{12cm}}
            \toprule[0.15em]
            \textbf{Exception} & \textbf{Condition}\\
            \midrule[0.1em]
            \multirow{2}{0.25\textwidth}{FILE EXISTS} & The filename already
            exists in the filesystem\\ 
            & $fname \in filesystem \Rightarrow$ FILE\_EXISTS\\ 
            \bottomrule[0.15em]
        \end{tabular}
    \end{center}
\end{itemize}


\section{MIS of Export png Module} \label{Mod:ExportPNG}

\subsection{Module}

ExportPNG

\subsection{Uses}
\begin{itemize}
    \item \hyperref[Mod:Mask2D]{Data Extraction 2D Mask}
    \item \hyperref[Mod:Mask3D]{Data Extraction 3D Mask}
    \item \hyperref[Mod:Image]{Data 2D Image}
    \item \hyperref[Mod:Disp1D]{Display 2D Image}
    \item \hyperref[Mod:HH]{Hardware Hiding}
\end{itemize}

\subsection{Syntax}

\subsubsection{Exported Access Programs}

\begin{center}
    \begin{tabular}{p{2cm} p{4cm} p{4cm} p{4cm}}
        \hline
        \textbf{Name} & \textbf{In} & \textbf{Out} & \textbf{Exceptions} \\
        \hline
        WritePNG & Image & file & FILE EXISTS \\
        \hline
    \end{tabular}
\end{center}

\subsection{Semantics}
This module writes Image data to a png file.

\subsubsection{State Variables}
N/A

\subsubsection{Environment Variables}
\begin{itemize}
    \item filesystem
\end{itemize}

\subsubsection{Access Routine Semantics}

\noindent WritePNG():
\begin{itemize}
    \item input: \hyperref[Mod:Image]{Image}
    \item transition: N/A
    \item output: png file containing image data, written to filesystem
    \item exception:
    \begin{center}
        \begin{tabular}{p{3cm} p{12cm}}
            \toprule[0.15em]
            \textbf{Exception} & \textbf{Condition}\\
            \midrule[0.1em]
            \multirow{2}{0.25\textwidth}{FILE EXISTS} & The filename already
            exists in the filesystem\\ 
            & $fname \in filesystem \Rightarrow$ FILE\_EXISTS\\ 
            \bottomrule[0.15em]
        \end{tabular}
    \end{center}
\end{itemize}

\section{MIS of Export rpl Module} \label{Mod:ExportRPL}

\subsection{Module}

ExportRPL

\subsection{Uses}
\begin{itemize}
    \item \hyperref[Mod:Spectrum]{Data 1D Spectrum}
    \item \hyperref[Mod:Image]{Data 2D Image}
    \item \hyperref[Mod:SI]{Data 3D Spectrum Image}
    \item \hyperref[Mod:HH]{Hardware Hiding}
\end{itemize}

\subsection{Syntax}

\subsubsection{Exported Access Programs}

\begin{center}
    \begin{tabular}{p{2cm} p{4cm} p{4cm} p{4cm}}
        \hline
        \textbf{Name} & \textbf{In} & \textbf{Out} & \textbf{Exceptions} \\
        \hline
        WriteRPL & Spectrum $|$ Image $|$ Spectrum Image & file & FILE EXISTS \\
        \hline
    \end{tabular}
\end{center}

\subsection{Semantics}
This module writes Spectrum data, Image data, or Spectrum Image data to an rpl
file.

\subsubsection{State Variables}
N/A

\subsubsection{Environment Variables}
\begin{itemize}
    \item filesystem
\end{itemize}

\subsubsection{Access Routine Semantics}

\noindent WriteRPL():
\begin{itemize}
    \item input: \hyperref[Mod:Spectrum]{Spectrum} $|$
    \hyperref[Mod:Image]{Image} $|$ \hyperref[Mod:SI]{Spectrum Image}
    \item transition: N/A
    \item output: rpl file containing spectrum data, image data, or spectrum
    image data (including any metadata), written to filesystem
    \item exception:
    \begin{center}
        \begin{tabular}{p{3cm} p{12cm}}
            \toprule[0.15em]
            \textbf{Exception} & \textbf{Condition}\\
            \midrule[0.1em]
            \multirow{2}{0.25\textwidth}{FILE EXISTS} & The filename already
            exists in the filesystem\\ 
            & $fname \in filesystem \Rightarrow$ FILE\_EXISTS\\ 
            \bottomrule[0.15em]
        \end{tabular}
    \end{center}
\end{itemize}

\section{MIS of Data Processing Richardson-Lucy Deconvolution Module}
\label{Mod: RLDeconvolution}

\subsection{Module}

RLDeconvolution

\subsection{Uses}
\begin{itemize}
    \item \hyperref[Mod:Array]{Array Data Structure}
    \item \hyperref[Mod:HH]{Hardware Hiding}
    \item \hyperref[Mod:SI]{Data 3D Spectrum Image}
    \item \hyperref[Mod:Spectrum]{Data 1D Spectrum}
\end{itemize}

\subsection{Syntax}

\subsubsection{Exported Access Programs}

\begin{center}
    \begin{tabular}{p{4cm} p{4cm} p{4cm} p{2cm}}
        \hline
        \textbf{Name} & \textbf{In} & \textbf{Out} & \textbf{Exceptions} \\
        \hline
        RLDeconvolution & Spectrum, PSF, iterations & deconvolved Spectrum & DIVIDE BY
        ZERO \\
        SIDeconvolution & Spectrum Image, PSF, iterations, threads & deconvolved
        Spectrum Image & - \\
        \hline
    \end{tabular}
\end{center}

\subsection{Semantics}
This module performs the Richardson-Lucy deconvolution algorithm on either a
Spectrum or Spectrum Image, following \hyperref[deconvolution]{IM2} in the SRS.

\subsubsection{State Variables}
N/A

\subsubsection{Environment Variables}
\begin{itemize}
    \item threads: $\mathbb{N}$, number of processing threads to use during
    Spectrum Image deconvolution
\end{itemize}

\subsubsection{Access Routine Semantics}

\noindent RLDeconvolution():
\begin{itemize}
    \item input: 
    \begin{itemize}
        \item \hyperref[Mod:Spectrum]{Spectrum}, the spectrum to be
        deconvolved
        \item Point Spread Function: \hyperref[Mod:Spectrum]{Spectrum}, the
        point spread function to deconvolve from Spectrum
        \item iterations: $\mathbb{N}$, the number of iterations of the
        algorithm to perform
    \end{itemize}
    \item transition: N/A
    \item output: Deconvolved \hyperref[Mod:Spectrum]{Spectrum}
    \item exception:
    \begin{center}
        \begin{tabular}{p{3.5cm} p{12cm}}
            \toprule[0.15em]
            \textbf{Exception} & \textbf{Condition}\\
            \midrule[0.1em]
            \multirow{2}{0.25\textwidth}{DIVIDE BY ZERO} & The algorithm
            encounters a 0 in the denominator\\ 
            & $\sum_l{I_{PSF}(E)I_{real(l)}^c(E)} = 0 \Rightarrow$
            DIVIDE\_BY\_ZERO\\ 
            \bottomrule[0.15em]
        \end{tabular}
    \end{center}
\end{itemize}

\noindent SIDeconvolution():
\begin{itemize}
    \item input: 
    \begin{itemize}
        \item \hyperref[Mod:SI]{Spectrum Image}, the spectrum image to be
        deconvolved
        \item Point Spread Function: \hyperref[Mod:Spectrum]{Spectrum}, the
        point spread function to deconvolve from Spectrum Image
        \item iterations: $\mathbb{N}$, the number of iterations of the
        algorithm to perform
        \item threads: $\mathbb{N}$, the number of processing threads to use in
        deconvolving the Spectrum Image
    \end{itemize}
    \item transition: N/A
    \item output: Deconvolved Spectrum Image
    \item exception: N/A
\end{itemize}

\section{MIS of Data Processing Normalization Module} \label{Mod:Normalization}

\subsection{Module}
Normalization

\subsection{Uses}
\begin{itemize}
    \item \hyperref[Mod:Array]{Array Data Structure}
    \item \hyperref[Mod:SI]{Data 3D Spectrum Image}
    \item \hyperref[Mod:Spectrum]{Data 1D Spectrum}
    \item \hyperref[Mod:Slice1D]{Data Extraction 1D Slice}
\end{itemize}

\subsection{Syntax}

\subsubsection{Exported Access Programs}

\begin{center}
    \begin{tabular}{p{3cm} p{4cm} p{4cm} p{2cm}}
        \hline
        \textbf{Name} & \textbf{In} & \textbf{Out} & \textbf{Exceptions} \\
        \hline
        Normalization & Spectrum $|$ Spectrum Image, slice & Normalized Spectrum &
        DIVIDE BY ZERO \\
        \hline
    \end{tabular}
\end{center}

\subsection{Semantics}
This module normalizes either a Spectrum or Spectrum Image to the sum over the
range defined by the user, following \hyperref[normalization]{IM1} in the SRS.

\subsubsection{State Variables}
N/A

\subsubsection{Access Routine Semantics}

\noindent Normalization():
\begin{itemize}
    \item input: 
    \begin{itemize}
        \item \hyperref[Mod:Spectrum]{Spectrum}, the spectrum to be normalized,
        or \hyperref[Mod:SI]{Spectrum Image}, the spectrum image to be normalized
        \item \hyperref[Mod:Slice1D]{slice}
    \end{itemize}
    \item transition: N/A
    \item output: Normalized \hyperref[Mod:Spectrum]{Spectrum} or
    \hyperref[Mod:SI]{Spectrum Image}
    \item exception:
    \begin{center}
        \begin{tabular}{p{3.5cm} p{12cm}}
            \toprule[0.15em]
            \textbf{Exception} & \textbf{Condition}\\
            \midrule[0.1em]
            \multirow{2}{0.25\textwidth}{DIVIDE BY ZERO} & The algorithm
            encounters a 0 in the denominator\\ 
            & $\sum_{E(k=k_1)}^{E(k=k_2)}{I(E(k))} = 0 \Rightarrow$
            DIVIDE\_BY\_ZERO\\ 
            \bottomrule[0.15em]
        \end{tabular}
    \end{center}
\end{itemize}

\section{MIS of Data Processing Gain Correction Module} \label{Mod:GainCorr}

\subsection{Module}
GainCorr

\subsection{Uses}
\begin{itemize}
    \item \hyperref[Mod:Array]{Array Data Structure}
    \item \hyperref[Mod:SI]{Data 3D Spectrum Image}
    \item \hyperref[Mod:Spectrum]{Data 1D Spectrum}
\end{itemize}

\subsection{Syntax}

\subsubsection{Exported Access Programs}

\begin{center}
    \begin{tabular}{p{3cm} p{4cm} p{4cm} p{4cm}}
        \hline
        \textbf{Name} & \textbf{In} & \textbf{Out} & \textbf{Exceptions} \\
        \hline
        GainCorrection & data, Gain Reference & Gain Corrected Spectrum & DIVIDE BY
        ZERO, SIZE MISMATCH \\
        \hline
    \end{tabular}
\end{center}

\subsection{Semantics}
This module corrects either a Spectrum or Spectrum Image using a gain reference
(obtained from the acquisition camera hardware), following \hyperref[gain]{IM4}
in the SRS.

\subsubsection{State Variables}
N/A

\subsubsection{Access Routine Semantics}
\noindent GainCorrection():
\begin{itemize}
    \item input: 
    \begin{itemize}
        \item data: \hyperref[Mod:Spectrum]{Spectrum}, the spectrum to be
        corrected, or \hyperref[Mod:SI]{Spectrum Image}, the spectrum image to be
        corrected
        \item Gain Reference: \hyperref[Mod:Spectrum]{Spectrum}, a reference
        Spectrum obtained from the hardware
    \end{itemize}
    \item transition: N/A
    \item output: Gain-corrected \hyperref[Mod:Spectrum]{Spectrum} or
    \hyperref[Mod:SI]{Spectrum Image}
    \item exception:
    \begin{center}
        \begin{tabular}{p{3.5cm} p{12cm}}
            \toprule[0.15em]
            \textbf{Exception} & \textbf{Condition}\\
            \midrule[0.1em]
            \multirow{2}{0.25\textwidth}{DIVIDE BY ZERO} & The algorithm
            encounters a 0 in the denominator\\ 
            & $g(E) = 0 \Rightarrow$ DIVIDE\_BY\_ZERO\\ 
            \midrule[0.05em]
            \multirow{2}{0.25\textwidth}{SIZE MISMATCH} & The size of the gain
            correction is different from the size of the input data spectral range\\ 
            & $len(g(E)) \neq len(data.Srange) = 0 \Rightarrow$ SIZE\_MISMATCH\\
            
            \bottomrule[0.15em]
        \end{tabular}
    \end{center}
\end{itemize}


\section{MIS of Data Processing Background Correction Module}
\label{Mod:BackgroundCorr}

\subsection{Module}
BkgndCorr

\subsection{Uses}
\begin{itemize}
    \item \hyperref[Mod:Array]{Array Data Structure}
    \item \hyperref[Mod:SI]{Data 3D Spectrum Image}
    \item \hyperref[Mod:Spectrum]{Data 1D Spectrum}
\end{itemize}

\subsection{Syntax}

\subsubsection{Exported Access Programs}
\begin{center}
    \begin{tabular}{p{4cm} p{3.5cm} p{4cm} p{3cm}}
        \hline
        \textbf{Name} & \textbf{In} & \textbf{Out} & \textbf{Exceptions} \\
        \hline
        BackgroundCorrection & data, Background Reference & Background Corrected
        Spectrum & SIZE MISMATCH \\
        \hline
    \end{tabular}
\end{center}

\subsection{Semantics}
This module corrects the background noise for either a Spectrum or Spectrum
Image, following \hyperref[background]{IM3} in the SRS.

\subsubsection{State Variables}
N/A

\subsubsection{Access Routine Semantics}
\noindent BackgroundCorrection():
\begin{itemize}
    \item input: 
    \begin{itemize}
        \item data: \hyperref[Mod:Spectrum]{Spectrum}, the spectrum to be
        corrected, or \hyperref[Mod:SI]{Spectrum Image}, the spectrum image to be
        corrected
        \item Background Reference: \hyperref[Mod:Spectrum]{Spectrum}, a
        reference Spectrum representing the background noise in the camera
    \end{itemize}
    \item transition: N/A
    \item output: Background-corrected \hyperref[Mod:Spectrum]{Spectrum} or
    \hyperref[Mod:SI]{Spectrum Image}
    \item exception:
    \begin{center}
        \begin{tabular}{p{3.5cm} p{12cm}}
            \toprule[0.15em]
            \textbf{Exception} & \textbf{Condition}\\
            \midrule[0.1em]
            \multirow{2}{0.25\textwidth}{SIZE MISMATCH} & The size of the
            background correction is different from the size of the input data spectral
            range\\ 
            & $len(b(E)) \neq len(data.Srange) = 0 \Rightarrow$ SIZE\_MISMATCH\\
            
            \bottomrule[0.15em]
        \end{tabular}
    \end{center}
\end{itemize}


\section{MIS of Data Extraction 1D Slice Module} \label{Mod:Slice1D}

\subsection{Module}
Slice1D

\subsection{Uses}
\begin{itemize}
    \item \hyperref[Mod:Spectrum]{Data 1D Spectrum}
    \item \hyperref[Mod:Disp1D]{Display 1D Spectrum}
\end{itemize}

\subsection{Syntax}

\subsubsection{Exported Access Programs}

\begin{center}
    \begin{tabular}{p{3cm} p{4cm} p{4cm} p{2cm}}
        \hline
        \textbf{Name} & \textbf{In} & \textbf{Out} & \textbf{Exceptions} \\
        \hline
        CreateSlice & data, $[k_1, k_2]$ & slice & RANGE OUTSIDE BOUNDS \\
        IntegrateSlice & slice & integral & - \\
        \hline
    \end{tabular}
\end{center}

\subsection{Semantics}
This module allows the user to extract slices from a 1D dataset for further
analysis with other modules.

\subsubsection{State Variables}
\begin{itemize}
    \item slice: interval of \hyperref[Mod:Spectrum]{Spectrum} between $[k_1,
    k_2]$
\end{itemize}

\subsubsection{Access Routine Semantics}

\noindent CreateSlice():
\begin{itemize}
    \item input:
    \begin{itemize}
        \item $[k_1, k_2] \in \mathbb{R}^2 | [max(min(k_1, k_2),
        min(data.Srange)..min(max(k_1, k_2), max(data.Srange)))] \in data.Srange$
        \item data: \hyperref[Mod:Spectrum]{Spectrum}
    \end{itemize}
    \item transition: Creation of slice
    \item output: slice
    \item exception: 
    \begin{center}
        \begin{tabular}{p{3.5cm} p{12cm}}
            \toprule[0.15em]
            \textbf{Exception} & \textbf{Condition}\\
            \midrule[0.1em]
            \multirow{2}{0.25\textwidth}{RANGE OUTSIDE BOUNDS} & The user tried
            to select a range of values which was wholly outside the data's spectral range\\
            
            & $[max(min(k_1, k_2), min(data.Srange)..min(max(k_1, k_2),
            max(data.Srange)))] \notin data.Srange \Rightarrow$ RANGE\_OUTSIDE\_BOUNDS\\ 
            \bottomrule[0.15em]
        \end{tabular}
    \end{center}
\end{itemize}

\noindent IntegrateSlice():
\begin{itemize}
    \item input
    \begin{itemize}
        \item slice
    \end{itemize}
    \item transition: N/A
    \item output: integral over slice, $\sum_{k_1}^{k_2}data.data$
    \item exception: N/A
\end{itemize}

\section{MIS of Data Extraction 2D Mask Module} \label{Mod:Mask2D}

\subsection{Module}

Mask2D

\subsection{Uses}
\begin{itemize}
    \item \hyperref[Mod:Image]{Data 2D Image}
    \item \hyperref[Mod:Disp2D]{Display 2D Image}
\end{itemize}

\subsection{Syntax}

\subsubsection{Exported Access Programs}

\begin{center}
    \begin{tabular}{p{3cm} p{4cm} p{4cm} p{3cm}}
        \hline
        \textbf{Name} & \textbf{In} & \textbf{Out} & \textbf{Exceptions} \\
        \hline
        CreateMask & vertex list, size(data) & mask2D & RANGE OUTSIDE BOUNDS \\
        ApplyMask & data & masked data & SIZE MISMATCH\\
        ModifyMask & vertex list & mask2D & RANGE OUTSIDE BOUNDS \\
        \hline
    \end{tabular}
\end{center}

\subsection{Semantics}
This module allows the user to extract portions of a 2D dataset for further
analysis with other modules.

\subsubsection{State Variables}
\begin{itemize}
    \item mask2D, $\forall\ value \in mask, value \in \{True,False\}$, mask of
    boolean values representing the desired mask from the user
\end{itemize}

\subsubsection{Access Routine Semantics}

\noindent CreateMask():
\begin{itemize}
    \item input
    \begin{itemize}
        \item vertex list: $[x_i, y_i], x_i, y_i \in \mathbb{N}$, a list of
        $(x,y)$ ordered pairs of indices to access an array of size(data)
        \item size(data): $\mathbb{N}^2$, the size of the data to which the mask
        and vertex list refers
    \end{itemize}
    \item transition: Creates mask
    \item output: mask2D
    \item exception:
    \begin{center}
        \begin{tabular}{p{3.5cm} p{12cm}}
            \toprule[0.15em]
            \textbf{Exception} & \textbf{Condition}\\
            \midrule[0.1em]
            \multirow{2}{0.25\textwidth}{RANGE OUTSIDE BOUNDS} & The user tried
            to select a mask with a vertex outside the data boundaries\\ 
            & $[x_i, y_i] \notin [0..size(data_x), 0..size(data_y)] \Rightarrow$
            RANGE\_OUTSIDE\_BOUNDS\\ 
            \bottomrule[0.15em]
        \end{tabular}
    \end{center}
\end{itemize}

\noindent ApplyMask():
\begin{itemize}
    \item input
    \begin{itemize}
        \item data: \hyperref[Mod:Image]{Image}
    \end{itemize}
    \item transition:
    \item output: masked data: \hyperref[Mod:Image]{Image}
    \item exception:
    \begin{center}
        \begin{tabular}{p{3.5cm} p{12cm}}
            \toprule[0.15em]
            \textbf{Exception} & \textbf{Condition}\\
            \midrule[0.1em]
            \multirow{2}{0.25\textwidth}{SIZE MISMATCH} & The size of the data
            is not the same as the size of the mask\\ 
            & $size(mask) \neq size(data) \Rightarrow$ SIZE\_MISMATCH\\ 
            \bottomrule[0.15em]
        \end{tabular}
    \end{center}
\end{itemize}

\noindent ModifyMask():
\begin{itemize}
    \item input
    \begin{itemize}
        \item vertex list: $[x_i, y_i], x_i, y_i \in \mathbb{N}$, a list of
        $(x,y)$ ordered pairs of indices to access an array of size(data)
    \end{itemize}
    \item transition: Update mask to the new set of input vertices
    \item output: mask2D
    \item exception:
    \begin{center}
        \begin{tabular}{p{3.5cm} p{12cm}}
            \toprule[0.15em]
            \textbf{Exception} & \textbf{Condition}\\
            \midrule[0.1em]
            \multirow{2}{0.25\textwidth}{RANGE OUTSIDE BOUNDS} & The user tried
            to select a mask with a vertex outside the data boundaries\\ 
            & $[x_i, y_i] \notin [0..size(data_x), 0..size(data_y)] \Rightarrow$
            RANGE\_OUTSIDE\_BOUNDS\\ 
            \bottomrule[0.15em]
        \end{tabular}
    \end{center}
\end{itemize}

\section{MIS of Data Extraction 3D Mask Module} \label{Mod:Mask3D}

\subsection{Module}

Mask3D

\subsection{Uses}
\begin{itemize}
    \item \hyperref[Mod:SI]{Data 3D Spectrum Image}
    \item \hyperref[Mod:Disp3D]{Display 3D Spectrum Image}
\end{itemize}

\subsection{Syntax}

\subsubsection{Exported Access Programs}

\begin{center}
    \begin{tabular}{p{3cm} p{4cm} p{4cm} p{4cm}}
        \hline
        \textbf{Name} & \textbf{In} & \textbf{Out} & \textbf{Exceptions} \\
        \hline
        CreateMask3D & vertex list, size(data) & mask3d & RANGE OUTSIDE BOUNDS \\
        ExtrudeMask2D & mask2D, size(data) & mask3d & SIZE MISMATCH \\
        ExtrudeMask1D & slice, size(data) & mask3d & SIZE MISMATCH \\
        ApplyMask3D & data & masked data & SIZE MISMATCH \\
        ModifyMask3D & vertex list & mask3d & RANGE OUTSIDE BOUNDS \\
        \hline
    \end{tabular}
\end{center}

\subsection{Semantics}
This module allows the user to extract portions of a 3D dataset for further
analysis with other modules.

\subsubsection{State Variables}
\begin{itemize}
    \item mask3D, $\forall\ value \in mask, value \in \{True,False\},
    size(mask3D) = size(data)$, mask of boolean values representing the desired mask
    from the user
\end{itemize}

\subsubsection{Access Routine Semantics}

\noindent CreateMask3D():
\begin{itemize}
    \item input
    \begin{itemize}
        \item vertex list: $[x_i, y_i, k_i], x_i, y_i, k_i \in \mathbb{N}$, a
        list of $(x,y,k)$ ordered pairs of indices to access an array of size(data)
        \item size(data): $\mathbb{N}^3$, the size of the data to which the mask
        and vertex list refers
    \end{itemize}
    \item transition: Creates mask
    \item output: mask2D
    \item exception:
    \begin{center}
        \begin{tabular}{p{3.5cm} p{12cm}}
            \toprule[0.15em]
            \textbf{Exception} & \textbf{Condition}\\
            \midrule[0.1em]
            \multirow{2}{0.25\textwidth}{RANGE OUTSIDE BOUNDS} & The user tried
            to select a mask with a vertex outside the data boundaries\\ 
            & $[x_i, y_i, k_i] \notin [0..size(data_x), 0..size(data_y),
            0..size(data_k)] \Rightarrow$ RANGE\_OUTSIDE\_BOUNDS\\ 
            \bottomrule[0.15em]
        \end{tabular}
    \end{center}
\end{itemize}

\noindent ExtrudeMask2D():
\begin{itemize}
    \item input
    \begin{itemize}
        \item mask2D: \hyperref[Mod:Mask2D]{mask2D}
        \item size(data): $\mathbb{N}^3$, the size of the data to which the mask
        and vertex list refers
    \end{itemize}
    \item transition: Creates mask3D from mask2D by extruding it along the third
    dimension
    \item output: mask3D
    \item exception:
    \begin{center}
        \begin{tabular}{p{3.5cm} p{12cm}}
            \toprule[0.15em]
            \textbf{Exception} & \textbf{Condition}\\
            \midrule[0.1em]
            \multirow{2}{0.25\textwidth}{SIZE MISMATCH} & The size of the data
            is not the same as the size of the mask\\ 
            & $size(mask) \neq size(data) \Rightarrow$ SIZE\_MISMATCH\\ 
            \bottomrule[0.15em]
        \end{tabular}
    \end{center}
\end{itemize}

\noindent ExtrudeMask1D():
\begin{itemize}
    \item input
    \begin{itemize}
        \item slice: \hyperref[Mod:Slice1D]{slice}
        \item size(data): $\mathbb{N}^3$, the size of the data to which the mask
        and vertex list refers
    \end{itemize}
    \item transition: Creates mask3D from slice by extruding it along the two
    extra dimensions
    \item output: mask3D
    \item exception:
    \begin{center}
        \begin{tabular}{p{3.5cm} p{12cm}}
            \toprule[0.15em]
            \textbf{Exception} & \textbf{Condition}\\
            \midrule[0.1em]
            \multirow{2}{0.25\textwidth}{SIZE MISMATCH} & The size of the data
            is not the same as the size of the mask\\ 
            & $size(mask) \neq size(data) \Rightarrow$ SIZE\_MISMATCH\\ 
            \bottomrule[0.15em]
        \end{tabular}
    \end{center}
\end{itemize}

\noindent ApplyMask3D():
\begin{itemize}
    \item input
    \begin{itemize}
        \item data: \hyperref[Mod:SI]{Spectrum Image}
    \end{itemize}
    \item transition: 
    \item output: Masked data, \hyperref[Mod:SI]{Spectrum Image}
    \item exception:
    \begin{center}
        \begin{tabular}{p{3.5cm} p{12cm}}
            \toprule[0.15em]
            \textbf{Exception} & \textbf{Condition}\\
            \midrule[0.1em]
            \multirow{2}{0.25\textwidth}{SIZE MISMATCH} & The size of the data
            is not the same as the size of the mask\\ 
            & $size(mask) \neq size(data) \Rightarrow$ SIZE\_MISMATCH\\ 
            \bottomrule[0.15em]
        \end{tabular}
    \end{center}
\end{itemize}

\noindent ModifyMask3D():
\begin{itemize}
    \item input
    \begin{itemize}
        \item vertex list: $[x_i, y_i, k_i], x_i, y_i, k_i \in \mathbb{N}$, a
        list of $(x,y,k)$ ordered pairs of indices to access an array of size(data)
        \item size(data): $\mathbb{N}^3$, the size of the data to which the mask
        and vertex list refers
    \end{itemize}
    \item transition: Creates mask
    \item output: mask3D
    \item exception:
    \begin{center}
        \begin{tabular}{p{3.5cm} p{12cm}}
            \toprule[0.15em]
            \textbf{Exception} & \textbf{Condition}\\
            \midrule[0.1em]
            \multirow{2}{0.25\textwidth}{RANGE OUTSIDE BOUNDS} & The user tried
            to select a mask with a vertex outside the data boundaries\\ 
            & $[x_i, y_i, k_i] \notin [0..size(data_x), 0..size(data_y),
            0..size(data_k)] \Rightarrow$ RANGE\_OUTSIDE\_BOUNDS\\ 
            \bottomrule[0.15em]
        \end{tabular}
    \end{center}
\end{itemize}

\section{MIS of Display 1D Spectrum Module} \label{Mod:Disp1D}

\subsection{Module}

Disp1D

\subsection{Uses}
\begin{itemize}
    \item \hyperref[Mod:Spectrum]{Data 1D Spectrum}
    \item \hyperref[Mod:Plotting]{Plotting library}
\end{itemize}

\subsection{Syntax}

\subsubsection{Exported Access Programs}

\begin{center}
    \begin{tabular}{p{2cm} p{4cm} p{4cm} p{2cm}}
        \hline
        \textbf{Name} & \textbf{In} & \textbf{Out} & \textbf{Exceptions} \\
        \hline
        plot & data & axis1D, event handler & - \\
        \hline
    \end{tabular}
\end{center}

\subsection{Semantics}
This module plots 1D spectrum data and allows event handling (\textit{eg}, to
slice the spectrum).

\subsubsection{State Variables}
\begin{itemize}
    \item Spectrum axis: \hyperref[Mod:Plotting]{axis1D} containing the plotted
    data and \hyperref[Mod:Plotting]{event handler}
\end{itemize}

\subsubsection{Access Routine Semantics}

\noindent plot():
\begin{itemize}
    \item input:
    \begin{itemize}
        \item data: \hyperref[Mod:Spectrum]{Spectrum}
    \end{itemize}
    \item transition: create axis and plot data on axis, initialize event
    handler for axis
    \item output:
    \begin{itemize}
        \item axis1D: \hyperref[Mod:Plotting]{axis1D}
        \item event handler: \hyperref[Mod:Plotting]{axis1D}
    \end{itemize}
    \item exception: N/A
\end{itemize}

\section{MIS of Display 2D Image Module} \label{Mod:Disp2D}

\subsection{Module}

Disp2D

\subsection{Uses}
\begin{itemize}
    \item \hyperref[Mod:Spectrum]{Data 1D Spectrum}
    \item \hyperref[Mod:Plotting]{Plotting library}
\end{itemize}

\subsection{Syntax}

\subsubsection{Exported Access Programs}

\begin{center}
    \begin{tabular}{p{4cm} p{3cm} p{3cm} p{4cm}}
        \hline
        \textbf{Name} & \textbf{In} & \textbf{Out} & \textbf{Exceptions} \\
        \hline
        plot & data & axis2D, event handler & - \\
        AddScalebar & - & - & NO CALIBRATION \\
        ChangeContrast & [minC, maxC], data & - & OUT OF RANGE \\
        \hline
    \end{tabular}
\end{center}

\subsection{Semantics}
This module plots 2D image data and allows event handling (\textit{eg}, to
create masks on the image).

\subsubsection{State Variables}
\begin{itemize}
    \item Image axis: \hyperref[Mod:Plotting]{axis2D} containing the plotted
    data and \hyperref[Mod:Plotting]{event handler}
\end{itemize}

\subsubsection{Access Routine Semantics}

\noindent plot():
\begin{itemize}
    \item input:
    \begin{itemize}
        \item data: \hyperref[Mod:Image]{Image}
    \end{itemize}
    \item transition: create axis and plot data on axis, initialize event
    handler for axis
    \item output:
    \begin{itemize}
        \item axis2D: \hyperref[Mod:Plotting]{axis2D}
        \item event handler: \hyperref[Mod:Plotting]{axis2D}
    \end{itemize}
    \item exception: N/A
\end{itemize}

\noindent AddScalebar():
\begin{itemize}
    \item input: N/A
    \item transition: Add scalebar to Image axis
    \item output: N/A
    \item exception: 
    \begin{center}
        \begin{tabular}{p{3.5cm} p{12cm}}
            \toprule[0.15em]
            \textbf{Exception} & \textbf{Condition}\\
            \midrule[0.1em]
            \multirow{2}{0.25\textwidth}{NO CALIBRATION} & No calibration exists
            in the image, so a scalebar cannot be added\\ 
            & $\nexists\ data.Imcal \Rightarrow$ NO\_CALIBRATION\\ 
            \bottomrule[0.15em]
        \end{tabular}
    \end{center}
\end{itemize}

\noindent ChangeContrast():
\begin{itemize}
    \item input: 
    \begin{itemize}
        \item $[minC, maxC]$, $\in \mathbb{R}^2$, the minimum and maximum
        contrast to stretch the colourscale to 
        \item data: \hyperref[Mod:Image]{Image}
    \end{itemize}
    \item transition: Change the contrast of the displayed image
    \item output: N/A
    \item exception: 
    \begin{center}
        \begin{tabular}{p{3.5cm} p{12cm}}
            \toprule[0.15em]
            \textbf{Exception} & \textbf{Condition}\\
            \midrule[0.1em]
            \multirow{2}{0.25\textwidth}{OUT OF RANGE} & The user tried to
            select a contrast range of values which was wholly outside the data's intensity
            limits\\ 
            & $[max(minC, min(data.data)..maxC, max(data.data)))] \notin
            data.Srange \Rightarrow$ OUT\_OF\_RANGE\\ 
            \bottomrule[0.15em]
        \end{tabular}
    \end{center}
\end{itemize}

\section{MIS of Display 3D Spectrum Image Module} \label{Mod:Disp3D}

\subsection{Module}

Disp3D

\subsection{Uses}
\begin{itemize}
    \item Data
    \item Plotting library
    \item 2D image plot
    \item 1D spectrum plot
\end{itemize}

\subsection{Syntax}

\subsubsection{Exported Access Programs}

\begin{center}
    \begin{tabular}{p{2cm} p{3cm} p{5cm} p{2cm}}
        \hline
        \textbf{Name} & \textbf{In} & \textbf{Out} & \textbf{Exceptions} \\
        \hline
        plot & data & figure3d, event handler & - \\
        \hline
    \end{tabular}
\end{center}

\subsection{Semantics}
This module arranges the elements of the 1D and 2D display modules to suit a 3D
dataset, and allows connection of the different axes (through event handling)
such that within one figure, all axes refer to the same dataset.

\subsubsection{State Variables}
\begin{itemize}
    \item figure3d: \hyperref[Mod:Plotting]{figure} containing image plot,
    spectrum plot, mask plot, colourbar, and image contrast histogram
    \item image plot: \hyperref[Mod:Plotting]{axis2D} for plotting images extracted
    from \hyperref[Mod:Slice1D]{slicing} the spectrum axis
    \item spectrum plot: \hyperref[Mod:Plotting]{axis1D} for plotting spectra
    extracted from \hyperref[Mod:Mask2D]{masks} on the image axis
    \item mask plot: \hyperref[Mod:Plotting]{axis2D}
    \item colourbar axis: colourbar \hyperref[Mod:Plotting]{axis} for image plot
    \item image contrast axis: histogram \hyperref[Mod:Plotting]{axis} for image
    plot
\end{itemize}

\subsubsection{Access Routine Semantics}

\noindent plot():
\begin{itemize}
    \item input:
    \begin{itemize}
        \item data: \hyperref[Mod:Image]{Image}
    \end{itemize}
    \item transition: create axis and plot data on axis, initialize event
    handler for axis
    \item output:
    \begin{itemize}
        \item figure3D: contains \hyperref[Mod:Plotting]{axis1D},
        \hyperref[Mod:Plotting]{axis2D}
        \item event handler for \hyperref[Mod:Plotting]{axis1D}
        \item event handler for \hyperref[Mod:Plotting]{axis2D}
    \end{itemize}
    \item exception: N/A
\end{itemize}

\section{MIS of Data 1D Spectrum Module} \label{Mod:Spectrum}
\subsection{Template Module}
Spectrum

\subsection{Uses}
\begin{itemize}
    \item \hyperref[Mod:Array]{Array data structure}
\end{itemize}

\subsection{Syntax}

\subsubsection{Types}
Spectrum

\subsubsection{Exported Access Programs}

\begin{center}
    \begin{tabular}{p{1.5cm} p{4cm} p{4cm} p{4cm}}
        \toprule
        \textbf{Name} & \textbf{In} & \textbf{Out} & \textbf{Exceptions} \\
        \midrule
        init & data, (Srange $|$ (dispersion $\&$ [index, value])), Slabel, Sunit & - &
        WRONG DATA TYPE, LENGTH MISMATCH \\
        \bottomrule
    \end{tabular}
\end{center}

\subsection{Semantics}
This module contains the abstract data type Spectrum, including the following
state variables.
\subsubsection{State Variables}
\begin{itemize}
    \item \textit{SRange}: $\mathbb{R}^K$
    \item \textit{data}: $\mathbb{R}^K$
    \item \textit{index}: $\mathbb{Z}$
    \item \textit{value}: $\mathbb{R}$
    \item \textit{dispersion}: $\mathbb{R}$
    \item \textit{Slabel}: \textit{str}
    \item \textit{Sunit}: \textit{str}
    \item \textit{metadata}: \textit{dict}
\end{itemize}

\subsubsection{Access Routine Semantics}

\noindent init():
init initializes a Spectrum object.
\begin{itemize}
    \item input:
    \begin{itemize}
        \item \textit{data}: intensity values, $\in \mathbb{R}^K$
        \item \textit{Srange}: spectral axis values, $\in \mathbb{R}^K$
        \item \textit{dispersion}: difference  between neighbouring channels
        along the spectral axis,  $\mathbb{R}$
        \item \textit{index}: location on the spectral axis at which
        \textit{value} is, $\mathbb{Z}$
        \item \textit{value}: value of the spectral axis (in spectral axis
        units) at the location given by \textit{index}, $\mathbb{R}$
        \item \textit{Slabel}: spectrum label, the name for the spectral axis
        (\textit{e.g.} Energy, Wavelength), \textit{str}
        \item \textit{Sunit}: spectrum units, the units which the spectral axis
        uses (\textit{e.g.} \si{\electronvolt}, \si{\nano\metre}), \textit{str}
    \end{itemize}
    \item transition: Creates all state variables
    \item output: N/A
    \item exception:
    \begin{center}
        \begin{tabular}{p{3cm} p{12cm}}
            \toprule[0.15em]
            \textbf{Exception} & \textbf{Condition}\\
            \midrule[0.1em]
            WRONG DATA TYPE & Any of the input data are the wrong type\\ 
            \midrule[0.05em]
            LENGTH MISMATCH & The length of Srange is not the same as the length of
            data\\
            & $len(Srange) \neq len(data)$\\
            \bottomrule[0.15em]
        \end{tabular}
    \end{center}
\end{itemize}

\section{MIS of Data 2D Image Module} \label{Mod:Image}

\subsection{Template Module}
Image

\subsection{Uses}
\begin{itemize}
    \item \hyperref[Mod:Array]{Array data structure}
\end{itemize}

\subsection{Syntax}
\subsubsection{Types}
Image

\subsubsection{Exported Access Programs}

\begin{center}
    \begin{tabular}{p{1.5cm} p{4cm} p{4cm} p{4cm}}
        \toprule
        \textbf{Name} & \textbf{In} & \textbf{Out} & \textbf{Exceptions} \\
        \midrule
        init & data, Imcal, metadata & - & WRONG DATA TYPE, WRONG DIMENSIONS \\
        \bottomrule
    \end{tabular}
\end{center}

\subsection{Semantics}
This module contains the abstract data type Spectrum, including the following
state variables.
\subsubsection{State Variables}
\begin{itemize}
    \item data: $\mathbb{R}^{X \times Y}$
    \item Imcal: $\mathbb{R}$
    \item metadata: dict
\end{itemize}

\subsubsection{Access Routine Semantics}

\noindent init():
init initializes an Image object.
\begin{itemize}
    \item input:
    \begin{itemize}
        \item \textit{data}: intensity values, $\in \mathbb{R}^{X \times Y}$
        \item \textit{Imcal}: image calibration values (\textit{e.g.} number of
        \si{\nano\metre} per pixel), $\in \mathbb{R}$
        \item \textit{metadata}: dictionary containing extra information about
        the source of the image (\textit{e.g.} experimental parameters)
    \end{itemize}
    \item transition: Creates all state variables
    \item output: N/A
    \item exception:
    \begin{center}
        \begin{tabular}{p{4cm} p{11.5cm}}
            \toprule[0.15em]
            \textbf{Exception} & \textbf{Condition}\\
            \midrule[0.1em]
            WRONG DATA TYPE & The input data are not real numbers or the Imcal
            value is not a real float\\
            & $data \notin \mathbb{R}^{X \times Y}  | Imcal \notin \mathbb{R}
            \Rightarrow$ WRONG\_DATA\_TYPE \\ 
            \midrule[0.05em]
            WRONG DIMENSIONS & The input data is not 2D\\
            & $size(data) \notin \mathbb{N}^2 \Rightarrow $ WRONG\_DIMENSIONS\\
            \bottomrule[0.15em]
        \end{tabular}
    \end{center}
\end{itemize}

\section{MIS of Data 3D Spectrum Image Module} \label{Mod:SI}

\subsection{Template Module}
SI

\subsection{Uses}
\begin{itemize}
    \item Array Data Structure
\end{itemize}


\subsubsection{Types}
Spectrum Image

\subsubsection{Exported Access Programs}

\begin{center}
    \begin{tabular}{p{1.5cm} p{4cm} p{4cm} p{4cm}}
        \toprule
        \textbf{Name} & \textbf{In} & \textbf{Out} & \textbf{Exceptions} \\
        \midrule
        init & data, Srange $|$ dispersion $\&$ [index, value], Slabel, Sunit,
        Imcal, metadata & - & WRONG DATA TYPE, WRONG DIMENSIONS \\
        \bottomrule
    \end{tabular}
\end{center}

\subsection{Semantics}
This module holds spectrum image data (a 3D dataset) and associated calibrations
and other related information.

\subsubsection{State Variables}
\begin{itemize}
    \item \textit{data}: $\mathbb{R}^{X \times Y \times K}$
    \item \textit{Imcal}: $\mathbb{R}$
    \item \textit{dispersion}: $\mathbb{R}$
    \item \textit{Srange}: $\mathbb{R}^{K}$
    \item \textit{index}: $\mathbb{Z}$
    \item \textit{value}: $\mathbb{R}$
    \item \textit{Slabel}: string
    \item \textit{Sunit}: string
    \item \textit{metadata}: dict
\end{itemize}

\subsubsection{Access Routine Semantics}
init
\begin{itemize}
    \item input: 
    \begin{itemize}
        \item \textit{data}: intensity values, $\in \mathbb{R}^{X \times Y
            \times K}$
        \item \textit{Srange}: spectral axis values, $\in \mathbb{R}^K$
        \item \textit{dispersion}: difference  between neighbouring channels
        along the spectral axis,  $\mathbb{R}$
        \item \textit{index}: location on the spectral axis at which
        \textit{value} is, $\mathbb{Z}$
        \item \textit{value}: value of the spectral axis (in spectral axis
        units) at the location given by \textit{index}, $\mathbb{R}$
        \item \textit{Slabel}: spectrum label, the name for the spectral axis
        (\textit{e.g.} Energy, Wavelength), \textit{str}
        \item \textit{Sunit}: spectrum units, the units which the spectral axis
        uses (\textit{e.g.} \si{\electronvolt}, \si{\nano\metre}), \textit{str}
        \item \textit{Imcal}: image calibration values (\textit{e.g.} number of
        \si{\nano\metre} per pixel), $\in \mathbb{R}$
        \item \textit{metadata}: dictionary containing extra information about
        the source of the image (\textit{e.g.} experimental parameters)
    \end{itemize}
    \item transition: Initialize all state variables
    \item output: N/A
    \item exception: 
    \begin{center}
        \begin{tabular}{p{3cm} p{12cm}}
            \toprule[0.15em]
            \textbf{Exception} & \textbf{Condition}\\
            \midrule[0.1em]
            WRONG DATA TYPE & Any of the input data are the wrong type\\
            & $(data \notin \mathbb{R}^{X \times Y \times K}) | (Srange \notin
            \mathbb{R}^K) | (dispersion \notin \mathbb{R}) | (index \notin \mathbb{Z}) |
            (value \notin \mathbb{R}) | (Slabel \notin str) | (Sunit \notin str) | (Imcal
            \notin \mathbb{R}) \Rightarrow$ WRONG\_DATA\_TYPE\\ 
            \midrule[0.05em]
            LENGTH MISMATCH & The length of Srange is not the same as the length of
            data's spectral axis\\
            & $len(Srange) \neq size(data)[2] \Rightarrow$ LENGTH\_MISMATCH\\
            \midrule[0.05em]
            WRONG DATA TYPE & The input data are not real numbers or the Imcal value
            is not a real float\\
            & $data \notin \mathbb{R}^{X \times Y}  | Imcal \notin \mathbb{R}
            \Rightarrow$ WRONG\_DATA\_TYPE \\ 
            \midrule[0.05em]
            WRONG DIMENSIONS & The input data is not 2D\\
            & $size(data) \notin \mathbb{N}^2 \Rightarrow $ WRONG\_DIMENSIONS\\
            \bottomrule[0.15em]
        \end{tabular}
    \end{center}
\end{itemize}

\section{MIS of Array Data Structure Module} \label{Mod:Array}

\subsection{Template Module}

Array

\subsection{Uses}
N/A

\subsection{Syntax}

\subsubsection{Type}
\begin{itemize}
    \item Array
\end{itemize}

\subsubsection{Exported Access Programs}

\begin{center}
    \begin{tabular}{p{2cm} p{4cm} p{4cm} p{2cm}}
        \hline
        \textbf{Name} & \textbf{In} & \textbf{Out} & \textbf{Exceptions} \\
        \hline
        CreateArray & data & Array & - \\
        ModifyArray & Array & Array & - \\
        \hline
    \end{tabular}
\end{center}

\subsection{Semantics}
This module holds the array structure and functions for performing various
calculations on arrays.

\subsubsection{State Variables}
\begin{itemize}
    \item Array: $\mathbb{C}^N, dim(N) \in \mathbb{N}$
\end{itemize}

\subsubsection{Access Routine Semantics}

\noindent CreateArray():
\begin{itemize}
    \item input: data, $\mathbb{C}^N$
    \item transition: Create array variable
    \item output: Array, $\mathbb{C}^N$
    \item exception: N/A
\end{itemize}

\noindent ModifyArray():
\begin{itemize}
    \item input: Array, $\mathbb{C}^N$
    \item transition: Modify array by some operation, including but not limited
    to, addition, subtraction, multiplication, division, \textit{etc.}
    \item output: Array, $\mathbb{C}^N$
    \item exception: N/A
\end{itemize}

\section{MIS of Plotting Library Module} \label{Mod:Plotting}

\subsection{Module}

Plotting

\subsection{Uses}
\begin{itemize}
    \item \hyperref[Mod:HH]{Hardware Hiding Module}
\end{itemize}

\subsection{Syntax}

\subsubsection{Exported Access Programs}

\begin{center}
    \begin{tabular}{p{2cm} p{4cm} p{4cm} p{2cm}}
        \hline
        \textbf{Name} & \textbf{In} & \textbf{Out} & \textbf{Exceptions} \\
        \hline
        plot & data & window & - \\
        \hline
    \end{tabular}
\end{center}

\subsection{Semantics}
This module is the basis for plotting 1D and 2D data and handling events such as
mouse clicks and keyboard presses.

\subsubsection{State Variables}
\begin{itemize}
    \item figure
    \item axis1D
    \item axis2D
    \item event handler
\end{itemize}

\subsubsection{Environment Variables}
\begin{itemize}
    \item window: 2D on-screen display of plot figure
\end{itemize}

\subsubsection{Access Routine Semantics}

\noindent plot():
\begin{itemize}
    \item input: data, $\mathbb{R}^{K} | \mathbb{R}^{X \times Y}$
    \item transition: Creates a figure to display the input data, with a 1D plot
    axis for 1D data or a 2D plot axis for 2D data. Provides handling for events
    such as mouse clicks or keyboard key presses and options to format the display.
    \item output: window
    \item exception:
\end{itemize}

\newpage

\bibliography {MIS}

\end{document}
