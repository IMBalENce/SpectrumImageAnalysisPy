\documentclass[12pt, titlepage]{article}

\usepackage{fullpage}
\usepackage{tikz}
\usepackage[round]{natbib}
\usepackage{multirow}
\usepackage{booktabs}
\usepackage{tabularx}
\usepackage{graphicx}
\usepackage{float}
\usepackage{hyperref}
\hypersetup{
    colorlinks,
    citecolor=black,
    filecolor=black,
    linkcolor=red,
    urlcolor=blue
}

%% Comments

\usepackage{color}

\newif\ifcomments\commentstrue

\ifcomments
\newcommand{\authornote}[3]{\textcolor{#1}{[#3 ---#2]}}
\newcommand{\todo}[1]{\textcolor{red}{[TODO: #1]}}
\else
\newcommand{\authornote}[3]{}
\newcommand{\todo}[1]{}
\fi

\newcommand{\wss}[1]{\authornote{blue}{SS}{#1}}
\newcommand{\an}[1]{\authornote{magenta}{Author}{#1}}


\newcounter{acnum}
\newcommand{\actheacnum}{AC\theacnum}
\newcommand{\acref}[1]{AC\ref{#1}}

\newcounter{ucnum}
\newcommand{\uctheucnum}{UC\theucnum}
\newcommand{\uref}[1]{UC\ref{#1}}

\newcounter{mnum}
\newcommand{\mthemnum}{M\themnum}
\newcommand{\mref}[1]{M\ref{#1}}
\newcommand{\progname}{SpectrumImageAnalysisPy}

\begin{document}

\title{Module Guide: Project Title} 
\author{Author Name}
\date{\today}

\maketitle

\pagenumbering{roman}

\section{Revision History}

\begin{tabularx}{\textwidth}{p{3cm}p{2cm}X}
\toprule {\bf Date} & {\bf Version} & {\bf Notes}\\
\midrule
Date 1 & 1.0 & Notes\\
Date 2 & 1.1 & Notes\\
\bottomrule
\end{tabularx}

\newpage

\tableofcontents

\listoftables

\listoffigures

\newpage

\pagenumbering{arabic}

\section{Introduction}

Decomposing a system into modules is a commonly accepted approach to developing
software.  A module is a work assignment for a programmer or programming
team~\citep{ParnasEtAl1984}.  We advocate a decomposition
based on the principle of information hiding~\citep{Parnas1972a}.  This
principle supports design for change, because the ``secrets'' that each module
hides represent likely future changes.  Design for change is valuable in SC,
where modifications are frequent, especially during initial development as the
solution space is explored.  

Our design follows the rules layed out by \citet{ParnasEtAl1984}, as follows:
\begin{itemize}
\item System details that are likely to change independently should be the
  secrets of separate modules.
\item Each data structure is used in only one module.
\item Any other program that requires information stored in a module's data
  structures must obtain it by calling access programs belonging to that module.
\end{itemize}

After completing the first stage of the design, the Software Requirements
Specification (SRS), the Module Guide (MG) is developed~\citep{ParnasEtAl1984}. The MG
specifies the modular structure of the system and is intended to allow both
designers and maintainers to easily identify the parts of the software.  The
potential readers of this document are as follows:

\begin{itemize}
\item New project members: This document can be a guide for a new project member
  to easily understand the overall structure and quickly find the
  relevant modules they are searching for.
\item Maintainers: The hierarchical structure of the module guide improves the
  maintainers' understanding when they need to make changes to the system. It is
  important for a maintainer to update the relevant sections of the document
  after changes have been made.
\item Designers: Once the module guide has been written, it can be used to
  check for consistency, feasibility and flexibility. Designers can verify the
  system in various ways, such as consistency among modules, feasibility of the
  decomposition, and flexibility of the design.
\end{itemize}

The rest of the document is organized as follows. Section
\ref{SecChange} lists the anticipated and unlikely changes of the software
requirements. Section \ref{SecMH} summarizes the module decomposition that
was constructed according to the likely changes. Section \ref{SecConnection}
specifies the connections between the software requirements and the
modules. Section \ref{SecMD} gives a detailed description of the
modules. Section \ref{SecTM} includes two traceability matrices. One checks
the completeness of the design against the requirements provided in the SRS. The
other shows the relation between anticipated changes and the modules. Section
\ref{SecUse} describes the use relation between modules.

\section{Anticipated and Unlikely Changes} \label{SecChange}

This section lists possible changes to the system. According to the likeliness
of the change, the possible changes are classified into two
categories. Anticipated changes are listed in Section \ref{SecAchange}, and
unlikely changes are listed in Section \ref{SecUchange}.

\subsection{Anticipated Changes} \label{SecAchange}

Anticipated changes are the source of the information that is to be hidden
inside the modules. Ideally, changing one of the anticipated changes will only
require changing the one module that hides the associated decision. The approach
adapted here is called design for
change.

\begin{description}
\item[\refstepcounter{acnum} \actheacnum \label{acHardware}:] The specific
  hardware on which the software is running.
\item[\refstepcounter{acnum} \actheacnum \label{acInput}:] The format of the
  initial input data.
\item [\refstepcounter{acnum} \actheacnum \label{acInput}:] The format of the output data.
\item [\refstepcounter{acnum} \actheacnum \label{acInput}:] The algorithms used for processing data.
\item [\refstepcounter{acnum} \actheacnum \label{acInput}:] The type of input data.
\item [\refstepcounter{acnum} \actheacnum \label{acInput}:] The plotting library used.

\end{description}

\subsection{Unlikely Changes} \label{SecUchange}

The module design should be as general as possible. However, a general system is
more complex. Sometimes this complexity is not necessary. Fixing some design
decisions at the system architecture stage can simplify the software design. If
these decision should later need to be changed, then many parts of the design
will potentially need to be modified. Hence, it is not intended that these
decisions will be changed.

\begin{description}
\item[\refstepcounter{ucnum} \uctheucnum \label{ucIO}:] Input/Output devices
  (Input: File and/or Keyboard, Output: File, Memory, and/or Screen).
\item[\refstepcounter{ucnum} \uctheucnum \label{ucInput}:] There will always be
  a source of input data external to the software.
\item ...
\end{description}

\section{Module Hierarchy} \label{SecMH}

This section provides an overview of the module design. Modules are summarized
in a hierarchy decomposed by secrets in Table \ref{TblMH}. The modules listed
below, which are leaves in the hierarchy tree, are the modules that will
actually be implemented.

\begin{description}
	\item [\refstepcounter{mnum} \mthemnum \label{mHH}:] Hardware-Hiding Module
	\item [\refstepcounter{mnum} \mthemnum \label{mcsvIm}:] Import .csv Module
	\item [\refstepcounter{mnum} \mthemnum \label{mdm3Im}:] Import .dm3 Module
	\item [\refstepcounter{mnum} \mthemnum \label{mh5Im}:] Import .h5 Module
	\item [\refstepcounter{mnum} \mthemnum \label{mrplIm}:] Import .rpl Module
	\item [\refstepcounter{mnum} \mthemnum \label{mcsvEx}:] Export .csv Module
	\item [\refstepcounter{mnum} \mthemnum \label{mh5Ex}:] Export .h5 Module
	\item [\refstepcounter{mnum} \mthemnum \label{mpngEx}:] Export .png Module
	\item [\refstepcounter{mnum} \mthemnum \label{mrplEx}:] Export .rpl Module
	\item [\refstepcounter{mnum} \mthemnum \label{mRL}:] Data Processing Richardson-Lucy Deconvolution Module
	\item [\refstepcounter{mnum} \mthemnum \label{mNorm}:] Data Processing Normalization Module
	\item [\refstepcounter{mnum} \mthemnum \label{mGain}:] Data Processing Gain Correction Module
	\item [\refstepcounter{mnum} \mthemnum \label{mBkgnd}:] Data Processing Background Correction Module
	\item [\refstepcounter{mnum} \mthemnum \label{m1Dslice}:] Data Extraction 1D Slice Module
	\item [\refstepcounter{mnum} \mthemnum \label{m2Dmask}:] Data Extraction 2D Mask Module
	\item [\refstepcounter{mnum} \mthemnum \label{m3Dmask}:] Data Extraction 3D Mask Module
	\item [\refstepcounter{mnum} \mthemnum \label{m1Dspecdisplay}:] Display 1D Spectrum Module
	\item [\refstepcounter{mnum} \mthemnum \label{m2Dimgdisplay}:] Display 2D Image Module
	\item [\refstepcounter{mnum} \mthemnum \label{m3DSIdisplay}:] Display 3D Spectrum Image Module
	\item [\refstepcounter{mnum} \mthemnum \label{m1Dspecdata}:] Data 1D Spectrum Module
	\item [\refstepcounter{mnum} \mthemnum \label{m2Dimgdata}:] Data 2D Image Module
	\item [\refstepcounter{mnum} \mthemnum \label{m3DSIdata}:] Data 3D Spectrum Image Module
	\item [\refstepcounter{mnum} \mthemnum \label{mArrayStruct}:] Array Data Structure
	\item [\refstepcounter{mnum} \mthemnum \label{mPlotStruct}:] Plotting Structure
\end{description}

\begin{table}[h!]
\centering
\begin{tabular}{p{0.25\textwidth} p{0.25\textwidth} p{0.4\textwidth}}
\toprule
\textbf{Level 1} & \textbf{Level 2} & \textbf{Level 3}\\
\midrule

{Hardware-Hiding Module} & ~ & ~ \\
\midrule

\multirow{18}{0.25\textwidth}{Behaviour-Hiding Module} & \multirow{4}{0.25\textwidth}{Import} & csv\\
& & dm3\\
& & h5\\
& & rpl\\\cline{2-3}
& \multirow{4}{0.25\textwidth}{Export} & csv\\
& & h5\\
& & png\\
& & rpl\\\cline{2-3}
& \multirow{4}{0.25\textwidth}{Data processing} & Richardson-Lucy Deconvolution\\
& & Normalization\\
& & Gain correction\\
& & Background correction\\\cline{2-3}
& \multirow{3}{0.25\textwidth}{Data extraction} & 1D slice\\
& & 2D mask\\
& & 3D mask\\\cline{2-3}
& \multirow{3}{0.25\textwidth}{Display} & 1D spectrum plot\\
& & 2D image plot\\
& & 3D spectrum image plot\\
\midrule

\multirow{2}{0.25\textwidth}{Software Decision Module} & \multirow{3}{0.25\textwidth}{Data} & Spectrum\\
& & Image\\
& & Spectrum Image\\\cline{2-3}
& Array data structure\\\cline{2-3}
& Plotting structure\\

\bottomrule

\end{tabular}
\caption{Module Hierarchy}
\label{TblMH}
\end{table}

\section{Connection Between Requirements and Design} \label{SecConnection}

The design of the system is intended to satisfy the requirements developed in
the SRS. In this stage, the system is decomposed into modules. The connection
between requirements and modules is listed in Table \ref{TblRT}.

\section{Module Decomposition} \label{SecMD}

Modules are decomposed according to the principle of ``information hiding''
proposed by \citet{ParnasEtAl1984}. The \emph{Secrets} field in a module
decomposition is a brief statement of the design decision hidden by the
module. The \emph{Services} field specifies \emph{what} the module will do
without documenting \emph{how} to do it. For each module, a suggestion for the
implementing software is given under the \emph{Implemented By} title. If the
entry is \emph{OS}, this means that the module is provided by the operating
system or by standard programming language libraries.  Also indicate if the
module will be implemented specifically for the software.

Only the leaf modules in the
hierarchy have to be implemented. If a dash (\emph{--}) is shown, this means
that the module is not a leaf and will not have to be implemented. Whether or
not this module is implemented depends on the programming language
selected.

\subsection{Hardware Hiding Modules (\mref{mHH})}

\begin{description}
\item[Secrets:]The data structure and algorithm used to implement the virtual
  hardware.
\item[Services:]Serves as a virtual hardware used by the rest of the
  system. This module provides the interface between the hardware and the
  software. So, the system can use it to display outputs or to accept inputs.
\item[Implemented By:] OS
\end{description}

\subsection{Behaviour-Hiding Module}

\begin{description}
\item[Secrets:]The contents of the required behaviours.
\item[Services:]Includes programs that provide externally visible behaviour of
  the system as specified in the software requirements specification (SRS)
  documents. This module serves as a communication layer between the
  hardware-hiding module and the software decision module. The programs in this
  module will need to change if there are changes in the SRS.
\item[Implemented By:] --
\end{description}

\subsubsection{Import}
\begin{description}
	\item[Secrets:]Container for modules concerning reading data from files.
	\item[Services:]Imports spectrum, image, or spectrum image data from various formats, as defined in the Level 3 modules to follow (\mref{mcsvIm}, \mref{mdm3Im}, \mref{mh5Im}, \mref{mrplIm}).
	\item[Implemented By:] --
\end{description}

\subsubsection{Import csv Module (\mref{mcsvIm})}
\begin{description}
	\item[Secrets:]Import from csv (\textit{e.g.}, as created by \progname).
	\item[Services:]Reads data from a csv file (which must be formatted appropriately) and assigns the data to the appropriate data module.
	\item[Implemented By:] \progname
\end{description}

\subsubsection{Import dm3 Module (\mref{mdm3Im})}
\begin{description}
	\item[Secrets:]Import data from Gatan Digital Micrograph file (dm3).
	\item[Services:]Reads data from a dm3 file and assigns the data to the appropriate data module. Will assign calibrations based on the metadata in the file.
	\item[Implemented By:] \progname
\end{description}

\subsubsection{Import h5 Module (\mref{mh5Im})}
\begin{description}
	\item[Secrets:]Import data from an h5 file (\textit{e.g.}, as produced by Odemis CL acquisition software, or by \progname).
	\item[Services:]Reads data from an h5 file and assigns the data to the appropriate data module. Will assign calibrations based on the metadata in the file.
	\item[Implemented By:] \progname
\end{description}

\subsubsection{Import dm3 Module (\mref{mrplIm})}
\begin{description}
	\item[Secrets:]Import data from rpl file (\textit{e.g.}, as exported by \progname)
	\item[Services:]Reads data from a rpl file and assigns the data to the appropriate data module. Will assign calibrations based on the metadata in the file.
	\item[Implemented By:] \progname
\end{description}

\subsubsection{Export}
\begin{description}
	\item[Secrets:]Container for modules concerning exporting data to files.
	\item[Services:]Writing data (spectrum, image, or spectrum image) to various formats, as defined by Level 3 modules (\mref{mcsvEx}, \mref{mh5Ex}, \mref{mpngEx}, \mref{mrplEx}).
	\item[Implemented By:] --
\end{description}

\subsubsection{Export csv Module (\mref{mcsvEx})}
\begin{description}
	\item[Secrets:]Export data to csv file.
	\item[Services:]Takes spectrum data, formats it as appropriate, and exports the spectrum range (x-axis) and the intensity (y-axis) as a comma separated value file.
	\item[Implemented By:] \progname
\end{description}

\subsubsection{Export h5 Module (\mref{mh5Ex})}
\begin{description}
	\item[Secrets:]Export data to h5 file.
	\item[Services:]Takes spectrum or spectrum image data, formats it as appropriate, and exports the spectrum range, calibrations and metadata, and the spectrum image data to an h5 file.
	\item[Implemented By:] \progname
\end{description}

\subsubsection{Export png Module (\mref{mpngEx})}
\begin{description}
	\item[Secrets:]Export image data to png file.
	\item[Services:]Takes image data, formats it as appropriate, and writes it to a png file, with scalebar if requested by the user.
	\item[Implemented By:] \progname
\end{description}

\subsubsection{Export rpl Module (\mref{mrplEx})}
\begin{description}
	\item[Secrets:]Export spectrum image data to rpl file.
	\item[Services:]Takes spectrum image data, formats it as appropriate, and writes it to a rpl file, including any metadata or calibrations which are known.
	\item[Implemented By:] \progname
\end{description}

\subsubsection{Data Processing}
\begin{description}
	\item[Secrets:]Container for data processing modules.
	\item[Services:]Performs data processing functions, as defined by Level 3 modules.
	\item[Implemented By:] --
\end{description}

\subsubsection{Data Processing: Richardson-Lucy Deconvolution (\mref{mRL})}
\begin{description}
	\item[Secrets:]Richardson-Lucy deconvolution algorithm.
	\item[Services:]Performs Richardson-Lucy deconvolution on the desired spectrum or spectrum image input, using the number of iterations desired by the user and returns a deconvolved spectrum or spectrum image. 
	\item[Implemented By:] \progname
\end{description}

\subsubsection{Data Processing: Normalization (\mref{mNorm})}
\begin{description}
	\item[Secrets:]Normalization algorithm.
	\item[Services:]Performs normalization of the spectrum or spectrum image, as desired by the user. User can input the channel or a range of channels on which to normalize.
	\item[Implemented By:] \progname
\end{description}

\subsubsection{Data Processing: Gain Correction (\mref{mGain})}
\begin{description}
	\item[Secrets:]Gain correction algorithm.
	\item[Services:] Performs gain correction on the spectrum or spectrum image input. A gain reference is also required.
	\item[Implemented By:] \progname
\end{description}

\subsubsection{Data Processing: Background Correction (\mref{mBkgnd})}
\begin{description}
	\item[Secrets:]Background correction algorithm.
	\item[Services:]Performs background correction on input spectrum or spectrum image and returns the corrected spectrum or spectrum image. A background reference is required.
	\item[Implemented By:] \progname
\end{description}

\subsection{Software Decision Module}

\begin{description}
\item[Secrets:] The design decision based on mathematical theorems, physical
  facts, or programming considerations. The secrets of this module are
  \emph{not} described in the SRS.
\item[Services:] Includes data structure and algorithms used in the system that
  do not provide direct interaction with the user. 
  % Changes in these modules are more likely to be motivated by a desire to
  % improve performance than by externally imposed changes.
\item[Implemented By:] --
\end{description}

\subsubsection{Etc.}

\section{Traceability Matrix} \label{SecTM}

This section shows two traceability matrices: between the modules and the
requirements and between the modules and the anticipated changes.

% the table should use mref, the requirements should be named, use something
% like fref
\begin{table}[H]
\centering
\begin{tabular}{p{0.2\textwidth} p{0.6\textwidth}}
\toprule
\textbf{Req.} & \textbf{Modules}\\
\midrule
R1 & \mref{mHH}, \mref{mInput}, \mref{mParams}, \mref{mControl}\\
R2 & \mref{mInput}, \mref{mParams}\\
R3 & \mref{mVerify}\\
R4 & \mref{mOutput}, \mref{mControl}\\
R5 & \mref{mOutput}, \mref{mODEs}, \mref{mControl}, \mref{mSeqDS}, \mref{mSolver}, \mref{mPlot}\\
R6 & \mref{mOutput}, \mref{mODEs}, \mref{mControl}, \mref{mSeqDS}, \mref{mSolver}, \mref{mPlot}\\
R7 & \mref{mOutput}, \mref{mEnergy}, \mref{mControl}, \mref{mSeqDS}, \mref{mPlot}\\
R8 & \mref{mOutput}, \mref{mEnergy}, \mref{mControl}, \mref{mSeqDS}, \mref{mPlot}\\
R9 & \mref{mVerifyOut}\\
R10 & \mref{mOutput}, \mref{mODEs}, \mref{mControl}\\
R11 & \mref{mOutput}, \mref{mODEs}, \mref{mEnergy}, \mref{mControl}\\
\bottomrule
\end{tabular}
\caption{Trace Between Requirements and Modules}
\label{TblRT}
\end{table}

\begin{table}[H]
\centering
\begin{tabular}{p{0.2\textwidth} p{0.6\textwidth}}
\toprule
\textbf{AC} & \textbf{Modules}\\
\midrule
\acref{acHardware} & \mref{mHH}\\
\acref{acInput} & \mref{mInput}\\
\acref{acParams} & \mref{mParams}\\
\acref{acVerify} & \mref{mVerify}\\
\acref{acOutput} & \mref{mOutput}\\
\acref{acVerifyOut} & \mref{mVerifyOut}\\
\acref{acODEs} & \mref{mODEs}\\
\acref{acEnergy} & \mref{mEnergy}\\
\acref{acControl} & \mref{mControl}\\
\acref{acSeqDS} & \mref{mSeqDS}\\
\acref{acSolver} & \mref{mSolver}\\
\acref{acPlot} & \mref{mPlot}\\
\bottomrule
\end{tabular}
\caption{Trace Between Anticipated Changes and Modules}
\label{TblACT}
\end{table}

\section{Use Hierarchy Between Modules} \label{SecUse}

In this section, the uses hierarchy between modules is
provided. \citet{Parnas1978} said of two programs A and B that A {\em uses} B if
correct execution of B may be necessary for A to complete the task described in
its specification. That is, A {\em uses} B if there exist situations in which
the correct functioning of A depends upon the availability of a correct
implementation of B.  Figure \ref{FigUH} illustrates the use relation between
the modules. It can be seen that the graph is a directed acyclic graph
(DAG). Each level of the hierarchy offers a testable and usable subset of the
system, and modules in the higher level of the hierarchy are essentially simpler
because they use modules from the lower levels.

\begin{figure}[h!]
	\centering
	% Graphic for TeX using PGF
% Title: /home/isobel/Documents/McMaster/PythonCodes/DataAnalysis/Doc/Design/MG/UseHierarchy.dia
% Creator: Dia v0.97.2
% CreationDate: Sun Dec 17 10:53:43 2017
% For: isobel
% \usepackage{tikz}
% The following commands are not supported in PSTricks at present
% We define them conditionally, so when they are implemented,
% this pgf file will use them.
\ifx\du\undefined
  \newlength{\du}
\fi
\setlength{\du}{15\unitlength}
\begin{tikzpicture}
\pgftransformxscale{1.600000}
\pgftransformyscale{-1.750000}
\definecolor{dialinecolor}{rgb}{0.000000, 0.000000, 0.000000}
\pgfsetstrokecolor{dialinecolor}
\definecolor{dialinecolor}{rgb}{1.000000, 1.000000, 1.000000}
\pgfsetfillcolor{dialinecolor}
\definecolor{dialinecolor}{rgb}{1.000000, 1.000000, 1.000000}
\pgfsetfillcolor{dialinecolor}
\fill (43.560600\du,6.500390\du)--(43.560600\du,8.417057\du)--(46.135600\du,8.417057\du)--(46.135600\du,6.500390\du)--cycle;
\pgfsetlinewidth{0.070000\du}
\pgfsetdash{}{0pt}
\pgfsetdash{}{0pt}
\pgfsetmiterjoin
\definecolor{dialinecolor}{rgb}{0.000000, 0.000000, 0.000000}
\pgfsetstrokecolor{dialinecolor}
\draw (43.560600\du,6.500390\du)--(43.560600\du,8.417057\du)--(46.135600\du,8.417057\du)--(46.135600\du,6.500390\du)--cycle;
% setfont left to latex
\definecolor{dialinecolor}{rgb}{0.000000, 0.000000, 0.000000}
\pgfsetstrokecolor{dialinecolor}
\node at (44.848100\du,7.350390\du){Import};
% setfont left to latex
\definecolor{dialinecolor}{rgb}{0.000000, 0.000000, 0.000000}
\pgfsetstrokecolor{dialinecolor}
\node at (44.848100\du,7.773723\du){(M2-M5)};
\definecolor{dialinecolor}{rgb}{1.000000, 1.000000, 1.000000}
\pgfsetfillcolor{dialinecolor}
\fill (37.890262\du,7.535436\du)--(37.890262\du,9.452102\du)--(40.895262\du,9.452102\du)--(40.895262\du,7.535436\du)--cycle;
\pgfsetlinewidth{0.070000\du}
\pgfsetdash{}{0pt}
\pgfsetdash{}{0pt}
\pgfsetmiterjoin
\definecolor{dialinecolor}{rgb}{0.000000, 0.000000, 0.000000}
\pgfsetstrokecolor{dialinecolor}
\draw (37.890262\du,7.535436\du)--(37.890262\du,9.452102\du)--(40.895262\du,9.452102\du)--(40.895262\du,7.535436\du)--cycle;
% setfont left to latex
\definecolor{dialinecolor}{rgb}{0.000000, 0.000000, 0.000000}
\pgfsetstrokecolor{dialinecolor}
\node at (39.392762\du,8.385436\du){Data};
% setfont left to latex
\definecolor{dialinecolor}{rgb}{0.000000, 0.000000, 0.000000}
\pgfsetstrokecolor{dialinecolor}
\node at (39.392762\du,8.808769\du){(M20-M22)};
\definecolor{dialinecolor}{rgb}{1.000000, 1.000000, 1.000000}
\pgfsetfillcolor{dialinecolor}
\fill (33.246823\du,0.099712\du)--(33.246823\du,2.439711\du)--(37.104323\du,2.439711\du)--(37.104323\du,0.099712\du)--cycle;
\pgfsetlinewidth{0.070000\du}
\pgfsetdash{}{0pt}
\pgfsetdash{}{0pt}
\pgfsetmiterjoin
\definecolor{dialinecolor}{rgb}{0.000000, 0.000000, 0.000000}
\pgfsetstrokecolor{dialinecolor}
\draw (33.246823\du,0.099712\du)--(33.246823\du,2.439711\du)--(37.104323\du,2.439711\du)--(37.104323\du,0.099712\du)--cycle;
% setfont left to latex
\definecolor{dialinecolor}{rgb}{0.000000, 0.000000, 0.000000}
\pgfsetstrokecolor{dialinecolor}
\node at (35.175573\du,1.161378\du){Data processing};
% setfont left to latex
\definecolor{dialinecolor}{rgb}{0.000000, 0.000000, 0.000000}
\pgfsetstrokecolor{dialinecolor}
\node at (35.175573\du,1.584711\du){(M10-M13)};
\definecolor{dialinecolor}{rgb}{1.000000, 1.000000, 1.000000}
\pgfsetfillcolor{dialinecolor}
\fill (39.330700\du,1.743810\du)--(39.330700\du,3.660477\du)--(43.065700\du,3.660477\du)--(43.065700\du,1.743810\du)--cycle;
\pgfsetlinewidth{0.070000\du}
\pgfsetdash{}{0pt}
\pgfsetdash{}{0pt}
\pgfsetmiterjoin
\definecolor{dialinecolor}{rgb}{0.000000, 0.000000, 0.000000}
\pgfsetstrokecolor{dialinecolor}
\draw (39.330700\du,1.743810\du)--(39.330700\du,3.660477\du)--(43.065700\du,3.660477\du)--(43.065700\du,1.743810\du)--cycle;
% setfont left to latex
\definecolor{dialinecolor}{rgb}{0.000000, 0.000000, 0.000000}
\pgfsetstrokecolor{dialinecolor}
\node at (41.198200\du,2.593810\du){Data extraction};
% setfont left to latex
\definecolor{dialinecolor}{rgb}{0.000000, 0.000000, 0.000000}
\pgfsetstrokecolor{dialinecolor}
\node at (41.198200\du,3.017143\du){(M14-M16)};
% setfont left to latex
\definecolor{dialinecolor}{rgb}{0.000000, 0.000000, 0.000000}
\pgfsetstrokecolor{dialinecolor}
\node[anchor=west] at (41.198200\du,2.702140\du){};
\definecolor{dialinecolor}{rgb}{1.000000, 1.000000, 1.000000}
\pgfsetfillcolor{dialinecolor}
\fill (45.636599\du,4.428750\du)--(45.636599\du,6.345417\du)--(48.641599\du,6.345417\du)--(48.641599\du,4.428750\du)--cycle;
\pgfsetlinewidth{0.070000\du}
\pgfsetdash{}{0pt}
\pgfsetdash{}{0pt}
\pgfsetmiterjoin
\definecolor{dialinecolor}{rgb}{0.000000, 0.000000, 0.000000}
\pgfsetstrokecolor{dialinecolor}
\draw (45.636599\du,4.428750\du)--(45.636599\du,6.345417\du)--(48.641599\du,6.345417\du)--(48.641599\du,4.428750\du)--cycle;
% setfont left to latex
\definecolor{dialinecolor}{rgb}{0.000000, 0.000000, 0.000000}
\pgfsetstrokecolor{dialinecolor}
\node at (47.139099\du,5.278750\du){Display};
% setfont left to latex
\definecolor{dialinecolor}{rgb}{0.000000, 0.000000, 0.000000}
\pgfsetstrokecolor{dialinecolor}
\node at (47.139099\du,5.702083\du){(M17-M19)};
\definecolor{dialinecolor}{rgb}{1.000000, 1.000000, 1.000000}
\pgfsetfillcolor{dialinecolor}
\fill (45.913900\du,0.995224\du)--(45.913900\du,2.911891\du)--(48.488900\du,2.911891\du)--(48.488900\du,0.995224\du)--cycle;
\pgfsetlinewidth{0.070000\du}
\pgfsetdash{}{0pt}
\pgfsetdash{}{0pt}
\pgfsetmiterjoin
\definecolor{dialinecolor}{rgb}{0.000000, 0.000000, 0.000000}
\pgfsetstrokecolor{dialinecolor}
\draw (45.913900\du,0.995224\du)--(45.913900\du,2.911891\du)--(48.488900\du,2.911891\du)--(48.488900\du,0.995224\du)--cycle;
% setfont left to latex
\definecolor{dialinecolor}{rgb}{0.000000, 0.000000, 0.000000}
\pgfsetstrokecolor{dialinecolor}
\node at (47.201400\du,1.845224\du){Export};
% setfont left to latex
\definecolor{dialinecolor}{rgb}{0.000000, 0.000000, 0.000000}
\pgfsetstrokecolor{dialinecolor}
\node at (47.201400\du,2.268557\du){(M6-M9)};
\definecolor{dialinecolor}{rgb}{1.000000, 1.000000, 1.000000}
\pgfsetfillcolor{dialinecolor}
\fill (37.537800\du,14.829000\du)--(37.537800\du,16.745667\du)--(42.475300\du,16.745667\du)--(42.475300\du,14.829000\du)--cycle;
\pgfsetlinewidth{0.070000\du}
\pgfsetdash{}{0pt}
\pgfsetdash{}{0pt}
\pgfsetmiterjoin
\definecolor{dialinecolor}{rgb}{0.000000, 0.000000, 0.000000}
\pgfsetstrokecolor{dialinecolor}
\draw (37.537800\du,14.829000\du)--(37.537800\du,16.745667\du)--(42.475300\du,16.745667\du)--(42.475300\du,14.829000\du)--cycle;
% setfont left to latex
\definecolor{dialinecolor}{rgb}{0.000000, 0.000000, 0.000000}
\pgfsetstrokecolor{dialinecolor}
\node at (40.006550\du,15.679000\du){Hardware-hiding };
% setfont left to latex
\definecolor{dialinecolor}{rgb}{0.000000, 0.000000, 0.000000}
\pgfsetstrokecolor{dialinecolor}
\node at (40.006550\du,16.102333\du){(M1)};
% setfont left to latex
\definecolor{dialinecolor}{rgb}{0.000000, 0.000000, 0.000000}
\pgfsetstrokecolor{dialinecolor}
\node[anchor=west] at (47.139099\du,5.387083\du){};
\definecolor{dialinecolor}{rgb}{1.000000, 1.000000, 1.000000}
\pgfsetfillcolor{dialinecolor}
\fill (33.926400\du,9.995180\du)--(33.926400\du,12.335180\du)--(37.536400\du,12.335180\du)--(37.536400\du,9.995180\du)--cycle;
\pgfsetlinewidth{0.070000\du}
\pgfsetdash{}{0pt}
\pgfsetdash{}{0pt}
\pgfsetmiterjoin
\definecolor{dialinecolor}{rgb}{0.000000, 0.000000, 0.000000}
\pgfsetstrokecolor{dialinecolor}
\draw (33.926400\du,9.995180\du)--(33.926400\du,12.335180\du)--(37.536400\du,12.335180\du)--(37.536400\du,9.995180\du)--cycle;
% setfont left to latex
\definecolor{dialinecolor}{rgb}{0.000000, 0.000000, 0.000000}
\pgfsetstrokecolor{dialinecolor}
\node at (35.731400\du,10.845180\du){Sequence data};
% setfont left to latex
\definecolor{dialinecolor}{rgb}{0.000000, 0.000000, 0.000000}
\pgfsetstrokecolor{dialinecolor}
\node at (35.731400\du,11.268513\du){structure};
% setfont left to latex
\definecolor{dialinecolor}{rgb}{0.000000, 0.000000, 0.000000}
\pgfsetstrokecolor{dialinecolor}
\node at (35.731400\du,11.691847\du){(M23)};
\definecolor{dialinecolor}{rgb}{1.000000, 1.000000, 1.000000}
\pgfsetfillcolor{dialinecolor}
\fill (45.112500\du,11.422400\du)--(45.112500\du,13.762400\du)--(47.777500\du,13.762400\du)--(47.777500\du,11.422400\du)--cycle;
\pgfsetlinewidth{0.070000\du}
\pgfsetdash{}{0pt}
\pgfsetdash{}{0pt}
\pgfsetmiterjoin
\definecolor{dialinecolor}{rgb}{0.000000, 0.000000, 0.000000}
\pgfsetstrokecolor{dialinecolor}
\draw (45.112500\du,11.422400\du)--(45.112500\du,13.762400\du)--(47.777500\du,13.762400\du)--(47.777500\du,11.422400\du)--cycle;
% setfont left to latex
\definecolor{dialinecolor}{rgb}{0.000000, 0.000000, 0.000000}
\pgfsetstrokecolor{dialinecolor}
\node at (46.445000\du,12.272400\du){Plotting};
% setfont left to latex
\definecolor{dialinecolor}{rgb}{0.000000, 0.000000, 0.000000}
\pgfsetstrokecolor{dialinecolor}
\node at (46.445000\du,12.695733\du){library};
% setfont left to latex
\definecolor{dialinecolor}{rgb}{0.000000, 0.000000, 0.000000}
\pgfsetstrokecolor{dialinecolor}
\node at (46.445000\du,13.119067\du){(M24)};
\pgfsetlinewidth{0.072000\du}
\pgfsetdash{}{0pt}
\pgfsetdash{}{0pt}
\pgfsetmiterjoin
\pgfsetbuttcap
{
\definecolor{dialinecolor}{rgb}{0.000000, 0.000000, 0.000000}
\pgfsetfillcolor{dialinecolor}
% was here!!!
\pgfsetarrowsend{stealth}
{\pgfsetcornersarced{\pgfpoint{0.000000\du}{0.000000\du}}\definecolor{dialinecolor}{rgb}{0.000000, 0.000000, 0.000000}
\pgfsetstrokecolor{dialinecolor}
\draw (41.198200\du,3.695587\du)--(41.198200\du,4.703690\du)--(40.144012\du,4.703690\du)--(40.144012\du,7.535436\du);
}}
\pgfsetlinewidth{0.072000\du}
\pgfsetdash{}{0pt}
\pgfsetdash{}{0pt}
\pgfsetmiterjoin
\pgfsetbuttcap
{
\definecolor{dialinecolor}{rgb}{0.000000, 0.000000, 0.000000}
\pgfsetfillcolor{dialinecolor}
% was here!!!
\pgfsetarrowsend{stealth}
{\pgfsetcornersarced{\pgfpoint{0.000000\du}{0.000000\du}}\definecolor{dialinecolor}{rgb}{0.000000, 0.000000, 0.000000}
\pgfsetstrokecolor{dialinecolor}
\draw (35.175573\du,2.475006\du)--(35.175573\du,6.235093\du)--(34.828900\du,6.235093\du)--(34.828900\du,9.995180\du);
}}
\pgfsetlinewidth{0.072000\du}
\pgfsetdash{}{0pt}
\pgfsetdash{}{0pt}
\pgfsetmiterjoin
\pgfsetbuttcap
{
\definecolor{dialinecolor}{rgb}{0.000000, 0.000000, 0.000000}
\pgfsetfillcolor{dialinecolor}
% was here!!!
\pgfsetarrowsend{stealth}
{\pgfsetcornersarced{\pgfpoint{0.000000\du}{0.000000\du}}\definecolor{dialinecolor}{rgb}{0.000000, 0.000000, 0.000000}
\pgfsetstrokecolor{dialinecolor}
\draw (45.878577\du,1.953557\du)--(44.489871\du,1.953557\du)--(44.489871\du,2.702143\du)--(43.101164\du,2.702143\du);
}}
\pgfsetlinewidth{0.072000\du}
\pgfsetdash{}{0pt}
\pgfsetdash{}{0pt}
\pgfsetmiterjoin
\pgfsetbuttcap
{
\definecolor{dialinecolor}{rgb}{0.000000, 0.000000, 0.000000}
\pgfsetfillcolor{dialinecolor}
% was here!!!
\pgfsetarrowsend{stealth}
{\pgfsetcornersarced{\pgfpoint{0.000000\du}{0.000000\du}}\definecolor{dialinecolor}{rgb}{0.000000, 0.000000, 0.000000}
\pgfsetstrokecolor{dialinecolor}
\draw (47.201400\du,2.935855\du)--(47.201400\du,3.366300\du)--(47.890349\du,3.366300\du)--(47.890349\du,4.428750\du);
}}
\pgfsetlinewidth{0.072000\du}
\pgfsetdash{}{0pt}
\pgfsetdash{}{0pt}
\pgfsetmiterjoin
\pgfsetbuttcap
{
\definecolor{dialinecolor}{rgb}{0.000000, 0.000000, 0.000000}
\pgfsetfillcolor{dialinecolor}
% was here!!!
\pgfsetarrowsend{stealth}
{\pgfsetcornersarced{\pgfpoint{0.000000\du}{0.000000\du}}\definecolor{dialinecolor}{rgb}{0.000000, 0.000000, 0.000000}
\pgfsetstrokecolor{dialinecolor}
\draw (43.065700\du,3.181310\du)--(46.387849\du,3.181310\du)--(46.387849\du,4.428750\du);
}}
\pgfsetlinewidth{0.072000\du}
\pgfsetdash{}{0pt}
\pgfsetdash{}{0pt}
\pgfsetmiterjoin
\pgfsetbuttcap
{
\definecolor{dialinecolor}{rgb}{0.000000, 0.000000, 0.000000}
\pgfsetfillcolor{dialinecolor}
% was here!!!
\pgfsetarrowsend{stealth}
{\pgfsetcornersarced{\pgfpoint{0.000000\du}{0.000000\du}}\definecolor{dialinecolor}{rgb}{0.000000, 0.000000, 0.000000}
\pgfsetstrokecolor{dialinecolor}
\draw (45.605565\du,5.387083\du)--(42.534000\du,5.387083\du)--(42.534000\du,7.535436\du)--(40.895262\du,7.535436\du);
}}
\pgfsetlinewidth{0.072000\du}
\pgfsetdash{}{0pt}
\pgfsetdash{}{0pt}
\pgfsetmiterjoin
\pgfsetbuttcap
{
\definecolor{dialinecolor}{rgb}{0.000000, 0.000000, 0.000000}
\pgfsetfillcolor{dialinecolor}
% was here!!!
\pgfsetarrowsend{stealth}
{\pgfsetcornersarced{\pgfpoint{0.000000\du}{0.000000\du}}\definecolor{dialinecolor}{rgb}{0.000000, 0.000000, 0.000000}
\pgfsetstrokecolor{dialinecolor}
\draw (44.848100\du,8.452211\du)--(44.848100\du,12.086900\du)--(40.006537\du,12.086900\du)--(40.006546\du,14.795371\du);
}}
\pgfsetlinewidth{0.070000\du}
\pgfsetdash{}{0pt}
\pgfsetdash{}{0pt}
\pgfsetmiterjoin
\pgfsetbuttcap
{
\definecolor{dialinecolor}{rgb}{0.000000, 0.000000, 0.000000}
\pgfsetfillcolor{dialinecolor}
% was here!!!
\pgfsetarrowsend{stealth}
{\pgfsetcornersarced{\pgfpoint{0.000000\du}{0.000000\du}}\definecolor{dialinecolor}{rgb}{0.000000, 0.000000, 0.000000}
\pgfsetstrokecolor{dialinecolor}
\draw (47.139099\du,6.380902\du)--(47.139099\du,8.884004\du)--(46.445000\du,8.884004\du)--(46.445000\du,11.387106\du);
}}
\pgfsetlinewidth{0.070000\du}
\pgfsetdash{}{0pt}
\pgfsetdash{}{0pt}
\pgfsetmiterjoin
\pgfsetbuttcap
{
\definecolor{dialinecolor}{rgb}{0.000000, 0.000000, 0.000000}
\pgfsetfillcolor{dialinecolor}
% was here!!!
\pgfsetarrowsend{stealth}
{\pgfsetcornersarced{\pgfpoint{0.000000\du}{0.000000\du}}\definecolor{dialinecolor}{rgb}{0.000000, 0.000000, 0.000000}
\pgfsetstrokecolor{dialinecolor}
\draw (45.112500\du,12.592400\du)--(45.112500\du,12.704200\du)--(41.240900\du,12.704200\du)--(41.240900\du,14.829000\du);
}}
\pgfsetlinewidth{0.070000\du}
\pgfsetdash{}{0pt}
\pgfsetdash{}{0pt}
\pgfsetmiterjoin
\pgfsetbuttcap
{
\definecolor{dialinecolor}{rgb}{0.000000, 0.000000, 0.000000}
\pgfsetfillcolor{dialinecolor}
% was here!!!
\pgfsetarrowsend{stealth}
{\pgfsetcornersarced{\pgfpoint{0.000000\du}{0.000000\du}}\definecolor{dialinecolor}{rgb}{0.000000, 0.000000, 0.000000}
\pgfsetstrokecolor{dialinecolor}
\draw (37.890262\du,7.535436\du)--(36.633900\du,7.535436\du)--(36.633900\du,9.995180\du);
}}
\pgfsetlinewidth{0.070000\du}
\pgfsetdash{}{0pt}
\pgfsetdash{}{0pt}
\pgfsetmiterjoin
\pgfsetbuttcap
{
\definecolor{dialinecolor}{rgb}{0.000000, 0.000000, 0.000000}
\pgfsetfillcolor{dialinecolor}
% was here!!!
\pgfsetarrowsend{stealth}
{\pgfsetcornersarced{\pgfpoint{0.000000\du}{0.000000\du}}\definecolor{dialinecolor}{rgb}{0.000000, 0.000000, 0.000000}
\pgfsetstrokecolor{dialinecolor}
\draw (44.204350\du,8.417057\du)--(44.204350\du,11.750180\du)--(37.536400\du,11.750180\du);
}}
% setfont left to latex
\definecolor{dialinecolor}{rgb}{0.000000, 0.000000, 0.000000}
\pgfsetstrokecolor{dialinecolor}
\node[anchor=west] at (46.445000\du,12.592400\du){};
% setfont left to latex
\definecolor{dialinecolor}{rgb}{0.000000, 0.000000, 0.000000}
\pgfsetstrokecolor{dialinecolor}
\node[anchor=west] at (40.006500\du,15.787300\du){};
% setfont left to latex
\definecolor{dialinecolor}{rgb}{0.000000, 0.000000, 0.000000}
\pgfsetstrokecolor{dialinecolor}
\node[anchor=west] at (35.175573\du,1.269711\du){};
\pgfsetlinewidth{0.070000\du}
\pgfsetdash{}{0pt}
\pgfsetdash{}{0pt}
\pgfsetmiterjoin
\pgfsetbuttcap
{
\definecolor{dialinecolor}{rgb}{0.000000, 0.000000, 0.000000}
\pgfsetfillcolor{dialinecolor}
% was here!!!
\pgfsetarrowsend{stealth}
{\pgfsetcornersarced{\pgfpoint{0.000000\du}{0.000000\du}}\definecolor{dialinecolor}{rgb}{0.000000, 0.000000, 0.000000}
\pgfsetstrokecolor{dialinecolor}
\draw (43.560600\du,8.417060\du)--(43.114900\du,8.417060\du)--(43.114900\du,8.493769\du)--(40.895262\du,8.493769\du);
}}
% setfont left to latex
\definecolor{dialinecolor}{rgb}{0.000000, 0.000000, 0.000000}
\pgfsetstrokecolor{dialinecolor}
\node[anchor=west] at (35.731400\du,11.165180\du){};
\pgfsetlinewidth{0.070000\du}
\pgfsetdash{}{0pt}
\pgfsetdash{}{0pt}
\pgfsetmiterjoin
\pgfsetbuttcap
{
\definecolor{dialinecolor}{rgb}{0.000000, 0.000000, 0.000000}
\pgfsetfillcolor{dialinecolor}
% was here!!!
\pgfsetarrowsend{stealth}
{\pgfsetcornersarced{\pgfpoint{0.000000\du}{0.000000\du}}\definecolor{dialinecolor}{rgb}{0.000000, 0.000000, 0.000000}
\pgfsetstrokecolor{dialinecolor}
\draw (39.330700\du,2.702143\du)--(35.731400\du,2.702143\du)--(35.731400\du,9.995180\du);
}}
\end{tikzpicture}

	\caption{Use Hierarchy}
	\label{figure:UH}
\end{figure}

\begin{figure}[h!]
	\centering
	% Graphic for TeX using PGF
% Title: /home/isobel/Documents/McMaster/PythonCodes/DataAnalysis/Doc/Design/MG/UseHierarchy_IO.dia
% Creator: Dia v0.97.2
% CreationDate: Mon Nov  6 23:35:04 2017
% For: isobel
% \usepackage{tikz}
% The following commands are not supported in PSTricks at present
% We define them conditionally, so when they are implemented,
% this pgf file will use them.
\ifx\du\undefined
  \newlength{\du}
\fi
\setlength{\du}{15\unitlength}
\begin{tikzpicture}
\pgftransformxscale{1.600000}
\pgftransformyscale{-1.600000}
\definecolor{dialinecolor}{rgb}{0.000000, 0.000000, 0.000000}
\pgfsetstrokecolor{dialinecolor}
\definecolor{dialinecolor}{rgb}{1.000000, 1.000000, 1.000000}
\pgfsetfillcolor{dialinecolor}
\definecolor{dialinecolor}{rgb}{1.000000, 1.000000, 1.000000}
\pgfsetfillcolor{dialinecolor}
\fill (23.517234\du,1.225277\du)--(23.517234\du,2.718610\du)--(25.114734\du,2.718610\du)--(25.114734\du,1.225277\du)--cycle;
\pgfsetlinewidth{0.070000\du}
\pgfsetdash{}{0pt}
\pgfsetdash{}{0pt}
\pgfsetmiterjoin
\definecolor{dialinecolor}{rgb}{0.000000, 0.000000, 0.000000}
\pgfsetstrokecolor{dialinecolor}
\draw (23.517234\du,1.225277\du)--(23.517234\du,2.718610\du)--(25.114734\du,2.718610\du)--(25.114734\du,1.225277\du)--cycle;
% setfont left to latex
\definecolor{dialinecolor}{rgb}{0.000000, 0.000000, 0.000000}
\pgfsetstrokecolor{dialinecolor}
\node at (24.315984\du,2.075277\du){I/O};
\definecolor{dialinecolor}{rgb}{1.000000, 1.000000, 1.000000}
\pgfsetfillcolor{dialinecolor}
\fill (18.571881\du,5.101413\du)--(18.571881\du,6.594746\du)--(20.671881\du,6.594746\du)--(20.671881\du,5.101413\du)--cycle;
\pgfsetlinewidth{0.070000\du}
\pgfsetdash{}{0pt}
\pgfsetdash{}{0pt}
\pgfsetmiterjoin
\definecolor{dialinecolor}{rgb}{0.000000, 0.000000, 0.000000}
\pgfsetstrokecolor{dialinecolor}
\draw (18.571881\du,5.101413\du)--(18.571881\du,6.594746\du)--(20.671881\du,6.594746\du)--(20.671881\du,5.101413\du)--cycle;
% setfont left to latex
\definecolor{dialinecolor}{rgb}{0.000000, 0.000000, 0.000000}
\pgfsetstrokecolor{dialinecolor}
\node at (19.621881\du,5.951413\du){*.dm3};
\definecolor{dialinecolor}{rgb}{1.000000, 1.000000, 1.000000}
\pgfsetfillcolor{dialinecolor}
\fill (21.168033\du,5.101413\du)--(21.168033\du,6.594746\du)--(22.948033\du,6.594746\du)--(22.948033\du,5.101413\du)--cycle;
\pgfsetlinewidth{0.070000\du}
\pgfsetdash{}{0pt}
\pgfsetdash{}{0pt}
\pgfsetmiterjoin
\definecolor{dialinecolor}{rgb}{0.000000, 0.000000, 0.000000}
\pgfsetstrokecolor{dialinecolor}
\draw (21.168033\du,5.101413\du)--(21.168033\du,6.594746\du)--(22.948033\du,6.594746\du)--(22.948033\du,5.101413\du)--cycle;
% setfont left to latex
\definecolor{dialinecolor}{rgb}{0.000000, 0.000000, 0.000000}
\pgfsetstrokecolor{dialinecolor}
\node at (22.058033\du,5.951413\du){*.h5};
\pgfsetlinewidth{0.070000\du}
\pgfsetdash{}{0pt}
\pgfsetdash{}{0pt}
\pgfsetmiterjoin
\pgfsetbuttcap
{
\definecolor{dialinecolor}{rgb}{0.000000, 0.000000, 0.000000}
\pgfsetfillcolor{dialinecolor}
% was here!!!
\pgfsetarrowsend{stealth}
{\pgfsetcornersarced{\pgfpoint{0.000000\du}{0.000000\du}}\definecolor{dialinecolor}{rgb}{0.000000, 0.000000, 0.000000}
\pgfsetstrokecolor{dialinecolor}
\draw (24.315984\du,2.753992\du)--(24.315984\du,3.927702\du)--(19.621881\du,3.927702\du)--(19.621881\du,5.101413\du);
}}
\pgfsetlinewidth{0.070000\du}
\pgfsetdash{}{0pt}
\pgfsetdash{}{0pt}
\pgfsetmiterjoin
\pgfsetbuttcap
{
\definecolor{dialinecolor}{rgb}{0.000000, 0.000000, 0.000000}
\pgfsetfillcolor{dialinecolor}
% was here!!!
\pgfsetarrowsend{stealth}
{\pgfsetcornersarced{\pgfpoint{0.000000\du}{0.000000\du}}\definecolor{dialinecolor}{rgb}{0.000000, 0.000000, 0.000000}
\pgfsetstrokecolor{dialinecolor}
\draw (24.315984\du,2.753992\du)--(24.315984\du,3.927702\du)--(22.058033\du,3.927702\du)--(22.058033\du,5.101413\du);
}}
% setfont left to latex
\definecolor{dialinecolor}{rgb}{0.000000, 0.000000, 0.000000}
\pgfsetstrokecolor{dialinecolor}
\node[anchor=west] at (24.315984\du,1.971944\du){};
% setfont left to latex
\definecolor{dialinecolor}{rgb}{0.000000, 0.000000, 0.000000}
\pgfsetstrokecolor{dialinecolor}
\node[anchor=west] at (19.621881\du,5.848080\du){};
\definecolor{dialinecolor}{rgb}{1.000000, 1.000000, 1.000000}
\pgfsetfillcolor{dialinecolor}
\fill (27.931710\du,5.101413\du)--(27.931710\du,6.594746\du)--(29.931710\du,6.594746\du)--(29.931710\du,5.101413\du)--cycle;
\pgfsetlinewidth{0.070000\du}
\pgfsetdash{}{0pt}
\pgfsetdash{}{0pt}
\pgfsetmiterjoin
\definecolor{dialinecolor}{rgb}{0.000000, 0.000000, 0.000000}
\pgfsetstrokecolor{dialinecolor}
\draw (27.931710\du,5.101413\du)--(27.931710\du,6.594746\du)--(29.931710\du,6.594746\du)--(29.931710\du,5.101413\du)--cycle;
% setfont left to latex
\definecolor{dialinecolor}{rgb}{0.000000, 0.000000, 0.000000}
\pgfsetstrokecolor{dialinecolor}
\node at (28.931710\du,5.951413\du){*.png};
\definecolor{dialinecolor}{rgb}{1.000000, 1.000000, 1.000000}
\pgfsetfillcolor{dialinecolor}
\fill (25.661463\du,5.101413\du)--(25.661463\du,6.594746\du)--(27.566463\du,6.594746\du)--(27.566463\du,5.101413\du)--cycle;
\pgfsetlinewidth{0.070000\du}
\pgfsetdash{}{0pt}
\pgfsetdash{}{0pt}
\pgfsetmiterjoin
\definecolor{dialinecolor}{rgb}{0.000000, 0.000000, 0.000000}
\pgfsetstrokecolor{dialinecolor}
\draw (25.661463\du,5.101413\du)--(25.661463\du,6.594746\du)--(27.566463\du,6.594746\du)--(27.566463\du,5.101413\du)--cycle;
% setfont left to latex
\definecolor{dialinecolor}{rgb}{0.000000, 0.000000, 0.000000}
\pgfsetstrokecolor{dialinecolor}
\node at (26.613963\du,5.951413\du){*.csv};
\definecolor{dialinecolor}{rgb}{1.000000, 1.000000, 1.000000}
\pgfsetfillcolor{dialinecolor}
\fill (23.417736\du,5.101413\du)--(23.417736\du,6.594746\du)--(25.252736\du,6.594746\du)--(25.252736\du,5.101413\du)--cycle;
\pgfsetlinewidth{0.070000\du}
\pgfsetdash{}{0pt}
\pgfsetdash{}{0pt}
\pgfsetmiterjoin
\definecolor{dialinecolor}{rgb}{0.000000, 0.000000, 0.000000}
\pgfsetstrokecolor{dialinecolor}
\draw (23.417736\du,5.101413\du)--(23.417736\du,6.594746\du)--(25.252736\du,6.594746\du)--(25.252736\du,5.101413\du)--cycle;
% setfont left to latex
\definecolor{dialinecolor}{rgb}{0.000000, 0.000000, 0.000000}
\pgfsetstrokecolor{dialinecolor}
\node at (24.335236\du,5.951413\du){*.rpl};
\pgfsetlinewidth{0.070000\du}
\pgfsetdash{}{0pt}
\pgfsetdash{}{0pt}
\pgfsetmiterjoin
\pgfsetbuttcap
{
\definecolor{dialinecolor}{rgb}{0.000000, 0.000000, 0.000000}
\pgfsetfillcolor{dialinecolor}
% was here!!!
\pgfsetarrowsend{stealth}
{\pgfsetcornersarced{\pgfpoint{0.000000\du}{0.000000\du}}\definecolor{dialinecolor}{rgb}{0.000000, 0.000000, 0.000000}
\pgfsetstrokecolor{dialinecolor}
\draw (24.315984\du,2.753417\du)--(24.315984\du,3.936900\du)--(24.335236\du,3.936900\du)--(24.335236\du,5.066997\du);
}}
\pgfsetlinewidth{0.070000\du}
\pgfsetdash{}{0pt}
\pgfsetdash{}{0pt}
\pgfsetmiterjoin
\pgfsetbuttcap
{
\definecolor{dialinecolor}{rgb}{0.000000, 0.000000, 0.000000}
\pgfsetfillcolor{dialinecolor}
% was here!!!
\pgfsetarrowsend{stealth}
{\pgfsetcornersarced{\pgfpoint{0.000000\du}{0.000000\du}}\definecolor{dialinecolor}{rgb}{0.000000, 0.000000, 0.000000}
\pgfsetstrokecolor{dialinecolor}
\draw (24.315984\du,2.753992\du)--(24.315984\du,3.927702\du)--(26.613963\du,3.927702\du)--(26.613963\du,5.101413\du);
}}
\pgfsetlinewidth{0.070000\du}
\pgfsetdash{}{0pt}
\pgfsetdash{}{0pt}
\pgfsetmiterjoin
\pgfsetbuttcap
{
\definecolor{dialinecolor}{rgb}{0.000000, 0.000000, 0.000000}
\pgfsetfillcolor{dialinecolor}
% was here!!!
\pgfsetarrowsend{stealth}
{\pgfsetcornersarced{\pgfpoint{0.000000\du}{0.000000\du}}\definecolor{dialinecolor}{rgb}{0.000000, 0.000000, 0.000000}
\pgfsetstrokecolor{dialinecolor}
\draw (24.315984\du,2.753993\du)--(24.315984\du,3.919223\du)--(28.931710\du,3.919223\du)--(28.931710\du,5.067307\du);
}}
\end{tikzpicture}

	\caption{Use Hierarchy I/O}
	\label{figure:UHIO}
\end{figure}

\begin{figure}[h!]
	\centering
	% Graphic for TeX using PGF
% Title: /home/isobel/Documents/McMaster/PythonCodes/DataAnalysis/Doc/Design/MG/UseHierarchy_DataProcessing.dia
% Creator: Dia v0.97.2
% CreationDate: Wed Nov  8 14:43:20 2017
% For: isobel
% \usepackage{tikz}
% The following commands are not supported in PSTricks at present
% We define them conditionally, so when they are implemented,
% this pgf file will use them.
\ifx\du\undefined
  \newlength{\du}
\fi
\setlength{\du}{15\unitlength}
\begin{tikzpicture}
\pgftransformxscale{1.500000}
\pgftransformyscale{-1.600000}
\definecolor{dialinecolor}{rgb}{0.000000, 0.000000, 0.000000}
\pgfsetstrokecolor{dialinecolor}
\definecolor{dialinecolor}{rgb}{1.000000, 1.000000, 1.000000}
\pgfsetfillcolor{dialinecolor}
\definecolor{dialinecolor}{rgb}{1.000000, 1.000000, 1.000000}
\pgfsetfillcolor{dialinecolor}
\fill (22.901007\du,0.456032\du)--(22.901007\du,2.796032\du)--(26.768507\du,2.796032\du)--(26.768507\du,0.456032\du)--cycle;
\pgfsetlinewidth{0.070000\du}
\pgfsetdash{}{0pt}
\pgfsetdash{}{0pt}
\pgfsetmiterjoin
\definecolor{dialinecolor}{rgb}{0.000000, 0.000000, 0.000000}
\pgfsetstrokecolor{dialinecolor}
\draw (22.901007\du,0.456032\du)--(22.901007\du,2.796032\du)--(26.768507\du,2.796032\du)--(26.768507\du,0.456032\du)--cycle;
% setfont left to latex
\definecolor{dialinecolor}{rgb}{0.000000, 0.000000, 0.000000}
\pgfsetstrokecolor{dialinecolor}
\node at (24.834757\du,1.517698\du){Data Processing};
% setfont left to latex
\definecolor{dialinecolor}{rgb}{0.000000, 0.000000, 0.000000}
\pgfsetstrokecolor{dialinecolor}
\node at (24.834757\du,1.941032\du){(M10-M13)};
\definecolor{dialinecolor}{rgb}{1.000000, 1.000000, 1.000000}
\pgfsetfillcolor{dialinecolor}
\fill (17.536444\du,4.920106\du)--(17.536444\du,6.836772\du)--(21.078944\du,6.836772\du)--(21.078944\du,4.920106\du)--cycle;
\pgfsetlinewidth{0.070000\du}
\pgfsetdash{}{0pt}
\pgfsetdash{}{0pt}
\pgfsetmiterjoin
\definecolor{dialinecolor}{rgb}{0.000000, 0.000000, 0.000000}
\pgfsetstrokecolor{dialinecolor}
\draw (17.536444\du,4.920106\du)--(17.536444\du,6.836772\du)--(21.078944\du,6.836772\du)--(21.078944\du,4.920106\du)--cycle;
% setfont left to latex
\definecolor{dialinecolor}{rgb}{0.000000, 0.000000, 0.000000}
\pgfsetstrokecolor{dialinecolor}
\node at (19.307694\du,5.770106\du){Deconvolution};
% setfont left to latex
\definecolor{dialinecolor}{rgb}{0.000000, 0.000000, 0.000000}
\pgfsetstrokecolor{dialinecolor}
\node at (19.307694\du,6.193439\du){(M10)};
\definecolor{dialinecolor}{rgb}{1.000000, 1.000000, 1.000000}
\pgfsetfillcolor{dialinecolor}
\fill (21.252241\du,4.920106\du)--(21.252241\du,6.836772\du)--(24.769741\du,6.836772\du)--(24.769741\du,4.920106\du)--cycle;
\pgfsetlinewidth{0.070000\du}
\pgfsetdash{}{0pt}
\pgfsetdash{}{0pt}
\pgfsetmiterjoin
\definecolor{dialinecolor}{rgb}{0.000000, 0.000000, 0.000000}
\pgfsetstrokecolor{dialinecolor}
\draw (21.252241\du,4.920106\du)--(21.252241\du,6.836772\du)--(24.769741\du,6.836772\du)--(24.769741\du,4.920106\du)--cycle;
% setfont left to latex
\definecolor{dialinecolor}{rgb}{0.000000, 0.000000, 0.000000}
\pgfsetstrokecolor{dialinecolor}
\node at (23.010991\du,5.770106\du){Normalization};
% setfont left to latex
\definecolor{dialinecolor}{rgb}{0.000000, 0.000000, 0.000000}
\pgfsetstrokecolor{dialinecolor}
\node at (23.010991\du,6.193439\du){(M11)};
\definecolor{dialinecolor}{rgb}{1.000000, 1.000000, 1.000000}
\pgfsetfillcolor{dialinecolor}
\fill (24.968710\du,4.920106\du)--(24.968710\du,6.836772\du)--(28.718710\du,6.836772\du)--(28.718710\du,4.920106\du)--cycle;
\pgfsetlinewidth{0.070000\du}
\pgfsetdash{}{0pt}
\pgfsetdash{}{0pt}
\pgfsetmiterjoin
\definecolor{dialinecolor}{rgb}{0.000000, 0.000000, 0.000000}
\pgfsetstrokecolor{dialinecolor}
\draw (24.968710\du,4.920106\du)--(24.968710\du,6.836772\du)--(28.718710\du,6.836772\du)--(28.718710\du,4.920106\du)--cycle;
% setfont left to latex
\definecolor{dialinecolor}{rgb}{0.000000, 0.000000, 0.000000}
\pgfsetstrokecolor{dialinecolor}
\node at (26.843710\du,5.770106\du){Gain correction};
% setfont left to latex
\definecolor{dialinecolor}{rgb}{0.000000, 0.000000, 0.000000}
\pgfsetstrokecolor{dialinecolor}
\node at (26.843710\du,6.193439\du){(M12)};
\definecolor{dialinecolor}{rgb}{1.000000, 1.000000, 1.000000}
\pgfsetfillcolor{dialinecolor}
\fill (28.936704\du,4.920106\du)--(28.936704\du,7.260106\du)--(32.199204\du,7.260106\du)--(32.199204\du,4.920106\du)--cycle;
\pgfsetlinewidth{0.070000\du}
\pgfsetdash{}{0pt}
\pgfsetdash{}{0pt}
\pgfsetmiterjoin
\definecolor{dialinecolor}{rgb}{0.000000, 0.000000, 0.000000}
\pgfsetstrokecolor{dialinecolor}
\draw (28.936704\du,4.920106\du)--(28.936704\du,7.260106\du)--(32.199204\du,7.260106\du)--(32.199204\du,4.920106\du)--cycle;
% setfont left to latex
\definecolor{dialinecolor}{rgb}{0.000000, 0.000000, 0.000000}
\pgfsetstrokecolor{dialinecolor}
\node at (30.567954\du,5.770106\du){Background };
% setfont left to latex
\definecolor{dialinecolor}{rgb}{0.000000, 0.000000, 0.000000}
\pgfsetstrokecolor{dialinecolor}
\node at (30.567954\du,6.193439\du){correction};
% setfont left to latex
\definecolor{dialinecolor}{rgb}{0.000000, 0.000000, 0.000000}
\pgfsetstrokecolor{dialinecolor}
\node at (30.567954\du,6.616772\du){(M13)};
% setfont left to latex
\definecolor{dialinecolor}{rgb}{0.000000, 0.000000, 0.000000}
\pgfsetstrokecolor{dialinecolor}
\node[anchor=west] at (30.567954\du,6.090106\du){};
\pgfsetlinewidth{0.070000\du}
\pgfsetdash{}{0pt}
\pgfsetdash{}{0pt}
\pgfsetmiterjoin
\pgfsetbuttcap
{
\definecolor{dialinecolor}{rgb}{0.000000, 0.000000, 0.000000}
\pgfsetfillcolor{dialinecolor}
% was here!!!
\pgfsetarrowsend{stealth}
{\pgfsetcornersarced{\pgfpoint{0.000000\du}{0.000000\du}}\definecolor{dialinecolor}{rgb}{0.000000, 0.000000, 0.000000}
\pgfsetstrokecolor{dialinecolor}
\draw (24.834757\du,2.831326\du)--(24.834757\du,3.875716\du)--(19.307694\du,3.875716\du)--(19.307694\du,4.920106\du);
}}
\pgfsetlinewidth{0.070000\du}
\pgfsetdash{}{0pt}
\pgfsetdash{}{0pt}
\pgfsetmiterjoin
\pgfsetbuttcap
{
\definecolor{dialinecolor}{rgb}{0.000000, 0.000000, 0.000000}
\pgfsetfillcolor{dialinecolor}
% was here!!!
\pgfsetarrowsend{stealth}
{\pgfsetcornersarced{\pgfpoint{0.000000\du}{0.000000\du}}\definecolor{dialinecolor}{rgb}{0.000000, 0.000000, 0.000000}
\pgfsetstrokecolor{dialinecolor}
\draw (24.834757\du,2.831326\du)--(24.834757\du,3.875716\du)--(23.890366\du,3.875716\du)--(23.890366\du,4.920106\du);
}}
\pgfsetlinewidth{0.070000\du}
\pgfsetdash{}{0pt}
\pgfsetdash{}{0pt}
\pgfsetmiterjoin
\pgfsetbuttcap
{
\definecolor{dialinecolor}{rgb}{0.000000, 0.000000, 0.000000}
\pgfsetfillcolor{dialinecolor}
% was here!!!
\pgfsetarrowsend{stealth}
{\pgfsetcornersarced{\pgfpoint{0.000000\du}{0.000000\du}}\definecolor{dialinecolor}{rgb}{0.000000, 0.000000, 0.000000}
\pgfsetstrokecolor{dialinecolor}
\draw (24.834757\du,2.796032\du)--(24.834757\du,3.873361\du)--(27.781210\du,3.873361\du)--(27.781210\du,4.920106\du);
}}
\pgfsetlinewidth{0.070000\du}
\pgfsetdash{}{0pt}
\pgfsetdash{}{0pt}
\pgfsetmiterjoin
\pgfsetbuttcap
{
\definecolor{dialinecolor}{rgb}{0.000000, 0.000000, 0.000000}
\pgfsetfillcolor{dialinecolor}
% was here!!!
\pgfsetarrowsend{stealth}
{\pgfsetcornersarced{\pgfpoint{0.000000\du}{0.000000\du}}\definecolor{dialinecolor}{rgb}{0.000000, 0.000000, 0.000000}
\pgfsetstrokecolor{dialinecolor}
\draw (24.834757\du,2.831326\du)--(24.834757\du,3.875716\du)--(30.567954\du,3.875716\du)--(30.567954\du,4.920106\du);
}}
% setfont left to latex
\definecolor{dialinecolor}{rgb}{0.000000, 0.000000, 0.000000}
\pgfsetstrokecolor{dialinecolor}
\node[anchor=west] at (24.834757\du,1.626032\du){};
\end{tikzpicture}

	\caption{Use Hierarchy Data Processing}
	\label{figure:UHDP}
\end{figure}

\begin{figure}[h!]
	\centering
	% Graphic for TeX using PGF
% Title: /home/isobel/Documents/McMaster/PythonCodes/DataAnalysis/Doc/Design/MG/UseHierarchy_DataExtraction.dia
% Creator: Dia v0.97.2
% CreationDate: Mon Nov  6 23:35:20 2017
% For: isobel
% \usepackage{tikz}
% The following commands are not supported in PSTricks at present
% We define them conditionally, so when they are implemented,
% this pgf file will use them.
\ifx\du\undefined
  \newlength{\du}
\fi
\setlength{\du}{15\unitlength}
\begin{tikzpicture}
\pgftransformxscale{1.600000}
\pgftransformyscale{-1.600000}
\definecolor{dialinecolor}{rgb}{0.000000, 0.000000, 0.000000}
\pgfsetstrokecolor{dialinecolor}
\definecolor{dialinecolor}{rgb}{1.000000, 1.000000, 1.000000}
\pgfsetfillcolor{dialinecolor}
\definecolor{dialinecolor}{rgb}{1.000000, 1.000000, 1.000000}
\pgfsetfillcolor{dialinecolor}
\fill (39.733265\du,0.952749\du)--(39.733265\du,2.869416\du)--(42.548265\du,2.869416\du)--(42.548265\du,0.952749\du)--cycle;
\pgfsetlinewidth{0.070000\du}
\pgfsetdash{}{0pt}
\pgfsetdash{}{0pt}
\pgfsetmiterjoin
\definecolor{dialinecolor}{rgb}{0.000000, 0.000000, 0.000000}
\pgfsetstrokecolor{dialinecolor}
\draw (39.733265\du,0.952749\du)--(39.733265\du,2.869416\du)--(42.548265\du,2.869416\du)--(42.548265\du,0.952749\du)--cycle;
% setfont left to latex
\definecolor{dialinecolor}{rgb}{0.000000, 0.000000, 0.000000}
\pgfsetstrokecolor{dialinecolor}
\node at (41.140765\du,1.802749\du){Data};
% setfont left to latex
\definecolor{dialinecolor}{rgb}{0.000000, 0.000000, 0.000000}
\pgfsetstrokecolor{dialinecolor}
\node at (41.140765\du,2.226083\du){extraction};
\definecolor{dialinecolor}{rgb}{1.000000, 1.000000, 1.000000}
\pgfsetfillcolor{dialinecolor}
\fill (36.402663\du,6.492022\du)--(36.402663\du,7.985356\du)--(38.847663\du,7.985356\du)--(38.847663\du,6.492022\du)--cycle;
\pgfsetlinewidth{0.070000\du}
\pgfsetdash{}{0pt}
\pgfsetdash{}{0pt}
\pgfsetmiterjoin
\definecolor{dialinecolor}{rgb}{0.000000, 0.000000, 0.000000}
\pgfsetstrokecolor{dialinecolor}
\draw (36.402663\du,6.492022\du)--(36.402663\du,7.985356\du)--(38.847663\du,7.985356\du)--(38.847663\du,6.492022\du)--cycle;
% setfont left to latex
\definecolor{dialinecolor}{rgb}{0.000000, 0.000000, 0.000000}
\pgfsetstrokecolor{dialinecolor}
\node at (37.625163\du,7.342022\du){1D slice};
\definecolor{dialinecolor}{rgb}{1.000000, 1.000000, 1.000000}
\pgfsetfillcolor{dialinecolor}
\fill (43.261239\du,6.492022\du)--(43.261239\du,7.985356\du)--(45.828739\du,7.985356\du)--(45.828739\du,6.492022\du)--cycle;
\pgfsetlinewidth{0.070000\du}
\pgfsetdash{}{0pt}
\pgfsetdash{}{0pt}
\pgfsetmiterjoin
\definecolor{dialinecolor}{rgb}{0.000000, 0.000000, 0.000000}
\pgfsetstrokecolor{dialinecolor}
\draw (43.261239\du,6.492022\du)--(43.261239\du,7.985356\du)--(45.828739\du,7.985356\du)--(45.828739\du,6.492022\du)--cycle;
% setfont left to latex
\definecolor{dialinecolor}{rgb}{0.000000, 0.000000, 0.000000}
\pgfsetstrokecolor{dialinecolor}
\node at (44.544989\du,7.342022\du){2D mask};
\definecolor{dialinecolor}{rgb}{1.000000, 1.000000, 1.000000}
\pgfsetfillcolor{dialinecolor}
\fill (39.857191\du,4.629399\du)--(39.857191\du,6.122733\du)--(42.424691\du,6.122733\du)--(42.424691\du,4.629399\du)--cycle;
\pgfsetlinewidth{0.070000\du}
\pgfsetdash{}{0pt}
\pgfsetdash{}{0pt}
\pgfsetmiterjoin
\definecolor{dialinecolor}{rgb}{0.000000, 0.000000, 0.000000}
\pgfsetstrokecolor{dialinecolor}
\draw (39.857191\du,4.629399\du)--(39.857191\du,6.122733\du)--(42.424691\du,6.122733\du)--(42.424691\du,4.629399\du)--cycle;
% setfont left to latex
\definecolor{dialinecolor}{rgb}{0.000000, 0.000000, 0.000000}
\pgfsetstrokecolor{dialinecolor}
\node at (41.140941\du,5.479399\du){3D mask};
\pgfsetlinewidth{0.070000\du}
\pgfsetdash{}{0pt}
\pgfsetdash{}{0pt}
\pgfsetmiterjoin
\pgfsetbuttcap
{
\definecolor{dialinecolor}{rgb}{0.000000, 0.000000, 0.000000}
\pgfsetfillcolor{dialinecolor}
% was here!!!
\pgfsetarrowsend{stealth}
{\pgfsetcornersarced{\pgfpoint{0.000000\du}{0.000000\du}}\definecolor{dialinecolor}{rgb}{0.000000, 0.000000, 0.000000}
\pgfsetstrokecolor{dialinecolor}
\draw (41.140765\du,2.869416\du)--(41.140765\du,3.677221\du)--(37.625163\du,3.677221\du)--(37.625163\du,6.492022\du);
}}
\pgfsetlinewidth{0.070000\du}
\pgfsetdash{}{0pt}
\pgfsetdash{}{0pt}
\pgfsetmiterjoin
\pgfsetbuttcap
{
\definecolor{dialinecolor}{rgb}{0.000000, 0.000000, 0.000000}
\pgfsetfillcolor{dialinecolor}
% was here!!!
\pgfsetarrowsend{stealth}
{\pgfsetcornersarced{\pgfpoint{0.000000\du}{0.000000\du}}\definecolor{dialinecolor}{rgb}{0.000000, 0.000000, 0.000000}
\pgfsetstrokecolor{dialinecolor}
\draw (41.140765\du,2.869416\du)--(41.140765\du,3.677221\du)--(44.544989\du,3.677221\du)--(44.544989\du,6.492022\du);
}}
\pgfsetlinewidth{0.070000\du}
\pgfsetdash{}{0pt}
\pgfsetdash{}{0pt}
\pgfsetmiterjoin
\pgfsetbuttcap
{
\definecolor{dialinecolor}{rgb}{0.000000, 0.000000, 0.000000}
\pgfsetfillcolor{dialinecolor}
% was here!!!
\pgfsetarrowsend{stealth}
{\pgfsetcornersarced{\pgfpoint{0.000000\du}{0.000000\du}}\definecolor{dialinecolor}{rgb}{0.000000, 0.000000, 0.000000}
\pgfsetstrokecolor{dialinecolor}
\draw (41.140765\du,2.869416\du)--(41.140765\du,3.749408\du)--(41.140941\du,3.749408\du)--(41.140941\du,4.629399\du);
}}
% setfont left to latex
\definecolor{dialinecolor}{rgb}{0.000000, 0.000000, 0.000000}
\pgfsetstrokecolor{dialinecolor}
\node[anchor=west] at (37.625163\du,7.238689\du){};
\pgfsetlinewidth{0.070000\du}
\pgfsetdash{}{0pt}
\pgfsetdash{}{0pt}
\pgfsetmiterjoin
\pgfsetbuttcap
{
\definecolor{dialinecolor}{rgb}{0.000000, 0.000000, 0.000000}
\pgfsetfillcolor{dialinecolor}
% was here!!!
\pgfsetarrowsend{stealth}
{\pgfsetcornersarced{\pgfpoint{0.000000\du}{0.000000\du}}\definecolor{dialinecolor}{rgb}{0.000000, 0.000000, 0.000000}
\pgfsetstrokecolor{dialinecolor}
\draw (41.140941\du,6.122733\du)--(41.140941\du,7.238689\du)--(43.225917\du,7.238689\du);
}}
\pgfsetlinewidth{0.070000\du}
\pgfsetdash{}{0pt}
\pgfsetdash{}{0pt}
\pgfsetmiterjoin
\pgfsetbuttcap
{
\definecolor{dialinecolor}{rgb}{0.000000, 0.000000, 0.000000}
\pgfsetfillcolor{dialinecolor}
% was here!!!
\pgfsetarrowsend{stealth}
{\pgfsetcornersarced{\pgfpoint{0.000000\du}{0.000000\du}}\definecolor{dialinecolor}{rgb}{0.000000, 0.000000, 0.000000}
\pgfsetstrokecolor{dialinecolor}
\draw (41.140941\du,6.122733\du)--(41.140941\du,7.238689\du)--(38.882970\du,7.238689\du);
}}
% setfont left to latex
\definecolor{dialinecolor}{rgb}{0.000000, 0.000000, 0.000000}
\pgfsetstrokecolor{dialinecolor}
\node[anchor=west] at (41.140941\du,5.376066\du){};
\end{tikzpicture}

	\caption{Use Hierarchy Data Extraction}
	\label{figure:UHDE}
\end{figure}

\begin{figure}[h!]
	\centering
	% Graphic for TeX using PGF
% Title: /home/isobel/Documents/McMaster/PythonCodes/DataAnalysis/Doc/Design/MG/UseHierarchy_Display.dia
% Creator: Dia v0.97.2
% CreationDate: Wed Nov  8 14:41:56 2017
% For: isobel
% \usepackage{tikz}
% The following commands are not supported in PSTricks at present
% We define them conditionally, so when they are implemented,
% this pgf file will use them.
\ifx\du\undefined
  \newlength{\du}
\fi
\setlength{\du}{15\unitlength}
\begin{tikzpicture}
\pgftransformxscale{1.500000}
\pgftransformyscale{-1.60000}
\definecolor{dialinecolor}{rgb}{0.000000, 0.000000, 0.000000}
\pgfsetstrokecolor{dialinecolor}
\definecolor{dialinecolor}{rgb}{1.000000, 1.000000, 1.000000}
\pgfsetfillcolor{dialinecolor}
\definecolor{dialinecolor}{rgb}{1.000000, 1.000000, 1.000000}
\pgfsetfillcolor{dialinecolor}
\fill (39.752769\du,0.586428\du)--(39.752769\du,2.503095\du)--(42.757769\du,2.503095\du)--(42.757769\du,0.586428\du)--cycle;
\pgfsetlinewidth{0.070000\du}
\pgfsetdash{}{0pt}
\pgfsetdash{}{0pt}
\pgfsetmiterjoin
\definecolor{dialinecolor}{rgb}{0.000000, 0.000000, 0.000000}
\pgfsetstrokecolor{dialinecolor}
\draw (39.752769\du,0.586428\du)--(39.752769\du,2.503095\du)--(42.757769\du,2.503095\du)--(42.757769\du,0.586428\du)--cycle;
% setfont left to latex
\definecolor{dialinecolor}{rgb}{0.000000, 0.000000, 0.000000}
\pgfsetstrokecolor{dialinecolor}
\node at (41.255269\du,1.436428\du){Display};
% setfont left to latex
\definecolor{dialinecolor}{rgb}{0.000000, 0.000000, 0.000000}
\pgfsetstrokecolor{dialinecolor}
\node at (41.255269\du,1.859762\du){(M17-M19)};
\definecolor{dialinecolor}{rgb}{1.000000, 1.000000, 1.000000}
\pgfsetfillcolor{dialinecolor}
\fill (34.749484\du,6.886260\du)--(34.749484\du,9.226260\du)--(38.209484\du,9.226260\du)--(38.209484\du,6.886260\du)--cycle;
\pgfsetlinewidth{0.070000\du}
\pgfsetdash{}{0pt}
\pgfsetdash{}{0pt}
\pgfsetmiterjoin
\definecolor{dialinecolor}{rgb}{0.000000, 0.000000, 0.000000}
\pgfsetstrokecolor{dialinecolor}
\draw (34.749484\du,6.886260\du)--(34.749484\du,9.226260\du)--(38.209484\du,9.226260\du)--(38.209484\du,6.886260\du)--cycle;
% setfont left to latex
\definecolor{dialinecolor}{rgb}{0.000000, 0.000000, 0.000000}
\pgfsetstrokecolor{dialinecolor}
\node at (36.479484\du,7.736260\du){1D};
% setfont left to latex
\definecolor{dialinecolor}{rgb}{0.000000, 0.000000, 0.000000}
\pgfsetstrokecolor{dialinecolor}
\node at (36.479484\du,8.159594\du){spectrum plot};
% setfont left to latex
\definecolor{dialinecolor}{rgb}{0.000000, 0.000000, 0.000000}
\pgfsetstrokecolor{dialinecolor}
\node at (36.479484\du,8.582927\du){(M17)};
\definecolor{dialinecolor}{rgb}{1.000000, 1.000000, 1.000000}
\pgfsetfillcolor{dialinecolor}
\fill (44.283584\du,6.886260\du)--(44.283584\du,9.226260\du)--(47.173584\du,9.226260\du)--(47.173584\du,6.886260\du)--cycle;
\pgfsetlinewidth{0.070000\du}
\pgfsetdash{}{0pt}
\pgfsetdash{}{0pt}
\pgfsetmiterjoin
\definecolor{dialinecolor}{rgb}{0.000000, 0.000000, 0.000000}
\pgfsetstrokecolor{dialinecolor}
\draw (44.283584\du,6.886260\du)--(44.283584\du,9.226260\du)--(47.173584\du,9.226260\du)--(47.173584\du,6.886260\du)--cycle;
% setfont left to latex
\definecolor{dialinecolor}{rgb}{0.000000, 0.000000, 0.000000}
\pgfsetstrokecolor{dialinecolor}
\node at (45.728584\du,7.736260\du){2D};
% setfont left to latex
\definecolor{dialinecolor}{rgb}{0.000000, 0.000000, 0.000000}
\pgfsetstrokecolor{dialinecolor}
\node at (45.728584\du,8.159594\du){image plot};
% setfont left to latex
\definecolor{dialinecolor}{rgb}{0.000000, 0.000000, 0.000000}
\pgfsetstrokecolor{dialinecolor}
\node at (45.728584\du,8.582927\du){(M18)};
\definecolor{dialinecolor}{rgb}{1.000000, 1.000000, 1.000000}
\pgfsetfillcolor{dialinecolor}
\fill (38.946439\du,4.298887\du)--(38.946439\du,6.638887\du)--(43.561439\du,6.638887\du)--(43.561439\du,4.298887\du)--cycle;
\pgfsetlinewidth{0.070000\du}
\pgfsetdash{}{0pt}
\pgfsetdash{}{0pt}
\pgfsetmiterjoin
\definecolor{dialinecolor}{rgb}{0.000000, 0.000000, 0.000000}
\pgfsetstrokecolor{dialinecolor}
\draw (38.946439\du,4.298887\du)--(38.946439\du,6.638887\du)--(43.561439\du,6.638887\du)--(43.561439\du,4.298887\du)--cycle;
% setfont left to latex
\definecolor{dialinecolor}{rgb}{0.000000, 0.000000, 0.000000}
\pgfsetstrokecolor{dialinecolor}
\node at (41.253939\du,5.148887\du){3D};
% setfont left to latex
\definecolor{dialinecolor}{rgb}{0.000000, 0.000000, 0.000000}
\pgfsetstrokecolor{dialinecolor}
\node at (41.253939\du,5.572221\du){spectrum image plot};
% setfont left to latex
\definecolor{dialinecolor}{rgb}{0.000000, 0.000000, 0.000000}
\pgfsetstrokecolor{dialinecolor}
\node at (41.253939\du,5.995554\du){(M19)};
\pgfsetlinewidth{0.070000\du}
\pgfsetdash{}{0pt}
\pgfsetdash{}{0pt}
\pgfsetmiterjoin
\pgfsetbuttcap
{
\definecolor{dialinecolor}{rgb}{0.000000, 0.000000, 0.000000}
\pgfsetfillcolor{dialinecolor}
% was here!!!
\pgfsetarrowsend{stealth}
{\pgfsetcornersarced{\pgfpoint{0.000000\du}{0.000000\du}}\definecolor{dialinecolor}{rgb}{0.000000, 0.000000, 0.000000}
\pgfsetstrokecolor{dialinecolor}
\draw (41.255269\du,2.503095\du)--(41.255269\du,2.989148\du)--(35.614484\du,2.989148\du)--(35.614484\du,6.886260\du);
}}
\pgfsetlinewidth{0.070000\du}
\pgfsetdash{}{0pt}
\pgfsetdash{}{0pt}
\pgfsetmiterjoin
\pgfsetbuttcap
{
\definecolor{dialinecolor}{rgb}{0.000000, 0.000000, 0.000000}
\pgfsetfillcolor{dialinecolor}
% was here!!!
\pgfsetarrowsend{stealth}
{\pgfsetcornersarced{\pgfpoint{0.000000\du}{0.000000\du}}\definecolor{dialinecolor}{rgb}{0.000000, 0.000000, 0.000000}
\pgfsetstrokecolor{dialinecolor}
\draw (41.255269\du,2.503095\du)--(41.255269\du,2.989148\du)--(46.451084\du,2.989148\du)--(46.451084\du,6.886260\du);
}}
\pgfsetlinewidth{0.070000\du}
\pgfsetdash{}{0pt}
\pgfsetdash{}{0pt}
\pgfsetmiterjoin
\pgfsetbuttcap
{
\definecolor{dialinecolor}{rgb}{0.000000, 0.000000, 0.000000}
\pgfsetfillcolor{dialinecolor}
% was here!!!
\pgfsetarrowsend{stealth}
{\pgfsetcornersarced{\pgfpoint{0.000000\du}{0.000000\du}}\definecolor{dialinecolor}{rgb}{0.000000, 0.000000, 0.000000}
\pgfsetstrokecolor{dialinecolor}
\draw (41.255269\du,2.503095\du)--(41.255269\du,3.400991\du)--(41.253939\du,3.400991\du)--(41.253939\du,4.298887\du);
}}
% setfont left to latex
\definecolor{dialinecolor}{rgb}{0.000000, 0.000000, 0.000000}
\pgfsetstrokecolor{dialinecolor}
\node[anchor=west] at (36.479484\du,8.056260\du){};
\pgfsetlinewidth{0.070000\du}
\pgfsetdash{}{0pt}
\pgfsetdash{}{0pt}
\pgfsetmiterjoin
\pgfsetbuttcap
{
\definecolor{dialinecolor}{rgb}{0.000000, 0.000000, 0.000000}
\pgfsetfillcolor{dialinecolor}
% was here!!!
\pgfsetarrowsend{stealth}
{\pgfsetcornersarced{\pgfpoint{0.000000\du}{0.000000\du}}\definecolor{dialinecolor}{rgb}{0.000000, 0.000000, 0.000000}
\pgfsetstrokecolor{dialinecolor}
\draw (43.561439\du,5.468887\du)--(45.006084\du,5.468887\du)--(45.006084\du,6.886260\du);
}}
\pgfsetlinewidth{0.070000\du}
\pgfsetdash{}{0pt}
\pgfsetdash{}{0pt}
\pgfsetmiterjoin
\pgfsetbuttcap
{
\definecolor{dialinecolor}{rgb}{0.000000, 0.000000, 0.000000}
\pgfsetfillcolor{dialinecolor}
% was here!!!
\pgfsetarrowsend{stealth}
{\pgfsetcornersarced{\pgfpoint{0.000000\du}{0.000000\du}}\definecolor{dialinecolor}{rgb}{0.000000, 0.000000, 0.000000}
\pgfsetstrokecolor{dialinecolor}
\draw (38.911153\du,5.468887\du)--(37.344484\du,5.468887\du)--(37.344484\du,6.886260\du);
}}
% setfont left to latex
\definecolor{dialinecolor}{rgb}{0.000000, 0.000000, 0.000000}
\pgfsetstrokecolor{dialinecolor}
\node[anchor=west] at (41.255269\du,1.544762\du){};
\end{tikzpicture}

	\caption{Use Hierarchy Display}
	\label{figure:UHDisp}
\end{figure}

\begin{figure}[h!]
	\centering
	% Graphic for TeX using PGF
% Title: /home/isobel/Documents/McMaster/PythonCodes/DataAnalysis/Doc/Design/MG/UseHierarchy_Data.dia
% Creator: Dia v0.97.2
% CreationDate: Wed Nov  8 14:44:26 2017
% For: isobel
% \usepackage{tikz}
% The following commands are not supported in PSTricks at present
% We define them conditionally, so when they are implemented,
% this pgf file will use them.
\ifx\du\undefined
  \newlength{\du}
\fi
\setlength{\du}{15\unitlength}
\begin{tikzpicture}
\pgftransformxscale{1.600000}
\pgftransformyscale{-1.7500000}
\definecolor{dialinecolor}{rgb}{0.000000, 0.000000, 0.000000}
\pgfsetstrokecolor{dialinecolor}
\definecolor{dialinecolor}{rgb}{1.000000, 1.000000, 1.000000}
\pgfsetfillcolor{dialinecolor}
\definecolor{dialinecolor}{rgb}{1.000000, 1.000000, 1.000000}
\pgfsetfillcolor{dialinecolor}
\fill (22.740884\du,1.165501\du)--(22.740884\du,3.082167\du)--(25.745884\du,3.082167\du)--(25.745884\du,1.165501\du)--cycle;
\pgfsetlinewidth{0.070000\du}
\pgfsetdash{}{0pt}
\pgfsetdash{}{0pt}
\pgfsetmiterjoin
\definecolor{dialinecolor}{rgb}{0.000000, 0.000000, 0.000000}
\pgfsetstrokecolor{dialinecolor}
\draw (22.740884\du,1.165501\du)--(22.740884\du,3.082167\du)--(25.745884\du,3.082167\du)--(25.745884\du,1.165501\du)--cycle;
% setfont left to latex
\definecolor{dialinecolor}{rgb}{0.000000, 0.000000, 0.000000}
\pgfsetstrokecolor{dialinecolor}
\node at (24.243384\du,2.015501\du){Data};
% setfont left to latex
\definecolor{dialinecolor}{rgb}{0.000000, 0.000000, 0.000000}
\pgfsetstrokecolor{dialinecolor}
\node at (24.243384\du,2.438834\du){(M20-M22)};
% setfont left to latex
\definecolor{dialinecolor}{rgb}{0.000000, 0.000000, 0.000000}
\pgfsetstrokecolor{dialinecolor}
\node[anchor=west] at (20.405248\du,6.272762\du){};
\definecolor{dialinecolor}{rgb}{1.000000, 1.000000, 1.000000}
\pgfsetfillcolor{dialinecolor}
\fill (25.832183\du,5.157965\du)--(25.832183\du,7.497965\du)--(28.577183\du,7.497965\du)--(28.577183\du,5.157965\du)--cycle;
\pgfsetlinewidth{0.070000\du}
\pgfsetdash{}{0pt}
\pgfsetdash{}{0pt}
\pgfsetmiterjoin
\definecolor{dialinecolor}{rgb}{0.000000, 0.000000, 0.000000}
\pgfsetstrokecolor{dialinecolor}
\draw (25.832183\du,5.157965\du)--(25.832183\du,7.497965\du)--(28.577183\du,7.497965\du)--(28.577183\du,5.157965\du)--cycle;
% setfont left to latex
\definecolor{dialinecolor}{rgb}{0.000000, 0.000000, 0.000000}
\pgfsetstrokecolor{dialinecolor}
\node at (27.204683\du,6.007965\du){Spectrum};
% setfont left to latex
\definecolor{dialinecolor}{rgb}{0.000000, 0.000000, 0.000000}
\pgfsetstrokecolor{dialinecolor}
\node at (27.204683\du,6.431298\du){Image};
% setfont left to latex
\definecolor{dialinecolor}{rgb}{0.000000, 0.000000, 0.000000}
\pgfsetstrokecolor{dialinecolor}
\node at (27.204683\du,6.854631\du){(M22)};
\definecolor{dialinecolor}{rgb}{1.000000, 1.000000, 1.000000}
\pgfsetfillcolor{dialinecolor}
\fill (23.162389\du,5.157965\du)--(23.162389\du,7.074631\du)--(25.307389\du,7.074631\du)--(25.307389\du,5.157965\du)--cycle;
\pgfsetlinewidth{0.070000\du}
\pgfsetdash{}{0pt}
\pgfsetdash{}{0pt}
\pgfsetmiterjoin
\definecolor{dialinecolor}{rgb}{0.000000, 0.000000, 0.000000}
\pgfsetstrokecolor{dialinecolor}
\draw (23.162389\du,5.157965\du)--(23.162389\du,7.074631\du)--(25.307389\du,7.074631\du)--(25.307389\du,5.157965\du)--cycle;
% setfont left to latex
\definecolor{dialinecolor}{rgb}{0.000000, 0.000000, 0.000000}
\pgfsetstrokecolor{dialinecolor}
\node at (24.234889\du,6.007965\du){Image};
% setfont left to latex
\definecolor{dialinecolor}{rgb}{0.000000, 0.000000, 0.000000}
\pgfsetstrokecolor{dialinecolor}
\node at (24.234889\du,6.431298\du){(M21)};
\definecolor{dialinecolor}{rgb}{1.000000, 1.000000, 1.000000}
\pgfsetfillcolor{dialinecolor}
\fill (19.897778\du,5.157965\du)--(19.897778\du,7.074631\du)--(22.642778\du,7.074631\du)--(22.642778\du,5.157965\du)--cycle;
\pgfsetlinewidth{0.070000\du}
\pgfsetdash{}{0pt}
\pgfsetdash{}{0pt}
\pgfsetmiterjoin
\definecolor{dialinecolor}{rgb}{0.000000, 0.000000, 0.000000}
\pgfsetstrokecolor{dialinecolor}
\draw (19.897778\du,5.157965\du)--(19.897778\du,7.074631\du)--(22.642778\du,7.074631\du)--(22.642778\du,5.157965\du)--cycle;
% setfont left to latex
\definecolor{dialinecolor}{rgb}{0.000000, 0.000000, 0.000000}
\pgfsetstrokecolor{dialinecolor}
\node at (21.270278\du,6.007965\du){Spectrum};
% setfont left to latex
\definecolor{dialinecolor}{rgb}{0.000000, 0.000000, 0.000000}
\pgfsetstrokecolor{dialinecolor}
\node at (21.270278\du,6.431298\du){(M20)};
\pgfsetlinewidth{0.070000\du}
\pgfsetdash{}{0pt}
\pgfsetdash{}{0pt}
\pgfsetmiterjoin
\pgfsetbuttcap
{
\definecolor{dialinecolor}{rgb}{0.000000, 0.000000, 0.000000}
\pgfsetfillcolor{dialinecolor}
% was here!!!
\pgfsetarrowsend{stealth}
{\pgfsetcornersarced{\pgfpoint{0.000000\du}{0.000000\du}}\definecolor{dialinecolor}{rgb}{0.000000, 0.000000, 0.000000}
\pgfsetstrokecolor{dialinecolor}
\draw (24.243384\du,3.082167\du)--(24.243384\du,4.120066\du)--(21.270278\du,4.120066\du)--(21.270278\du,5.157965\du);
}}
\pgfsetlinewidth{0.070000\du}
\pgfsetdash{}{0pt}
\pgfsetdash{}{0pt}
\pgfsetmiterjoin
\pgfsetbuttcap
{
\definecolor{dialinecolor}{rgb}{0.000000, 0.000000, 0.000000}
\pgfsetfillcolor{dialinecolor}
% was here!!!
\pgfsetarrowsend{stealth}
{\pgfsetcornersarced{\pgfpoint{0.000000\du}{0.000000\du}}\definecolor{dialinecolor}{rgb}{0.000000, 0.000000, 0.000000}
\pgfsetstrokecolor{dialinecolor}
\draw (24.243384\du,3.082167\du)--(24.243384\du,4.120066\du)--(24.234889\du,4.120066\du)--(24.234889\du,5.157965\du);
}}
\pgfsetlinewidth{0.070000\du}
\pgfsetdash{}{0pt}
\pgfsetdash{}{0pt}
\pgfsetmiterjoin
\pgfsetbuttcap
{
\definecolor{dialinecolor}{rgb}{0.000000, 0.000000, 0.000000}
\pgfsetfillcolor{dialinecolor}
% was here!!!
\pgfsetarrowsend{stealth}
{\pgfsetcornersarced{\pgfpoint{0.000000\du}{0.000000\du}}\definecolor{dialinecolor}{rgb}{0.000000, 0.000000, 0.000000}
\pgfsetstrokecolor{dialinecolor}
\draw (24.243384\du,3.082167\du)--(24.243384\du,4.120066\du)--(27.204683\du,4.120066\du)--(27.204683\du,5.157965\du);
}}
% setfont left to latex
\definecolor{dialinecolor}{rgb}{0.000000, 0.000000, 0.000000}
\pgfsetstrokecolor{dialinecolor}
\node[anchor=west] at (21.270278\du,6.116298\du){};
% setfont left to latex
\definecolor{dialinecolor}{rgb}{0.000000, 0.000000, 0.000000}
\pgfsetstrokecolor{dialinecolor}
\node[anchor=west] at (21.270278\du,6.116298\du){};
% setfont left to latex
\definecolor{dialinecolor}{rgb}{0.000000, 0.000000, 0.000000}
\pgfsetstrokecolor{dialinecolor}
\node[anchor=west] at (24.243384\du,2.123834\du){};
\end{tikzpicture}

	\caption{Use Hierarchy Data}
	\label{figure:UHData}
\end{figure}


\newpage
\bibliographystyle {plainnat}
\bibliography {MG}

\end{document}